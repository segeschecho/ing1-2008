\chapter{Alcance de la soluci�n} 

El sistema se encarga de monitorear el ciclo de vida de los pedidos que se realizan a la pizzer�a y el stock de las materias primas. Es responsable tambi�n de la optimizaci�n del uso del horno, la estimaci�n de lapsos de producci�n y entrega, y por �ltimo de conservar un registro de los eventos de inter�s con fines estad�sticos. 

En el caso de los pedidos realizados remotamente (v�a Web o SMS), el sistema se encarga tambi�n de su ingreso. Para los pedidos realizados de forma directa (ya sea a un mesero o llamando por tel�fono) existir� un inviduo responsable de su ingreso.

Desde el momento del ingreso de los pedidos, el sistema verifica su factibilidad en funci�n del stock de insumos y de disponibilidad de productos preparados con antelaci�n. Una vez ingresados los pedidos, el sistema actualiza la informaci�n de stock inmediatamente (y at�micamente junto con la inserci�n del pedido). A continuaci�n se inserta el pedido en la cola (seg�n una pol�tica predefinida por el encargado) y se presenta a quien ingres� el pedido (usuario o encargado) una estimaci�n de la demora en la preparaci�n del pedido.

Dentro de la cocina, los maestros pizzero y empanadero tienen acceso al sistema en el cual pueden observar los pedidos que deben preparar e indicar su estado de preparaci�n (en espera, preparado, en el horno, listo). El sistema les indica tambi�n qu� sector del horno corresponde para la cocci�n del pedido. Tras la preparaci�n, si el pedido debe ser entregado el sistema contin�a monitoreando su estado y la demora en la entrega, tras la cual se recibe una notificaci�n por parte del delivery. Cuando corresponda, el sistema podr� (mediante una interfaz con el sistema de facturaci�n) realizar el cobro del pedido y registrar su pago.

En todo momento el encargado puede reordenar la cola de pedidos, o cancelar alguno de ellos, as� como los usuarios pueden consultar el estado de los mismos (v�a Web o SMS). Tambi�n se puede actualizar el stock de insumos y consultar las estad�sticas y algunos �ndices del rendimiento del proceso de producci�n y entrega.

El sistema no es responsable de la log�stica de la distribuci�n ni de la interacci�n directa con los clientes que no utilizan los medios Web o SMS. Tampoco existen operaciones automatizadas que afecten al negocio: todo lo que el sistema hace es coordinar a los actores que participan. El proceso de preparaci�n y cocci�n de los alimentos es totalmente manual, as� como lo es la reposici�n del stock y el armado de los pedidos para su posterior entrega.