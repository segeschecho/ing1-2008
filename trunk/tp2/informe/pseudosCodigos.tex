\begin{algorithm}[H]
\caption{Determina que meter en el horno segun politica normal}
\begin{algorithmic}[1]
\REQUIRE{hay a lo sumo un unico pedido a medio cocinar}
\ENSURE{se decide el proximo pedido a preparar segun lo especificado en el tp1 (salvando el hecho de que se vacia de a un modulo por vez)}
\IF{hay un pedido a medio cocinar}
\STATE productos a meter = tomar un grupo de productos del pedido para llenar un modulo (o lo mas que se pueda si no alcanza) \COMMENT{ver algoritmo de division del pedido}
\IF{eso era lo ultimo que quedaba por poner del pedido a medio cocinar}
\STATE poner pedido a medio cocinar = $\textcolor{ForestGreen}{\bot}$
\ENDIF
\RETURN productos a meter 
\ELSE
\STATE buscar en la cola el primer pedido
\IF{hay alguno}
\STATE productos a meter= tomar un grupo de productos del pedido sacado de la cola
\IF{queda algo de ese pedido por meter}
\STATE poner al pedido como pedido a medio cocinar
\ENDIF
\RETURN productos a meter
\ELSE
\RETURN $\textcolor{ForestGreen}{\bot}$
\ENDIF
\ENDIF
\end{algorithmic}
\end{algorithm}

\begin{algorithm}[H]
\caption{parte los productos de un pedido en grupos de productos que entran en un modulo}
\begin{algorithmic}[1]
\PARAMS{p:Pedido a partir}
\STATE res = $[]$
\STATE pizzas del pedido = [pr for p.productos if pr.tipoProd = Pizza]
\WHILE{ halla mas pizzas en pizzas del pedido que la cantidad de pizzas que entran en el modulo}
\STATE tomar A $\subset$ pizzas del pedido con $\sharp A ==$ pizzas que entran en el modulo
\STATE res += A
\STATE pizzas del pedido -= A
\ENDWHILE
\IF{ $\sharp$ pizzas del pedido $\neq$ 0}
\STATE res += pizzas del pedido
\ENDIF
\STATE empanadas del pedido = [pr for p.productos if pr.tipoProd = empanada]
\WHILE{ halla mas empanadas en empanadas del pedido que la cantidad de empanadas que entran en el modulo}
\STATE tomar A $\subset$ empanadas del pedido con $\sharp A ==$ empanadas que entran en el modulo
\STATE res += A
\STATE empanadas del pedido -= A
\ENDWHILE
\RETURN res
\end{algorithmic}
\end{algorithm}

\begin{algorithm}[H]
\caption{determina como llenar un modulo agil que se vacia}
\begin{algorithmic}[1]
\REQUIRE{hay a lo sumo un unico pedido a medio cocinar normal y uno chico a medio cocinar}
\ENSURE{se decide el proximo pedido a preparar segun lo especificado en el tp1 (salvando el hecho de que se vacia de a un modulo por vez)}
\IF{hay un pedido chico a medio cocinar}
\STATE productos a meter = tomar un grupo de productos del pedido chico para llenar un modulo (o lo mas que se pueda si no alcanza) \COMMENT{ver algoritmo de division del pedido}
\IF{eso era lo ultimo que quedaba por poner del pedido chico a medio cocinar}
\STATE poner pedido a chico a medio cocinar = $\textcolor{ForestGreen}{\bot}$
\ENDIF
\RETURN productos a meter
\ELSE
\STATE buscar en la cola el primer pedido chico
\IF{hay alguno}
\STATE productos a meter = tomar un grupo de productos que llenen un modulo
\IF{queda algo del pedido fuera del horno}
\STATE poner chico a medio cocinar = pedido
\ENDIF
\RETURN productos a meter
\ENDIF
\IF{hay algun producto en la cola}
\STATE tomar el primer pedido
\STATE productos a meter= tomar un grupo de productos del pedido sacado de la cola
\IF{queda algo de ese pedido por meter}
\STATE poner al pedido como pedido a medio cocinar
\ENDIF
\RETURN productos a meter
\ENDIF
\RETURN $\textcolor{ForestGreen}{\bot}$
\ENDIF
\end{algorithmic}
\end{algorithm}


\begin{algorithm}[H]
\caption{Estima el tiempo de terminacion de un pedido}
\begin{algorithmic}[1]
\PARAMS{p:Pedido cuyo tiempo se desea estimar}
\STATE listaPedidos = Pedido.allInstances()
\STATE tiempoCoccionPizzas = 0
\STATE tiempoCoccionEmpanadas = 0
\STATE tiempoPreparacion = 0
\FOR{cada pedido en lista}
\STATE\COMMENT{notar que en la lista esta tambi�n el mismo p}
\IF{el pedido esta en la cola del mismo horno que p}
\IF{el pedido esta en un estado anterior a preparado}
\FOR{cada producto del pedido}
\STATE tiempoPreparacion += tiempo de preparacion del producto
\ENDFOR
\ENDIF
\FOR{cada producto del pedido}
\IF{El producto es una pizza}
\STATE tiempoCoccionPizzas += tiempo de cocci�n del producto
\ENDIF
\IF{El producto es una empanada}
\STATE tiempoCoccionEmpanadas += tiempo de cocci�n del producto
\ENDIF
\ENDFOR
\ENDIF
\ENDFOR
\STATE tiempo de coccion = $\frac{ \frac{tiempoCoccionPizzas}{ pizzas por modulo del horno de p} + \frac{tiempoCoccionEmpanadas}{empanadas por modulo del horno de p}}{cantidad de modulos del horno de p}$
\STATE p.tiempoEstimado = tiempo de coccion + tiempoPreparacion
\end{algorithmic}
\end{algorithm}