\chapter{Conclusión}

El trabajo práctico nos permitió aplicar los principios y patrones de diseño
en un caso realista y de complejidad no trivial. Esto nos dio la oportunidad de
razonar sobre el diseño de los componentes de un sistema mediano, y sobre
las implicaciones que este diseño tiene sobre la flexibilidad, modificabilidad y
mantenibilidad de un sistema.

La cantidad sustancial de trabajo que requirió documentar, aún de forma 
incompleta, el diseño de un sistema, así como el razonamiento detrás del
mismo nos permite entrever las limitaciones propias del modelo en cascada.
Si bien parece factible realizar, con un esfuerzo importante, el diseño 
completo de un sistema utilizando las técnicas de UML vistas en clase,
simplemente no parece realista la modificación y actualización de un documento
de este tipo para reflejar los cambios que vayan sucediéndose en la medida
en que cambian los requerimientos o se agregan nuevos.

Por otra parte, la utilidad de una instancia de diseño en el momento
previo a la implementación de un sistema fue bastante aparente. Muchas
decisiones iniciales de diseño debieron ser revisadas para ajustarse a
nuevas necesidades en la medida que fuimos completando el trabajo.
Esta instancia además brinda una oportunidad para reflexionar sobre
los aspectos en los que es importante brindar flexibilidad, y aquellos
en los que resulta más conveniente optar por una implementación simple
y acotada.

\subsection{Dificultades en la realización del TP}

Al igual que con el trabajo práctico anterior enfrentamos varias
dificultades producto de la organización de los tiempos que sentimos
nos complicaron el desarrollo del TP.

Como en la entrega anterior, el plazo teórico de realización del TP se
vio fuertemente acotado por la escasa disponibilidad de clases de consulta.
En la práctica, el tiempo utilizable para llevar a cabo el trabajo fue
sustancialmente menor dado que pasamos gran parte del tiempo bloqueados
frente a dudas sobre criterios de corrección u otras decisiones de la
materia sobre lo que es aceptable y lo que no lo es.

Dada la naturaleza fuertemente dependiente de los diagramas de secuencia
respecto del diseño del sistema completo, nos fue imposible asignar de forma
uniforme el tiempo disponible. Debimos en cambio trabajar lentamente al
principio  hasta obtener un diagrama de clases aceptable y luego apurarnos
sobre el final para elaborar todos los diagramas de secuencia que se derivan
del mismo. Esto inevitablemente repercute en la calidad del trabajo final,
y prácticamente garantiza la existencia de inconsistencias de pequeña o 
mediana magnitud, ya que resulta imposible revisar todas las referencias
cuando se produce algún cambio.
