\chapter{Conclusi�n}

El trabajo pr�ctico nos permiti� aplicar los principios y patrones de dise�o
en un caso realista y de complejidad no trivial. Esto nos dio la oportunidad de
razonar sobre el dise�o de los componentes de un sistema mediano, y sobre
las implicaciones que este dise�o tiene sobre la flexibilidad, modificabilidad y
mantenibilidad de un sistema.

La cantidad sustancial de trabajo que requiria documentar, a�n de forma 
incompleta, el dise�o de un sistema, as� como el razonamiento detr�s del
mismo nos permite entrever las limitaciones propias del modelo en cascada.
Si bien parece factible realizar, con un esfuerzo importante, el dise�o 
completo de un sistema utilizando las t�cnicas de UML vistas en clase,
simplemente no parece realista la modificaci�n y actualizaci�n de un documento
de este tipo para reflejar los cambios que vayan sucedi�ndose en la medida
en que cambian los requerimientos o se agregan nuevos.

Por otra parte, la utilidad de una instancia de dise�o en el momento
previo a la implementaci�n de un sistema fue bastante aparente. Muchas
decisiones iniciales de dise�o debieron ser revisadas para ajustarse a
nuevas necesidades en la medida que fuimos completando el trabajo.
Esta instancia adem�s brinda una oportunidad para reflexionar sobre
los aspectos en los que es importante brindar flexibilidad, y aquellos
en los que resulta m�s conveniente optar por una implementaci�n simple
y acotada.

\subsection{Dificultades en la realizaci�n del TP}

Al igual que con el trabajo pr�ctico anterior enfrentamos varias
dificultades producto de la organizaci�n de los tiempos que sentimos
nos complicaron el desarrollo del TP.

Como en la entrega anterior, el plazo te�rico de realizaci�n del TP se
vio fuertemente acotado por la escasa disponibilidad de clases de consulta.
En la pr�ctica, el tiempo utilizable para llevar a cabo el trabajo fue
sustancialmente menor dado que pasamos gran parte del tiempo bloqueados
frente a dudas sobre criterios de correcci�n u otras decisiones de la
materia sobre lo que es aceptable y lo que no lo es.

Dada la naturaleza fuertemente dependiente de los diagramas de secuencia
respecto del dise�o del sistema completo, nos fue imposible asignar de forma
uniforme el tiempo disponible. Debimos en cambio trabajar lentamente al
principio  hasta obtener un diagrama de clases aceptable y luego apurarnos
sobre el final para elaborar todos los diagramas de secuencia que se derivan
del mismo. Esto inevitablemente repercute en la calidad del trabajo final,
y pr�cticamente garantiza la existencia de inconsistencias de peque�a o 
mediana magnitud, ya que resulta imposible revisar todas las referencias
cuando se produce alg�n cambio.
