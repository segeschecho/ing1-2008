%%	SECCION documentclass																									 %%	
%%---------------------------------------------------------------------------%%
\documentclass[a4paper]{report}

%%---------------------------------------------------------------------------%%
%%	SECCION usepackage																											 %%	
%%---------------------------------------------------------------------------%%
\usepackage{amsmath, amsthm}
\usepackage[spanish,activeacute]{babel}
\usepackage{caratula}
\usepackage{a4wide}
\usepackage{hyperref}
\usepackage{fancyhdr}
% \usepackage{moreverb}
\usepackage{graphicx} % Para el logo magico!
\usepackage{capt-of}
\usepackage{afterpage}
\usepackage{float}
\usepackage{amssymb}
\usepackage{amsmath}
\usepackage[latin1]{inputenc}
\usepackage{subfigure}
\usepackage[dvipsnames,usenames]{color}
\usepackage{amsfonts}
\usepackage{pdflscape}
\usepackage{booktabs}
\usepackage{colortbl}
\usepackage{tabularx}
\usepackage{ifthen}

%%---------------------------------------------------------------------------%%
%%	SECCION opciones																												 %%	
%%---------------------------------------------------------------------------%%
\parskip    = 11 pt
\headheight	= 13.1pt
\pagestyle	{fancy}
\definecolor{orange}{rgb}{1,0.5,0}

\addtolength{\headwidth}{1.0in}

\addtolength{\oddsidemargin}{-0.5in}
\addtolength{\textwidth}{1.0in}
\addtolength{\topmargin}{-0.5in}
\addtolength{\textheight}{0.7in}

%%---------------------------------------------------------------------------%%
%%	SECCION document	 %%	
%%---------------------------------------------------------------------------%%
\begin{document}
\renewcommand{\chaptername}{Parte }

%%---- Caratula -------------------------------------------------------------%%
\materia{Ingenier�a de Software I (2do cuatrimestre de 2008)}
\titulo{Trabajo Pr�ctico 1}

\integrante{Gonzalez, Emiliano}{426/06}{xjesse\_jamesx@hotmail.com}
\integrante{Gonzalez, Sergio}{481/06}{seges.ar@gmail.com}
\integrante{Mart'inez, Federico}{17/06}{federicoemartinez@gmail.com}
\integrante{Sainz-Tr�paga, Gonzalo}{454/06}{gonzalo@sainztrapaga.com.ar}
\grupo{Grupo 5}
\resumen{
Se presenta en este trabajo una especificaci�n completa de la soluci�n propuesta para el proyecto de software de
administraci�n de pizzer�a. En el mismo se presenta un panorama general as� como un an�lisis detallado del problema, 
y nuestra propuesta para su resoluci�n. En primer lugar se plantea una descripci�n general de la soluci�n, y a
continuaci�n se detallan algunos aspectos importantes haciendo uso de herramientas desarrolladas en clase como
diagramas de actividad, m�quinas de estado finito y otras.
}

\definecolor{light-gray}{gray}{0.9}

\newcommand{\clase}[4]{
\subsection{#1}
#2
\ifthenelse{\equal{#3}{}}{}
{\subsubsection{M�todos}
#3
} 
\ifthenelse{\equal{#4}{}}{}
{
\subsubsection{Atributos}
#4
}
}

% ----- Token list para las instrucciones ------------------------------------
\newtoks\oplist\oplist={}
% ----- Comando para que el usuario agregue operaciones del CU

\newcounter{PasoCu}
\setcounter{PasoCu}{1}

\newcommand{\op}[2]
{
\oplist=\expandafter{\the\oplist #1 & #2 \\ \hline}
\stepcounter{PasoCu}
}
\newcommand{\negrita}[1]{{\bf #1}}
\newcounter{casoUso}
\setcounter{casoUso}{1}

\definecolor{light-gray}{gray}{0.9}
\newcommand{\cu}[6]{ 
{\setlength{\arrayrulewidth}{1mm}

\begin{tabularx}{16cm}{|X|X|}
\hline
\multicolumn{2}{|>{\columncolor{Black}}l|}{\textcolor{White}{\negrita{Caso de Uso: #1}}} \\
\hline
\multicolumn{2}{|>{\columncolor{Black}}l|}{\textcolor{White}{\negrita{N�mero \thecasoUso}}} \\
\hline
\multicolumn{2}{|>{\columncolor{light-gray}}l|}{\negrita{Actores intervinientes: #2}} \\
\hline
\multicolumn{2}{|>{\columncolor{light-gray}}l|}{\negrita{Requerimientos relacionados: #3}} \\
\hline
\multicolumn{2}{|>{\columncolor{light-gray}}l|}{\negrita{Precondici�n: #4}} \\
\hline
\multicolumn{2}{|>{\columncolor{light-gray}}l|}{\negrita{Poscondici�n: #5}} \\
\hline
\multicolumn{1}{|>{\columncolor{light-gray}}X|}{\negrita{Descripcion:}} &
\multicolumn{1}{>{\columncolor{light-gray}}X|}{\negrita{#6}} \\
\hline
\multicolumn{1}{|>{\columncolor{light-gray}}X|}{\negrita{Curso normal}} &
\multicolumn{1}{>{\columncolor{light-gray}}X|}{\negrita{Curso alternativo}}\\
\hline
\the\oplist
\end{tabularx}
\stepcounter{casoUso}
}
\newtoks\oplist\oplist={}
}

% TOC, usa estilos locos
\maketitle
\pagestyle{empty}
{
\fancypagestyle{plain}
    {
    \fancyhead{}
    \fancyfoot{}
    \renewcommand{\headrulewidth}{0.0pt}
    } % clear header and footer of plain page because of ToC
\tableofcontents
}

\newpage
% arreglos los estilos para el resto del documento, y
% reseteo los numeros de pagina para que queden bien
\pagenumbering{arabic}
\fancypagestyle{plain} {
    \fancyhead[LO]{Gonzalez, Gonzalez, Mart�nez, Sainz-Tr�paga}
    \fancyhead[C]{}
    \fancyhead[RO]{P\'agina \thepage\ de \pageref{LastPage}}
    \fancyfoot{}
    \renewcommand{\headrulewidth}{0.4pt}
}
\pagestyle{plain}

\chapter{Modificaciones a la especificaci�n}

Al momento de realizar el dise�o, decidimos realizar ciertas modificaciones a la 
especificaci�n presentada en el informe anterior. A continuaci�n, explicaremos cu�les 
fueron estas modificaciones, cu�l fue la motivaci�n para realizarlas y qu� impacto tienen
en el sistema resultante.

\section{Identificaci�n individual de los m�dulos del horno}

Se decidi� llevar a cabo una modificaci�n que anteriormente hab�a sido considerada
una mejora a futuro: los m�dulos de los hornos ser�n identificables de forma �nica,
mientras que antes se los consideraba indistinguibles.

\subsection{Justificaci�n}
Ya en el trabajo anterior establecimos que no identificar los m�dulos del horno
conlleva un problema de usabilidad, ya que cuando un cocinero extrae comidas del horno
no puede indicarle al sistema r�pidamente que fue lo que sac�, sino que debe
indicar qu� productos estaban en ese m�dulo y el sistema deber� entonces reconocer qu�
pedido sali� del horno.

En esta situaci�n, si dos m�dulos tienen los mismos
items, hay que recurrir a una decisi�n heur�stica para determinar qu�
m�dulo corresponde a cada pedido (como por ejemplo, ``el que entr�
primero va a al pedido que ingres� antes''), pero esto podr�a resultar
en que los productos cocinados se asignen incorrectamente a los
pedidos cuando salen del horno, y esto hace que no funcione como se
espera la pol�tica de cola.

Por otra parte, si bien en pol�tica de cola normal no es indispensable
realizar la distinci�n, s� lo es en el caso de la pol�tica �gil de cola. Como
decidimos separar la pol�tica del mecanismo utilizado para llevarla a cabo,
resultaba razonable ofrecer a toda pol�tica de horno los medios necesarios
para funcionar. Esto involucrar�a una diferencia muy grande de funcionamiento
entre la pol�tica de cola normal y la pol�tica de cola �gil. Esto redunda
en c�digo m�s complejo y acoplado. Por otra parte, la identificaci�n
individual de los m�dulos representa un servicio minimalista y que es
razonable para muchas pol�ticas distintas que pudieran implementarse. En
funci�n de eso, consideramos que es mucho m�s extensible esta modalidad.

En particular, si no se desea distinguir m�dulos entre ellos, es necesario
en la pol�tica �gil distinguir dos ``categor�as'' de los mismos: �giles y
normales. La distinci�n individual de m�dulos permite al sistema hacer todo
tipo de categorizaci�n, y el cocinero solo debe indicar de qu� m�dulo se
trata (y no caracter�sticas \textit{ad hoc} a la pol�tica tales como si
el m�dulo es �gil o no).

\subsection{Impacto del cambio}
En esta secci�n realizaremos una revisi�n de que cambios acarrea a la 
operatoria la identificaci�n individual de los m�dulos.

A nivel de objetivos este cambio nos agrega un requerimiento nuevo, que 
consiste en mantener la informaci�n de los m�dulos. La figura \ref{objetivos} 
permite observar el fragmento del diagrama que se ve modificado por el cambio.

\begin{figure}[H]
\centering
\subfigure[Diagrama de objetivos original]{
\includegraphics[scale=0.3]{./figuras/objetivos_viejos.png} }
\subfigure[Diagrama de objetivos modificado]{
\includegraphics[scale=0.3]{./figuras/objetivos_nuevo.png}}
\label{objetivos}
\caption{Impacto en el modelo de objetivos}
\setcounter{subfigure}{0}
\end{figure}

A nivel del diagrama de contexto no se producen cambios produce un cambio mayor,
ya que las comunicaciones entre agentes se mantienen (si bien la informaci�n transmitida
es levemente distinta cuando el maestro de cocina se comunica con el sistema). En cambio, s� 
se genera un cambio en la descripci�n de los casos de uso relacionados con la cocci�n de los
productos. En particular, se modifican los casos de uso \textit{Indicando producto cocinado} y
\textit{Siendo informado de pr�ximo pedido a cocinar}. Se detalla a continuaci�n.

% Indicando producto cocinado
\op{1. El maestro indica al sistema que finaliz'o la cocci'on de ciertas partes de un pedido, seleccionando el m�dulo que desaloja}{}
\op{2. El sistema registra la parte como cocinada}{}
\op{3. El sistema verifica si la 'ultima parte cocinada completa el pedido}{}
\op{4. Si es as'i, el sistema registra al pedido como listo}{}
\op{5. Si hay productos para cocinar el sistema le informa al maestro que de debe poner a continuaci�n. EXTIENDE caso de uso Siendo informado de proximo producto a cocinar}{}
\op{6. Fin CU}{}
\cu{Indicando producto cocinado}{Maestro}{8, 11, 12, 15, 33}{True}{La parte de pedido se registra como cocinada}{El maestro, luego de cocinar una parte de un pedido, indica al sistema que la misma est'a cocinada}

% Siendo informado de proximo pedido a cocinar
\op{1. El sistema indica al maestro una parte a cocinar y qu� m�dulo libre le corresponde}{}
\op{2. Si es la primera parte de un pedido, el sistema cambia el estado del mismo a ``En Horno''}{}
\op{4. Fin CU}{}
\cu{Siendo informado de pr�ximo producto a cocinar}{Maestro}{8, 11, 12, 15, 33}{La cola del horno no est'a vac'ia}{La parte comienza a cocinarse}{El sistema le ordena al maestro que parte de pedido debe cocinar y en que m�dulo del horno}

Con respecto al funcionamiento del ingreso al horno, este es similar al funcionamiento anterior, 
pero ahora el maestro deber� indicar qu� modulo libera, y el mismo sistema se encargar� de deterinar
si el mismo era �gil o no. Esto es m�s razonable ya que dicha decisi�n puede ser realizada por
la pol�tica de cola, que es la entidad m�s id�nea para hacerlo.

%TODO: hacer 2 diagramas de actividad, uno cuando se libera modulo agil y otro cuando se libera un modulo no agil

\section{Estimaci�n de tiempos}
\label{modifEstim}

Se modific� el algoritmo de estimaci�n de tiempos de preparaci�n y cocci�n de pedidos
por uno m�s fiable, ya que se encontraron errores en el algoritmo propuesto en la
especificaci�n.

\subsection{Justificaci�n}
En la especificaci�n presentamos una operaci�n para realizar la estimaci�n que, si 
bien permit�a obtener una cota superior al tiempo necesario para terminar un pedido, resultaba
en muchos casos una estimaci�n muy grosera. En particular, el algoritmo consideraba al
horno como un proceso secuencial, mientras que �ste tiene la capacidad de cocinar muchos
productos en paralelo. Si bien la estimaci�n anterior da un alto grado de confianza en que
no se exceda el tiempo estimado, subestima muy fuertemente la capacidad de producci�n en la
cocina y en momentos de mucha ocupaci�n la cota superior puede tomar valores rid�culos.

Consideremos un ejemplo: La pizzer�a no tiene pedidos, y se realiza un pedido de 6 pizzas, 
donde cada pizza tarda 30 minutos en cocinarse y consideremos despreciable el tiempo de preparaci�n.
Supongamos tambi�n que en el horno entran 6 pizzas al mismo tiempo. En este caso una buena estimaci�n 
ser�a 30 minutos. Sin embargo, por como calculabamos la estimaci�n, la misma ser�a de 3 horas, lo
cual es excesivo al punto que es probable que el cliente cancele el pedido.

Por esta raz�n decidimos utilizar una nueva m�trica para calcular el tiempo estimado. La idea no 
es dar un tiempo exacto, sino corregir la sobreestimaci�n grosera que generaba la operaci�n anterior.
La nueva operaci�n de estimaci�n es la siguiente:

$$tiempoEstimado(p) = tiempoPreparacion(p) + \sum{tiempoPreparacion(ped)} + $$
$$ \frac{tiempoCoccionPizzasDe(p)+ \sum{tiempoCoccionPizzasDe(peds)}}{pizzasPorModulo*modulosPorHorno} + $$ 
$$ \frac{tiempoCoccionEmpanadasDe(p) + \sum{tiempoCoccionEmpanadasDe(peds)}}{EmpanadasPorModulo*modulosPorHorno}$$

donde $ped$ son los pedidos que est�n por ingresar o esperando prepararse, y $peds$ son los pedidos que est�n 
esperando por ingresar a cocina, prepararse o ingresar al mismo horno asignado a $p$.

Si bien el dise�o que presentamos a continuaci�n permite reemplazar f�cilmente el algoritmo
de estimaci�n de tiempos por otros m�s sofisticados y precisos, consideramos necesario corregir
este error para proveer al cliente de una opci�n razonable desde la entrada en producci�n del
sistema.

\subsection{Impacto del cambio}
Dado que la estimaci�n de tiempos es una operaci�n interna de la que lo �nico que observa
el usuario es el resultado, no involucra modificaciones en la especificaci�n del sistema.

\chapter{Clases}
\section{Diagrama de clases}
\begin{landscape}
\begin{figure}
\centering
\includegraphics[height=18cm]{./figuras/clases.png}
\end{figure}
\end{landscape}

\section{Explicaci�n de las clases}
\clase
{ABMproductos}
{Esta clase se encarga de realizar las altas, bajas y modificaciones de los productos}
{}{}

\clase
{ABMstock}
{Esta clase se encarga de realizar las altas, bajas y modificaciones de los insumos}
{}{}

\clase
{Agil}
{Especializaci�n de gestor de horno, permite aplicar la politica agil}
{}{}

\clase
{Aviso}
{Clase utilizada para realizar el \textit{callback} desde el gestor de horno hacia el despachador de preparaci�n, es la encargada de ejecutar el metodo del despachador que lo notifica de que se termino de preparar algo. Esta clase permite que el despachador no necesite saber quien le avisa, y por lo tanto permite que los preparadores no requieran de metodos diferenciados para avisar que se terminaron de preparar las pizzas, o las empanadas}
{}{}

\clase{Cliente}
{Esta clase representa a un cliente, conteniendo todos los datos del mismo.}
{}{}

\clase{ColaListos} %FIXME: nombre poco feliz
{Esta clase contiene a los pedidos que ya estan listos. Su responsabilidad es la de conocer a todos los que estan en este estado, a fin de que despachar un pedido se haga desde esta clase}
{}{}

\clase
{ControladorDeIngreso}
{Controla la cola de ingreso, la cual puede ser modificada por el encargado de pedidos. Cuando algun preparador queda libre, envia el proximo pedido a preparar}
{}{}

\clase
{ControladorCliente}
{El controlador de cliente tiene por responsabilidad encargarse de autentificar un usuario}
{}{}

\clase
{ControladorStock}
{El controlador de stock, tiene por responsabilidad chequear la disponibilidad de insumos al momento de un ingreso, asi como la de hacer el decremento del stock al ingresar un pedido, generando el aviso de stock critico en caso de ser necesario.}
{}{}

\clase
{CoordinadorDePedidos}
{El coordinador de pedidos se encarga de controlar el ingreso de pedidos, y su ciclo de vida fuera de la cocina}
{}{}

\clase
{DespachadorDePreparaci�n}
{Esta clase tiene por responsabilidad manejar la cola de pedidos que se estan preparando, recordemos que puede existir una cola de preparaci'on si hay pedidos mixtos a la espera de uno de los maestros. El controlador de ingresos distribuye los pedidos a los distintos preparadores y despacha cada pedido a su gestor de horno correspondiente cuando ya esta preparado.}
{}{}

\clase
{EstimadorDeTiempos}
{El estimador de tiempos, como lo dice su nombre, se encarga de estimar el tiempo de preparacion y cocci�n de un pedido}
{}{}

\clase
{GeneradorDePedidos}
{Esta clase se encarga de crear pedidos, creando pedidos de solo bebidas o con comida segun los productos}%FIXME: justificacion
{}{}

\clase
{GestorHorno}
{Clase abstracta que permite implementar diferentes politicas para el manejo del horno}
{}{}

\clase
{Insumo}
{Contiene la informaci�n de los distintos insumos de la pizzer�a}
{}{}

\clase
{Normal}
{Permite implementar la politica normal de manejo del horno}
{}{}

\clase
{Pedido}
{Contiene la informaci�n de cada pedido}
{}{}


\chapter{Modelado de escenarios}
A continuacion presentaremos diversos escenarios que permiten lograr un modelado del sistema basado en interacciones

\section{Ingreso de un pedido}
%TODO: escenario uno, ingresa un pedido de solo bebidas
Como primer escenario mostraremos lo que ocurre cuando ingresa un pedido al sistema y este esta formado por solo bebidas. En este caso el pedido va ser creado  y luego el contralador de preingreso se encargar� de pasarselo al manejador de listos, ya que no hay nada que hacer con este pedido. Asumimosen este escenario que el cliente ya se valido (o es NULL si es un pedido de cliente anonimo) y que hay stock sufiencte para armar el pedido.

En este escenario ademas no modelamos todo el proceso de creaci�n de un pedido, el cual ser� desarrollado posteriormente

Veamos el diagrama de secuencias relativo al escenario:
%\begin{landscape}
\begin{figure}[H]
\centering
\includegraphics[scale=0.35]{./figuras/ingresaBebida.png}
\end{figure}
%\end{landscape}


La operacion que determina si un pedido es preparable, es decir tiene que ser preparado o no, se puede modelar con el siguiente diagrama de secuencia:

\begin{figure}[H]
\centering
\includegraphics[scale=0.4]{./figuras/determinarPreparable.png}
\end{figure}
%FIXME: las lineas de return ultimo hay q hacerlas punteadas con por ej photoshop :p
Consideremos el caso donde el pedido que ingresa si era preparable y por lo tanto debe pasar a la cola de ingreso a la espera de ser preparado. En el siguiente escenario, veremos la secuencia que se desata desde el momento en que el pedido llega al controlador de preIngreso, hasta que es encolado y marcado como ingresado.

\begin{figure}[H]
\centering
\includegraphics[scale=0.4]{./figuras/ingresaComidaEncolar.png}
\end{figure}

Como vemos, cuando el pedido pasa al controlador de ingreso, este pregunta si el pedido puede empezar a prepararse ahora mismo o debe encolarlo. El no esta capacitado para decidir eso, ya que no sabe que ocurre con la preparacion de los pedidos. Notemos que tiene que preguntar, porque podr�a ocurrir que en la cola hallan muchos pedidos, por ejemplo todos de pizza, y por lo tanto el maestro empanadero este ocioso. Al preguntar lo que esta haciendo indirectamente es avisar que llego un pedido nuevo y dejando que el despachador de preparacion decida si esta listo para prepararlo.
%TODO: por que no un observer??

%TODO: ingresa un pedido mixto y queda en la cola de listos, tener en cuenta q la seleccion de horno se hace mientras es intentna ingresar

%TODO: generacion de un pedido
Anteriormente nos encargamos de mostrar como una vez que el pedido ya estaba armado, se comenzaba a recorrer su ciclo de vida. Sin embargo un aspecto previo a desarrollar es como se logra armar este pedido.

Al intentar armar un pedido, el generador de pedidos debe antes que nada utilizar al controlador de stock para verificar que la cantidad de insumos sea suficiente para satisfacer las necesidades del pedido. Esto es responsabilidad del controlador de stock, y para hacerlo se tiene el m�todo verificarEIngresar. Veamos un diagrama de secuencias para el caso en el que se quiere chequear si es viable o no ingresar ciertos productos para armar un pedido.

\begin{figure}[H]
\centering
\includegraphics[scale=0.4]{./figuras/verificarEIngresar.png}
\label{verifEIngr}
\end{figure}

Algunos de los metodos no estan desarrollados completamente en \ref{verifEIngr}, sino que las desarrollaremos por separado. Asi tenemos el caso de ingresar que lo que va a hacer es dado un producto, verificar si se tienen los insumos necesarios para hacer el ingreso. Ademas en caso de encontrar que alguno de los stocks queda con un valor menor a su cantidad critica, genera el aviso de stock critico.

\begin{figure}[H]
\centering
\includegraphics[scale=0.4]{./figuras/ingresar(controladorStock).png}
\end{figure}
%TODO: generacion del aviso
%TODO: ver que todos los metodos queden bien explicados, ya sea con pesudocodigo o con DS

\section{Iniciar sesion de cliente}

\section{aviso de stock critico}
A continuaci�n modelaremos escenarios relacionados con la creacion de un pedido, en donde pensamos mostrar como actuan el generador de pedidos, el ve
%TODO: no hay suficiente relleno de una empanada y de una pizza

\section{modificacion de la cola de pedidos}

%TODO: mover un pedido un lugar hacia arriba y otro hacia abajo

\section{Preparacion de pedidos}

%TODO: ingresa un pedido y no hay nada esperando preparandose tonces pasa a preparse

%TODO: se empieza a preparar un pedido mixto de la cola de ingreso

%TODO: maestro pide pedido a preparar y se le da un pedido que el otro maestro esta preparando

\section{Cocci�n de pedidos}

%TODO: se termina de prepara un pedido y no hay nada esperando

%TODO: se termina de prepara un pedido y se encola en su horno

%TODO: se termina de cocinar algo en un modulo agil y se toma un pedido chico

%TODO: se termina de cocinar algo en un modulo normal y se toma un pedido normal

%TODO: se termina de cocinar algo en un modulo y no hay nada a continuaci�n

%TODO: se saca un producto y se termino de cocinar

\section{Despachar pedido}

%TODO: se despacha un pedido remoto

%TODO: se despacha un pedido local

\section{Se cierra una mesa}

%TODO: se cierra una mesa y se registra la forma de pago

\section{Cancelacion}

%TODO: todos los escenarios de cancelacion

\section{Actualizar precios}
La actualizacion de los precios de pedidos se realiza de la siguiente manera. Al abrirse la pantalla de ABM de productos, se cargan todos los productos en pantalla. Luego se puede elegir un producto de la lista y un nuevo valor para el precio.

Vamos a modelar el escenario donde se cambia el precio de la pizza de muzzarella a \$30.

Notar que probablemente la GUI realice operaciones con la lista de productos que obtiene, por ejemplo cargar una lista en pantalla para mostrar los pedidos, etc. Sin embargo consideramos que los aspectos propios de la GUI no entran en lo que se debe modelar en este trabajo.

\begin{figure}[H]
\centering
\includegraphics[scale=0.7]{./figuras/modifPrecio.png}
\end{figure}


\section{Consulta de estado de pedido}
El escenario a modelar es el siguiente: El encargado de pedidos abre la gui para consultar el estado de un pedido, entonces la gui muestra los pedidos y cuando el encargado encuentra el que busca, pide ver el estado. El coordinador es el encargado de hacer de adaptador entre la GUI y los pedidos.

En el diagrama suponemos que se llama un metodo de la gui verEstado, esto en verdad es abrir la interfaz para ver los estados de pedidos, es decir no es un metodo como tal, pero sirve para dar idea de que alguien le dice a la GUI que busque los pedidos y consulte el estado.

\begin{figure}[H]
\centering
\includegraphics[scale=0.7]{./figuras/verEstado.png}
\end{figure}



\label{LastPage}
\end{document}
