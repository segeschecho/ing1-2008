 %%	SECCION documentclass																									 %%	
%%---------------------------------------------------------------------------%%
\documentclass[a4paper]{report}

%%---------------------------------------------------------------------------%%
%%	SECCION usepackage																											 %%	
%%---------------------------------------------------------------------------%%
\usepackage{amsmath, amsthm}
\usepackage[spanish,activeacute]{babel}
\usepackage{caratula}
\usepackage{a4wide}
\usepackage{hyperref}
\usepackage{fancyhdr}
% \usepackage{moreverb}
\usepackage{graphicx} % Para el logo magico!
\usepackage{capt-of}
\usepackage{afterpage}
\usepackage{float}
\usepackage{amssymb}
\usepackage{amsmath}
\usepackage[latin1]{inputenc}
\usepackage{subfigure}
\usepackage[dvipsnames,usenames]{color}
\usepackage{amsfonts}
\usepackage{pdflscape}
\usepackage{booktabs}
\usepackage{colortbl}
\usepackage{tabularx}
\usepackage{ifthen}
\usepackage{algorithm}
\usepackage{algorithmic}

%%---------------------------------------------------------------------------%%
%%	SECCION opciones																												 %%	
%%---------------------------------------------------------------------------%%
\parskip    = 11 pt
\headheight	= 13.1pt
\pagestyle	{fancy}
\definecolor{orange}{rgb}{1,0.5,0}

\addtolength{\headwidth}{1.0in}

\addtolength{\oddsidemargin}{-0.5in}
\addtolength{\textwidth}{1.0in}
\addtolength{\topmargin}{-0.5in}
\addtolength{\textheight}{0.7in}

%%---------------------------------------------------------------------------%%
%%	SECCION document	 %%	
%%---------------------------------------------------------------------------%%
\begin{document}
\input{trans_algorithmic.tex}
\renewcommand{\chaptername}{Parte }

%%---- Caratula -------------------------------------------------------------%%
\materia{Ingenier�a de Software I (2do cuatrimestre de 2008)}
\titulo{Trabajo Pr�ctico 2}

\integrante{Gonzalez, Emiliano}{426/06}{xjesse\_jamesx@hotmail.com}
\integrante{Gonzalez, Sergio}{481/06}{seges.ar@gmail.com}
\integrante{Mart'inez, Federico}{17/06}{federicoemartinez@gmail.com}
\integrante{Sainz-Tr�paga, Gonzalo}{454/06}{gonzalo@sainztrapaga.com.ar}
\grupo{Grupo 5}
\resumen{
Se presenta en este trabajo el dise�o completo de la soluci�n propuesta para el desarrollo de software
de gesti�n de pizzer�a. En el mismo se exhibe el diagrama de clases para el dise�o propuesto, as� como
el comentario y justificaci�n de las decisiones de dise�o tomadas. Asimismo, se incluyen diagramas
de secuencia donde se aprecian los pasajes de mensajes entre los objetos que componen al sistema en
escenarios diversos, y el pseudoc�digo de algunos algoritmos en los casos en que se consider�
pertinente.
}

\definecolor{light-gray}{gray}{0.9}

\newcommand{\clase}[4]{
\subsection{#1}
#2
\ifthenelse{\equal{#3}{}}{}
{\subsubsection{M�todos}
#3
} 
\ifthenelse{\equal{#4}{}}{}
{
\subsubsection{Atributos}
#4
}
}

% ----- Token list para las instrucciones ------------------------------------
\newtoks\oplist\oplist={}
% ----- Comando para que el usuario agregue operaciones del CU

\newcounter{PasoCu}
\setcounter{PasoCu}{1}

\newcommand{\op}[2]
{
\oplist=\expandafter{\the\oplist #1 & #2 \\ \hline}
\stepcounter{PasoCu}
}
\newcommand{\negrita}[1]{{\bf #1}}
\newcounter{casoUso}
\setcounter{casoUso}{1}

\definecolor{light-gray}{gray}{0.9}
\newcommand{\cu}[6]{ 
{\setlength{\arrayrulewidth}{1mm}

\begin{tabularx}{16cm}{|X|X|}
\hline
\multicolumn{2}{|>{\columncolor{Black}}l|}{\textcolor{White}{\negrita{Caso de Uso: #1}}} \\
\hline
\multicolumn{2}{|>{\columncolor{Black}}l|}{\textcolor{White}{\negrita{N�mero \thecasoUso}}} \\
\hline
\multicolumn{2}{|>{\columncolor{light-gray}}l|}{\negrita{Actores intervinientes: #2}} \\
\hline
\multicolumn{2}{|>{\columncolor{light-gray}}l|}{\negrita{Requerimientos relacionados: #3}} \\
\hline
\multicolumn{2}{|>{\columncolor{light-gray}}l|}{\negrita{Precondici�n: #4}} \\
\hline
\multicolumn{2}{|>{\columncolor{light-gray}}l|}{\negrita{Poscondici�n: #5}} \\
\hline
\multicolumn{1}{|>{\columncolor{light-gray}}X|}{\negrita{Descripcion:}} &
\multicolumn{1}{>{\columncolor{light-gray}}X|}{\negrita{#6}} \\
\hline
\multicolumn{1}{|>{\columncolor{light-gray}}X|}{\negrita{Curso normal}} &
\multicolumn{1}{>{\columncolor{light-gray}}X|}{\negrita{Curso alternativo}}\\
\hline
\the\oplist
\end{tabularx}
\stepcounter{casoUso}
}
\newtoks\oplist\oplist={}
}

% TOC, usa estilos locos
\maketitle
\pagestyle{empty}
{
\fancypagestyle{plain}
    {
    \fancyhead{}
    \fancyfoot{}
    \renewcommand{\headrulewidth}{0.0pt}
    } % clear header and footer of plain page because of ToC
\tableofcontents
}

\newpage
% arreglos los estilos para el resto del documento, y
% reseteo los numeros de pagina para que queden bien
\pagenumbering{arabic}
\fancypagestyle{plain} {
    \fancyhead[LO]{Gonzalez, Gonzalez, Mart�nez, Sainz-Tr�paga}
    \fancyhead[C]{}
    \fancyhead[RO]{P\'agina \thepage\ de \pageref{LastPage}}
    \fancyfoot{}
    \renewcommand{\headrulewidth}{0.4pt}
}
\pagestyle{plain}
\chapter{Introducci�n}

\section{Consideraciones sobre el informe}

El siguiente informe encompasa el dise�o que desarrollamos para la implementaci�n del
software de pizzer�a que fuera especificado en el Trabajo Pr�ctico 1. Nuestros objetivos
a la hora de dise�ar se enfocaron esencialmente en la flexibilidad y mantenibilidad del
software, con el objeto de que introducir cambios y correcciones en el software en
el futuro sea tan sencillo como sea posible.

En esta primera secci�n se describe brevemente el siguiente informe, as� como
su organizaci�n y contenidos y los objetivos que se plantea. En particular, este
documento est� dirigido a los desarrolladores que deber�n implementar el sistema.
Si bien no es de car�cter definitivo, el esp�ritu del documento intenta no ser
ambiguo y tener en cuenta tantas decisiones de dise�o significativas como sea
posible.

Al realizar el dise�o fue necesario en algunos casos introducir modificaciones o mejoras
a lo anteriormente especificado. Dichos cambios se evaluaron y fueron juzgados necesarios
por las ventajas que representan en varios planos, pero particularmente en la extensibilidad
y flexibilidad del sistema para tolerar cambios futuros en su modo de operaci�n. La segunda
secci�n del informe describe los cambios realizados as� como la justificaci�n de los mismos
y el impacto que tienen en el dise�o.

En tercer lugar se presenta la descripci�n del dise�o propiamente dicha. Como se adopt�
un acercamiento por componentes, se describe inicialmente esta divisi�n en componentes,
sus ventajas y limitaciones as� como la raz�n de utilizar esta modalidad. Enseguida se
presenta el dise�o de cada componente, as� como las relaciones que existen entre los
objetos de los mismos. Por �ltimo, se utiliza la herramienta de diagramas de secuencia
para modelar las comunicaciones entre los objetos (dentro de un mismo componente, o
entre varios de ellos). En los casos en que fue considerado necesario, se adjunta adem�s
el pseudoc�digo de algunos algoritmos de inter�s.

\section{Convenciones de notaci�n}

A lo largo de este documento se utilizan las siguientes convenciones:
\begin{itemize}
\item \textbf{DIP}: \textit{Dependency Inversion Principle}
\item \textbf{OCP}: \textit{Open/Closed Principle}
\item \textbf{SRP}: \textit{Single Responsibility Principle}
\item \textbf{LSP}: \textit{Liskov Substitution Principle}
\item \textbf{ISP}: \textit{Interface Segregation Principle}
\end{itemize}

En los diagramas de secuencia muchas veces la sintaxis no se ajusta exactamente a la descripta
en clase (por ejemplo, en el caso de las variables locales a un \textit{loop}). Esto se debe en parte
a limitaciones de la herramienta utilizada para realizar los diagramas, y en parte a que la sintaxis
propuesta no es UML y por lo tanto no est� soportada.

\chapter{Modificaciones a la especificaci�n}

Al momento de realizar el dise�o, decidimos realizar ciertas modificaciones a la 
especificaci�n presentada en el informe anterior. A continuaci�n, explicaremos cu�les 
fueron estas modificaciones, cu�l fue la motivaci�n para realizarlas y qu� impacto tienen
en el sistema resultante.

\section{Identificaci�n individual de los m�dulos del horno}

Se decidi� llevar a cabo una modificaci�n que anteriormente hab�a sido considerada
una mejora a futuro: los m�dulos de los hornos ser�n identificables de forma �nica,
mientras que antes se los consideraba indistinguibles.

\subsection{Justificaci�n}
Ya en el trabajo anterior establecimos que no identificar los m�dulos del horno
conlleva un problema de usabilidad, ya que cuando un cocinero extrae comidas del horno
no puede indicarle al sistema r�pidamente que fue lo que sac�, sino que debe
indicar qu� productos estaban en ese m�dulo y el sistema deber� entonces reconocer qu�
pedido sali� del horno.

En esta situaci�n, si dos m�dulos tienen los mismos
items, hay que recurrir a una decisi�n heur�stica para determinar qu�
m�dulo corresponde a cada pedido (como por ejemplo, ``el que entr�
primero va a al pedido que ingres� antes''), pero esto podr�a resultar
en que los productos cocinados se asignen incorrectamente a los
pedidos cuando salen del horno, y esto hace que no funcione como se
espera la pol�tica de cola.

Por otra parte, si bien en pol�tica de cola normal no es indispensable
realizar la distinci�n, s� lo es en el caso de la pol�tica �gil de cola. Como
decidimos separar la pol�tica del mecanismo utilizado para llevarla a cabo,
resultaba razonable ofrecer a toda pol�tica de horno los medios necesarios
para funcionar. Esto involucrar�a una diferencia muy grande de funcionamiento
entre la pol�tica de cola normal y la pol�tica de cola �gil. Esto redunda
en c�digo m�s complejo y acoplado. Por otra parte, la identificaci�n
individual de los m�dulos representa un servicio minimalista y que es
razonable para muchas pol�ticas distintas que pudieran implementarse. En
funci�n de eso, consideramos que es mucho m�s extensible esta modalidad.

En particular, si no se desea distinguir m�dulos entre ellos, es necesario
en la pol�tica �gil distinguir dos ``categor�as'' de los mismos: �giles y
normales. La distinci�n individual de m�dulos permite al sistema hacer todo
tipo de categorizaci�n, y el cocinero solo debe indicar de qu� m�dulo se
trata (y no caracter�sticas \textit{ad hoc} a la pol�tica tales como si
el m�dulo es �gil o no).

\subsection{Impacto del cambio}
En esta secci�n realizaremos una revisi�n de que cambios acarrea a la 
operatoria la identificaci�n individual de los m�dulos.

A nivel de objetivos este cambio nos agrega un requerimiento nuevo, que 
consiste en mantener la informaci�n de los m�dulos. La figura \ref{objetivos} 
permite observar el fragmento del diagrama que se ve modificado por el cambio.

\begin{figure}[H]
\centering
\subfigure[Diagrama de objetivos original]{
\includegraphics[scale=0.3]{./figuras/objetivos_viejos.png} }
\subfigure[Diagrama de objetivos modificado]{
\includegraphics[scale=0.3]{./figuras/objetivos_nuevo.png}}
\label{objetivos}
\caption{Impacto en el modelo de objetivos}
\setcounter{subfigure}{0}
\end{figure}

A nivel del diagrama de contexto no se producen cambios produce un cambio mayor,
ya que las comunicaciones entre agentes se mantienen (si bien la informaci�n transmitida
es levemente distinta cuando el maestro de cocina se comunica con el sistema). En cambio, s� 
se genera un cambio en la descripci�n de los casos de uso relacionados con la cocci�n de los
productos. En particular, se modifican los casos de uso \textit{Indicando producto cocinado} y
\textit{Siendo informado de pr�ximo pedido a cocinar}. Se detalla a continuaci�n.

% Indicando producto cocinado
\op{1. El maestro indica al sistema que finaliz'o la cocci'on de ciertas partes de un pedido, seleccionando el m�dulo que desaloja}{}
\op{2. El sistema registra la parte como cocinada}{}
\op{3. El sistema verifica si la 'ultima parte cocinada completa el pedido}{}
\op{4. Si es as'i, el sistema registra al pedido como listo}{}
\op{5. Si hay productos para cocinar el sistema le informa al maestro que de debe poner a continuaci�n. EXTIENDE caso de uso Siendo informado de proximo producto a cocinar}{}
\op{6. Fin CU}{}
\cu{Indicando producto cocinado}{Maestro}{8, 11, 12, 15, 33}{True}{La parte de pedido se registra como cocinada}{El maestro, luego de cocinar una parte de un pedido, indica al sistema que la misma est'a cocinada}

% Siendo informado de proximo pedido a cocinar
\op{1. El sistema indica al maestro una parte a cocinar y qu� m�dulo libre le corresponde}{}
\op{2. Si es la primera parte de un pedido, el sistema cambia el estado del mismo a ``En Horno''}{}
\op{4. Fin CU}{}
\cu{Siendo informado de pr�ximo producto a cocinar}{Maestro}{8, 11, 12, 15, 33}{La cola del horno no est'a vac'ia}{La parte comienza a cocinarse}{El sistema le ordena al maestro que parte de pedido debe cocinar y en que m�dulo del horno}

Con respecto al funcionamiento del ingreso al horno, este es similar al funcionamiento anterior, 
pero ahora el maestro deber� indicar qu� modulo libera, y el mismo sistema se encargar� de deterinar
si el mismo era �gil o no. Esto es m�s razonable ya que dicha decisi�n puede ser realizada por
la pol�tica de cola, que es la entidad m�s id�nea para hacerlo.

%TODO: hacer 2 diagramas de actividad, uno cuando se libera modulo agil y otro cuando se libera un modulo no agil

\section{Estimaci�n de tiempos}
\label{modifEstim}

Se modific� el algoritmo de estimaci�n de tiempos de preparaci�n y cocci�n de pedidos
por uno m�s fiable, ya que se encontraron errores en el algoritmo propuesto en la
especificaci�n.

\subsection{Justificaci�n}
En la especificaci�n presentamos una operaci�n para realizar la estimaci�n que, si 
bien permit�a obtener una cota superior al tiempo necesario para terminar un pedido, resultaba
en muchos casos una estimaci�n muy grosera. En particular, el algoritmo consideraba al
horno como un proceso secuencial, mientras que �ste tiene la capacidad de cocinar muchos
productos en paralelo. Si bien la estimaci�n anterior da un alto grado de confianza en que
no se exceda el tiempo estimado, subestima muy fuertemente la capacidad de producci�n en la
cocina y en momentos de mucha ocupaci�n la cota superior puede tomar valores rid�culos.

Consideremos un ejemplo: La pizzer�a no tiene pedidos, y se realiza un pedido de 6 pizzas, 
donde cada pizza tarda 30 minutos en cocinarse y consideremos despreciable el tiempo de preparaci�n.
Supongamos tambi�n que en el horno entran 6 pizzas al mismo tiempo. En este caso una buena estimaci�n 
ser�a 30 minutos. Sin embargo, por como calculabamos la estimaci�n, la misma ser�a de 3 horas, lo
cual es excesivo al punto que es probable que el cliente cancele el pedido.

Por esta raz�n decidimos utilizar una nueva m�trica para calcular el tiempo estimado. La idea no 
es dar un tiempo exacto, sino corregir la sobreestimaci�n grosera que generaba la operaci�n anterior.
La nueva operaci�n de estimaci�n es la siguiente:

$$tiempoEstimado(p) = tiempoPreparacion(p) + \sum{tiempoPreparacion(ped)} + $$
$$ \frac{tiempoCoccionPizzasDe(p)+ \sum{tiempoCoccionPizzasDe(peds)}}{pizzasPorModulo*modulosPorHorno} + $$ 
$$ \frac{tiempoCoccionEmpanadasDe(p) + \sum{tiempoCoccionEmpanadasDe(peds)}}{EmpanadasPorModulo*modulosPorHorno}$$

donde $ped$ son los pedidos que est�n por ingresar o esperando prepararse, y $peds$ son los pedidos que est�n 
esperando por ingresar a cocina, prepararse o ingresar al mismo horno asignado a $p$.

Si bien el dise�o que presentamos a continuaci�n permite reemplazar f�cilmente el algoritmo
de estimaci�n de tiempos por otros m�s sofisticados y precisos, consideramos necesario corregir
este error para proveer al cliente de una opci�n razonable desde la entrada en producci�n del
sistema.

\subsection{Impacto del cambio}
Dado que la estimaci�n de tiempos es una operaci�n interna de la que lo �nico que observa
el usuario es el resultado, no involucra modificaciones en la especificaci�n del sistema.

\chapter{Clases}
\section{Diagrama de clases}
\begin{landscape}
\begin{figure}
\centering
\includegraphics[height=18cm]{./figuras/clases.png}
\end{figure}
\end{landscape}

\section{Explicaci�n de las clases}
\clase
{ABMproductos}
{Esta clase se encarga de realizar las altas, bajas y modificaciones de los productos}
{}{}

\clase
{ABMstock}
{Esta clase se encarga de realizar las altas, bajas y modificaciones de los insumos}
{}{}

\clase
{Agil}
{Especializaci�n de gestor de horno, permite aplicar la politica agil}
{}{}

\clase
{Aviso}
{Clase utilizada para realizar el \textit{callback} desde el gestor de horno hacia el despachador de preparaci�n, es la encargada de ejecutar el metodo del despachador que lo notifica de que se termino de preparar algo. Esta clase permite que el despachador no necesite saber quien le avisa, y por lo tanto permite que los preparadores no requieran de metodos diferenciados para avisar que se terminaron de preparar las pizzas, o las empanadas}
{}{}

\clase{Cliente}
{Esta clase representa a un cliente, conteniendo todos los datos del mismo.}
{}{}

\clase{ColaListos} %FIXME: nombre poco feliz
{Esta clase contiene a los pedidos que ya estan listos. Su responsabilidad es la de conocer a todos los que estan en este estado, a fin de que despachar un pedido se haga desde esta clase}
{}{}

\clase
{ControladorDeIngreso}
{Controla la cola de ingreso, la cual puede ser modificada por el encargado de pedidos. Cuando algun preparador queda libre, envia el proximo pedido a preparar}
{}{}

\clase
{ControladorCliente}
{El controlador de cliente tiene por responsabilidad encargarse de autentificar un usuario}
{}{}

\clase
{ControladorStock}
{El controlador de stock, tiene por responsabilidad chequear la disponibilidad de insumos al momento de un ingreso, asi como la de hacer el decremento del stock al ingresar un pedido, generando el aviso de stock critico en caso de ser necesario.}
{}{}

\clase
{CoordinadorDePedidos}
{El coordinador de pedidos se encarga de controlar el ingreso de pedidos, y su ciclo de vida fuera de la cocina}
{}{}

\clase
{DespachadorDePreparaci�n}
{Esta clase tiene por responsabilidad manejar la cola de pedidos que se estan preparando, recordemos que puede existir una cola de preparaci'on si hay pedidos mixtos a la espera de uno de los maestros. El controlador de ingresos distribuye los pedidos a los distintos preparadores y despacha cada pedido a su gestor de horno correspondiente cuando ya esta preparado.}
{}{}

\clase
{EstimadorDeTiempos}
{El estimador de tiempos, como lo dice su nombre, se encarga de estimar el tiempo de preparacion y cocci�n de un pedido}
{}{}

\clase
{GeneradorDePedidos}
{Esta clase se encarga de crear pedidos, creando pedidos de solo bebidas o con comida segun los productos}%FIXME: justificacion
{}{}

\clase
{GestorHorno}
{Clase abstracta que permite implementar diferentes politicas para el manejo del horno}
{}{}

\clase
{Insumo}
{Contiene la informaci�n de los distintos insumos de la pizzer�a}
{}{}

\clase
{Normal}
{Permite implementar la politica normal de manejo del horno}
{}{}

\clase
{Pedido}
{Contiene la informaci�n de cada pedido}
{}{}


\chapter{Descripci�n de los componentes}
A continuaci�n se presenta una descripci�n de los componentes del modelo, mostrando sus interacciones tanto dentro de cada uno de ellos, como las que involucran a varios componentes.

Veamos por ejemplo un esquema del flujo de responsabilidades desde que entre un pedido con pizzas hasta que se notifica su entrega (figura \ref{flujoPed})

\begin{figure}[H]
\centering
\includegraphics[height=14cm]{./figuras/flujoPedido.png}
\caption{Flujo de la realizaci�n de un pedido}
\label{flujoPed}
\end{figure}

En este esquema podemos ver como la orden de ingresar un pedido entra por el componente de gestion de pedidos mediante el coordinador de pedidos, este lo pasa al componente de creacion de pedidos, invocando al generador. Este �ltimo distribuye las responsabilidades dentro del componente y lo devuelve al coordinador de pedidos.

De nuevo en el componente de gesti�n de pedidos, el controlador de preingreso decide a donde ubicarlo y lo pasa al controlador de ingreso, pues el pedido en el ejmplo, tiene pizzas. 

Cuando el pedido puede comenzar a prepararse cambia de componente pasando a la cocina. La entrada a la misma se da por el coordinador que se encarga de rutear al pedido, primero pasandoselo al despachador de preparaci�n. Cuando este termina vuelve a pasarlo de nuevo al coordinador, quien delega el pedido al despachador de cocci�n. 

Cuando la cocci�n termina, el despachador lo pasa al coordinador quien saca al pedido del ambito de la cocina entregandoselo al coordinador de pedidos.

El coordinador de pedidos delega el manejo del pedido al controlador de listos, quien cuando se despacha el pedido lo pasa al controlador de entregas.

Finalmente el pedido permanece bajo la orbita de este controlador hasta que se notifica su entrega y el pedido queda finalizado.

Como vemos, cada componente realiza un conjunto de funcionalidades acotado y necesita poder interactuar con los otros componentes.

A lo largo de este cap�tulo mostraremos mediante escenarios que ocurre dentro de cada componente y como se realizan las interacciones entre ellos.

\section{Pedidos fuera de la cocina}
Como dijimos este componente tiene por responsabilidad el manejo de la vida de los pedidos que estan fuera de la cocina (por cocina entendemos no solo al horno, sino tambi�n a la preparaci�n de los pedidos).

Nuestra idea fue tener un coordinadorDePedidos que siga el patron Fa�ade, de modo que muestre a la interfaz grafica una intrefaz \textit{``gruesa''} con las funciones que se realizan dentro del componente. Esta clase no va a tener gran inteligencia, sino que se limitara a propagar la llamada originada por la interfaz grafica a la clase responsable de manejar esa llamada. As�, por ejemplo para ver el estado de un pedido o ingrear un nuevo pedido se deber� pasar por esta clase.
La idea de esta clase es desacoplar la interfaz grafica de las clases que manejan a los pedidos fuera de la cocina. Si bien esta clase para tener baja cohesi�n, al mirarla de cerca vemos que lo �nico que hace es derivar las llamadas. De este modo si bien a trazo grueso parece tener una interfaz con poca cohesi�n, esta nos permite lograr un menor grado de acoplamiento entre la interfaz y el sistema en si. Por esta raz�n decidimos pese al aparente conflicto con la cohesi�n, decidimos mantener esta clase.
Ademas el coordinadorDePedidos se comunica con el componente encargado de crearPedidos, haciendo de puente entre ambos componentes.
Cuando se realiza un pedido de ingreso, el coordinador se encarga de pedir que se genere el pedido, y en caso de que se pueda crear, lo deriva al controladorDePreIngreso, cuya funci�n es determinar si el pedido debe ir a la cola de listos, porque no hay nada que preparar, ni cocinar, o lo tiene que mandar al controlador de ingreso, porque hay algo que cocinar o preparar. Este comportamiento se hizo con la intenci�n de permitir en un futuro incorporar otros productos ademas de pizzas o empanadas. Por ejemplo, podrian venderse ensaladas, las cuales no requieren de cocci�n, pero si de preparaci�n. Por eso decidimos que un producto tuviera atributos de cocinable y preparable.

El controladorDeIngreso se encarga de mantener la cola de ingreso, asi como de suministrar los pedidos al CoordinadorDeCocina, para que este los distribuya al preparador o al coordinador de horno.
El coordinadorDeIngreso puede recibir recibir una solicitud de un pedido de cierto tipo por parte del CoordinadorDeCocina, por ejemplo puede recibir una solicitud del proximo pedido que contenga algun producto del tipo empanada. 
Por otro lado, cada vez que ingresa un nuevo pedido, el controladorDeIngreso pregunta al CoordinadorDeCocina si puede recibir dicho pedido. Esta funcionalidad sirve para aquellos casos en los que por ejemplo no hay ningun pedido ingresado con empanadas y el maestro empanadero esta ocioso. Si llega un nuevo pedido con empanadas, el maestro debe ser notificado, por esta razon el ControladorDeIngreso pregunta si debe encolar el pedido o hay alguien que lo vaya a preparar.
%FIXME: la parte de preguntar es propia del standard, es decir del concreto

Otra clase de este componente, es el ControladorDeListos, este controlador va a recibir los pedido listos y va a encargarse en el momento del despacho de decidir que hacer con el pedido. Por ejemplo si es un pedido con delivery hay que marcar que salio con el delivery, y si era local hay que marcarlo como finalizado.

La clase controladorDeEntragas contiene a todos los pedidos cuya entrega esta pendiente, y se encargan de finalizar el pedido cuando se notifica la entrega.

Por ultimo el controladorPedidosMesa monitorea los pedidos entregados a una cierta mesa, permitiendo que al cerrar la mesa se fijen sus formas de pago.

La clase controladorDeIngreso es abstracta porque consideramos que la estrategia con la que se maneja la cola de ingreso podria cambiar, de esta manera la versi�n propuesta por nosotros en la etapa de especificaci�n es implementada por controladorDeIngresoStandard. Utilzar una clase abstracta nos permite lograr flexibilidad si se quiere cambiar de politica de manejo de esta cola, por ejemplo usando un manejo del tipo mas corto primero.


\textcolor{Red}{TODO: interacciones de estas clases con la GUI}

\textcolor{Red}{TODO: explicacion de metodos importantes}

\subsection{Modelado de escenarios}
\subsubsection{Aclaraci�n}
Algunas veces la sintaxis del diagrama no se ajusta del todo a la vista en clase, por ejemplo en el caso de las variables locales a un loop, esto se debe en parte a las limitaciones de la herramienta utilizada para realizar los diagramas (Quick Sequence Diagram Editor) y en parte a que algunas cuestiones de sintaxis, como la antes mencionada, no son propias de UML. No se utiliz� otra herramienta diferente debido principalmente, a que esta herramienta permite escribir el codigo de los diagramas en vez de arrastrar componentes, lo cual hace mas rapida la correcci�n y modificaci�n de los diagramas.

Esta aclaraci�n vale en general para todos los diagramas del presente documento.


%FIXME: en el ingreso se da a la gui la responsabilidad de mostrar los datos del pedido, tiempo estimado y precio, no se si eso esta bien
\subsubsection{Ingreso de un pedido de solo bebidas}
A continuaci�n intentaremos mostrar las interacciones existentes en este componente con el fin de modelar su comportamiento.
Como primer escenario veamos que ocurre cuando ingresa un pedido de solo bebidas Telefonico. En este caso el pedido ser� creado por el generador de pedidos, sin embargo las interacciones propias de la creaci�n no se detallaran en este escenario, asi como tampoco la validaci�n previa del cliente. Una vez que el pedido es creado, pasa al controlador de pre ingreso que lo examina para decidir si debe ir a la cocina o considerarse un pedido listo. En este caso, como solo hay bebidas, el pedido queda listo. El controlador de listos agrega el pedido a su lista de pedidos, se hace responsable del mismo y cambia su estado. 
Al agregar el pedido a la lista notifica a su observador de que ocurrio un cambio, por ejemplo para que se repinte la lista de pedidos listos.

\begin{figure}[H]
\centering
\includegraphics[height=21cm]{./figuras/remotoBebidas.png}
\caption{Ingreso de un pedido remoto de solo bebidas}
\end{figure}

El verifcarPreprable y su analogo para cocinable, basicamente recorren los productos del pedido, buscando si alguno tiene un tipo preparable o cocinable.

\begin{figure}[H]
\centering
\includegraphics[height=9cm]{./figuras/determinarPreparable.png}
\caption{VerificarPreparable}
\end{figure}

\begin{figure}[H]
\centering
\includegraphics[height=9cm]{./figuras/determinarCocinable.png}
\caption{VerificarCocinable}
\end{figure}
%FIXME: o tocamos la imagen con el inkscape o justifamos las cosas que son de notacion rara

\subsubsection{Ingreso de un pedido con comidas}
De forma analoga al escenario anterior, supongamos que se va a ingresar un pedido, pero en este caso, el pedido si ten�a comidas, por lo que el controlador de pre ingresos se lo va a mandar al de ingresos. Este intenta pasarlo a la cocina para ver si esta puede hacerse cargo del pedido. Esto es asi porque en la especificaci�n se pide que si entra un pedido y el maestro estaba ocioso, se le notifique que prepare el pedido ingresado. Como conocer si los maestros estan preparando algo, no es asunto de este controlador lo que decidimos es que lo pase hacia la cocina y espere respuesta de esta. Modelaremos los dos escenarios, primero el caso en el que la cocina le dice que no puede hacerse cargo y en segundo lugar el caso en el que la cocina si acepta el pedido.

En el primer caso, el controlador de ingresos se hace cargo del pedido, cambiando su estado, marcando el pedido como ingresado y agregandolo a la cola.

\begin{figure}[H]
\centering
\includegraphics[height=6cm]{./figuras/remotoComidas.png}
\caption{Ingreso de un pedido remoto con comida que queda encolado para su ingreso}
\end{figure}

En el segundo caso, como se va a hacer cargo la cocina, el controlador de ingreso no debe hacer nada cuando se regresa la llamada.

\begin{figure}[H]
\centering
\includegraphics[height=4cm]{./figuras/remotoComidasquedapreparando.png}
\caption{Ingreso de un pedido remoto con comida que pasa a estar preparando}
\end{figure}

\subsubsection{Despacho de un pedido}
Cuando el pedido esta cocinado es responsabilidad de el controlador de listos. Una vez que el pedido esta listo, se puede despachar. Despachar tiene una semantica diferente segun el origen del pedido, por eso es que el controlador posse un despachar para los subtipos remotos, otro para el pedido de mesa y finalmente un tercer despachar para pedidos de mostrador.

En el escenario en el cual el pedido a despachar es de origen remoto, el controlador de listos, lo que hace es sacarlo de la lista, marcarlo como que salio en entrega y avisar de esto al coordinadorDePedido para que avise al controlador de entrega. Ademas notifica a su observador del evento. Esto se hace con la intenci�n de que la gui se entere de que un pedido salio de esta lista y la refresque.

El controlador de entregas pone al pedido en su lista de pedidos, y se asigna como responsable de la cancelaci�n del mismo.

Veamos el diagrama de secuencia comenzando la misma con la llegada de un mensaje de despachar al coordinador de pedidos

\begin{figure}[H]
\centering
\includegraphics[height=7cm]{./figuras/despacharPedidoTelefono.png}
\caption{Despacho de un pedido telefonico}
\end{figure}

Otro escenario diferente lo constituye el despacho de un pedido de mesa. En este caso, el pedido deja la orbita del controlador de lista, para pasar al controlador de pedidos de mesa, el cual se encargara cuando se cierre la misma de asignar la forma de pago a los pedidos. Este encargado tambi�n se hace cargo de la cancelaci�n. Ambos controladores ademas notifican a su observador de los cambios en su lista.

El diagrama de secuencia es el siguiente:

\begin{figure}[H]
\centering
\includegraphics[height=7cm]{./figuras/despacharPedidoMesa.png}
\caption{Despacho de un pedido de mesa}
\end{figure}

Finalmente en el caso de un pedido de mostrador, el controlador de listos solo lo marca como terminado, ya que el pedido fue entregado y se conoce su forma de pago. Por lo tanto la secuencia es la siguiente:

\begin{figure}[H]
\centering
\includegraphics[height=8cm]{./figuras/despacharPedidoMostrador.png}
\caption{Despacho de un pedido de mostrador}
\end{figure}

\subsubsection{Pedido de proximo pedido a preparar}
Cuando el maestro termina de preparar un pedido (o sub pedido) se notifica al sistema. Entonces el despachador se encargado de pedirle al coordinador de la cocina que le consiga un pedido. Este habla con el controlador de ingreso para pedirle el pedido. Modelaremos el escenario en el cual el coordinador pide un pedido al controlador de ingreso. 

Para solicitar un pedido, se suministra al controlador de ingreso un TipoProducto, que le permite realizar la busqueda del primer pedido en la cola que contenga dicho tipo de producto. Esto es util si en un futuro se extienden los tipos de productos, o por ejemplo un maestro pasa a ser capaz de preparar otras cosas.

El controlador de ingreso entonces se encarga de buscar si en su cola hay alguien que cumpla tener algun producto del tipo solicitado. Si lo encuentra lo devuelve, sino devuelve NULL para informar que no tiene ningun pedido que contenga ese producto, por lo que el controlador quedara ocioso.

En el caso de tener que sacar un pedido de la cola, el controlador de listos hace un notify para invocar el update de su observador, por ejemplo para que se redibuje la cola de ingreso.

\begin{figure}[H]
\centering
\includegraphics[height=8cm]{./figuras/proximoPedido.png}
\caption{Despacho de un pedido de mostrador}
\end{figure}

%TODO: decidir aridad de esta funci�n
%TODO: mostrar pseudocodigo

\subsubsection{Notificaci�n de entrega}
Cuando el usuario notifica una entrega, selecciona el pedido de la lista de pedidos con entrega pendiente e indica que fue entregado. Al hacerlo, se notifica al coordinador de pedidos, que pasa la llamada al controlador de entregas. El mismo busca el pedido que debe marcar como finalizado, lo marca y setea en \verb0NULL0 el encargado de cancelaci�n, ya que el pedido no puede ser cancelado a partir de este momento. Luego notifica a su observador que un pedido salio de su cola, por ejemplo para que la GUI que muestra los pedidos con entrega pendiente se refresque.

El escenario donde se notifica la entrega de un pedido, puede modelarse con el siguiente diagrama de secuencia:

\begin{figure}[H]
\centering
\includegraphics[height=7cm]{./figuras/notificarEntrega.png}
\caption{Notificacion de pedido entregado}
\end{figure}

\subsubsection{Cerrado de mesa}
Para cerrar una mesa el usuario ingresa el numero de mesa que desea cerrar, elegiendo tambi�n la forma de pago. La GUI pasa el mensaje al coordinador de pedidos, el cual propaga la llamada hacia el controlador de mesa, que se va a encargar de completar la forma de pago de los pedidos de la mesa seg�n el parametro pasado y los quita de la lista de pedidos con entrega pendiente. Entonces notifica a su observador de que se modific� su lista de pedidos.

El siguiente diagrama nos permite modelar dicho escenario:

\begin{figure}[H]
\centering
\includegraphics[height=9cm]{./figuras/cerrarMesa.png}
\caption{Notificacion de pedido entregado}
\end{figure}

\subsubsection{Consulta de estado}
%TODO: que lo haga gonza

\subsubsection{Mover un pedido en la cola de ingreso}
%TODO: decidir si se entra por la facade o por el cont de ingreso
%TODO: hacer el subir y el bajar


%%FIXME: ver diagrmas porq gonza cambio el dc
%FIXME: asignacion de horno
\section{Creaci�n y registro de pedidos}
En este componente se agrupan las clases que entran en juego al momento 
de registrar y crear un nuevo pedido. Consideramos como parte de este
componente a la clase \textit{Pedido} propiamente dicha ya que este
componente es quien se encarga de construir sus instancias. A continuaci�n
se detalla el dise�o de cada uno de los componentes.

Este componente acopla con el componente de gesti�n de stock y con el de
gesti�n de clientes, puesto que es necesario que el ingreso de un pedido
implique modificaciones de stock, y que los pedidos se asignen a los clientes
correspondientes.

\subsection{Clase Pedido}

La clase Pedido consta de una estructura simple. Contiene, adem�s de los datos
sobre los items que componen al pedido, informaci�n sobre el cliente que lo realiz�
y otra metainformaci�n como el horario de ingreso o la modalidad que se utiliz�
para realizar el pedido (por tel�fono, en el local, etc).

Dicha modalidad no se almacena como un atributo sino que se indica mediante
la herencia de subclases de Pedido. La jerarqu�a de herencia, que puede apreciarse
en el diagrama, no es casual sino que responde a la necesidad de tratar de forma
diferenciada los diferentes tipos de pedido. Por ejemplo, los pedidos locales no 
deben ser entregados a domicilio, y �nicamente los pedidos que fueron hechos desde
una mesa tienen dicha informaci�n.

% TODO: Hacer un diagrama con la herencia de pedido

La raz�n por la que no se utiliz� un atributo es para servirnos del patr�n \textit{Visitor}
cuando es necesario implementar funcionalidad particular a solo un tipo de pedido. Si bien
hubiera sido posible utilizar implementaciones con condicionales que discriminen uno u otro
tipo de pedido, esto viola OCP y por lo tanto decidimos evitarlo.

Cabe destacar que la herencia propuesta respeta LSP puesto que la discriminaci�n en
subclases se hace principalmente por motivos funcionales. La �nica operaci�n que se
especializa es la correspondiente a la asignaci�n de mesas, pero el resto de las
cualidades de la clase padre se mantienen intactas en todas sus subclases.

La clase Pedido contiene un atributo Horno que lo relaciona con uno de los
Hornos de la cocina. Esto constituye una clara violaci�n de OCP, puesto que
la relaci�n arbitraria de un pedido con un Horno no es universal sino que m�s
bien es propia a la pol�tica actual de asignaci�n de hornos. Sin embargo, optamos
por realizarlo as� para evitar a�adir mucha complejidad al sistema.

\subsection{Otros miembros}

Ahora bien, para disponer de instancias de Pedido diferenciadas en clase, es necesario
que se tome la decisi�n en alg�n punto de cual ser� la clase para un cierto pedido. Para
esto nos valemos del patr�n \textit{Factory}, implementado por la interfaz GeneradorDePedidos,
que produce instancias de la clase apropiada a partir de la informaci�n de un pedido. Inevitablemente
el c�digo del \textit{Factory} viola OCP, pero lo hacemos de esta manera para luego evitar
discriminar por tipo en el resto del sistema, lo cual ser�a peor. Agregamos una implementaci�n
de dicha interfaz, GeneradorDePedidosStandard, que se ocupa de la operatoria definida al momento
de realizar el dise�o, y genera �nicamente instancias de las subclases de Pedido definidas hasta 
el momento. Sin embargo, es f�cil implementar un nuevo generador de pedidos que respete la interfaz
y soporte nuevos tipos de pedido.


La clase GeneradorDePedidos sirve de punto de entrada a este componente. La misma
posee el m�todo generarPedido, que es invocado por el Coordinador de Pedidos, a fin de que se 
ingrese al sistema un nuevo pedido. El Generador se encarga de llamar al ControladorDeStock para 
que verifique que el stock existente sea capaz de satisfacer al pedido. Adem�s, el generador indica
al controlador que realice el decremento del stock. Este a su vez podr�a notificar a la GUI si el stock
quedara por debajo de niveles cr�ticos.

A continuaci�n, el generador se encarga de llamar al EstimadorDeTiempos. Esta clase es abstracta, ya que 
pensamos que como la pizzer�a desea ir refinando estas estimaciones, es probable que la forma de estimar 
se modifique de forma peri�dica. Esto corresponde al patr�n \textit{Strategy}. La estimaci�n desarrollada 
en \ref{modifEstim} es implementada por la clase EstimadorBasico. Es de esperarse que estrategias de estimaci�n
m�s sofisticadas induzcan m�s acomplamiento para lograr mayor precisi�n, pero esto es inevitable.

Finalmente, el Calculador de Precios se encarga de obtener el precio a cobrar por el pedido. Nuevamente,
se implement� una interfaz gen�rica para permitir cambiar la implementaci�n. Implementaciones m�s
sofisticadas podr�an permitir realizar promociones si se compran determinados subconjuntos de productos,
o asignar descuentos a clientes preferenciales.

% TODO: Hablar del asignador de horno cuando se arregle

\subsection{Modelado de escenarios}

\subsubsection{Creaci�n de un pedido}
Cuando el coordinador de pedidos recibe la orden de crear un nuevo pedido, la deriva al generador 
de pedidos. Este se encargar� de devolverle un pedido nuevo. El generador de pedidos invoca al 
controlador del stock, para que verifique la factibilidad de ingresar el pedido. En caso de no 
ser posible, el generador producir� una excepci�n que ser� propagada para poder mostrar qu� 
producto no pudo ser ordenado.

En este escenario modelaremos el proceso de creaci�n de una forma general (usando como ejemplo
un pedido de mostrador) para luego mostrar escenarios particulares que pueden ocurrir durante 
este proceso.

En primer lugar, el generador invoca al controlador de stock para que realice el chequeo y 
en caso de ser posible decremente el stock. Luego, se genera un ID para el pedido y se crea
la instancia con el tipo correspondiente. Una vez creado el pedido, este pasa al asignador de 
horno que establece que horno le corresponde al pedido (en caso de que corresponda). Luego se procede 
a realizar la estimaci�n de tiempos de cocci�n y el c�lculo del precio. Finalmente, el pedido queda ingresado
y se delega al resto del sistema.

El diagrama de secuencias es el siguiente:

\begin{figure}[H]
\centering
\includegraphics[height=25cm]{./figuras/crearMostrador.png}
\caption{Creaci�n de un nuevo pedido}
\end{figure}

\subsubsection{Estimaci�n de tiempos}

% TODO: hacer diagrama ac�?
\textcolor{Red}{TODO: pseudocodigos que muestren algoritmos como por ej estimacion de tiempos}

\subsubsection{Verificaci�n de stock}

\textcolor{Red}{FIXME: esto va a la parte de stock}

% FIXME FIXME FIXME: todo esto tiene que volar, es cosa de la gesti�n de stock, no
% del ingreso de los pedidos, hay que moverlo y eventualmente rehacer estos diagramas
% pero sin tanto enfasis en el stock

El verificador de stock tiene por responsabilidad controlar que solo ingresen pedidos que puedan ser satisfechos. Ademas en caso de ser necesario debera notificar la existencia de insumos en stock critico.

Como vimos en el escenario anterior, el generador invoca el metodo verificarEIngresar. Este metodo va a intentar decrementar el stock de los insumos de cada producto del pedido que se desea armar. Para eso va a decrementar el stock siempre que sea posible, guardando aquellos productos cuyos insumos ya modifico para poder hacer rollback en caso de que el pedido no se pueda satisfacer. Si ocurre que hay un insumo de un producto cuyo insumo es insuficiente, se procede a restablecer el stock ya decrementado y luego se genera un excepcion que permite que se pueda mostrar en pantalla cual fue el pedido que genero el error al intentar ingresar.

En cambio si todos los productos se pudieron ingresar, la funci�n retorna True para indicar que termino exitosamente. Notar que hay un problema de sintaxis en el diagrama, ya que se hace delete del multiconjuinto pero su linea de vida se extiende. Esto es porque el programa utilizado para los diagramas no soporta eliminar dos veces al mismo elemento (en verdad es solo una vez, ya que ambas no pueden ocurrir, pero no se da cuenta de eso).

\begin{figure}[H]
\centering
\includegraphics[height=15cm]{./figuras/verificarEIngresar.png}
\caption{Verificaci�n y decremento de stock de los insumos}
\end{figure}

Decidimos factorizar el diagrama de modo que algunas interacciones las mostraremos a continuaci�n. El metodo ingresar realiza la verificaci�n pero a nivel de cada producto, es decir revisa dado un producto que exista una cantidad de insumos necesaria. Al igual que el metodo anterior va recordando los stocks que ya modifico para hacer rollback en caso de que sea necesario.

\begin{figure}[H]
\centering
\includegraphics[height=11cm]{./figuras/ingresar(ControladorStock)}
\end{figure}

Hay dos metodos restablecerStock, uno trabaja sobre productos y otro a nivel de insumos. El primero recorre los productos llamando al segundo para los insumos de cada producto que itera. Mientras que a nivel de insumos, lo que se hace es incrementar la cantidad de cada insumo que aparece en la lista. Como dijimos anteriormente, estos metodos permiten realizar un rollback para dehacer los cambios hechos en el stock en el caso de que la operaci�n de ingreso no sea exitosa

\begin{figure}[H]
\centering
\includegraphics[height=9cm]{./figuras/reestablecerStockProductos}
\caption{restableciendo el stock de los productos cuyo stock se decremento }
\end{figure}

\begin{figure}[H]
\centering
\includegraphics[height=9cm]{./figuras/reestablecerStockInsumos}
\caption{restableciendo el stock de los insumos que se decrementaron }
\end{figure}

{\color{Purple}
\subsubsection{asignar horno}
%TODO: estos items
\begin{itemize}
\item asignacion automatica
\item asignacion manual
\end{itemize}
}



%\section{Gesti�n de clientes}
Este componente es el responsable de validar clientes, para realizar las operaciones que requieren que se valide al usuario antes de proceder, como por ejemplo, ingresar un nuevo pedido telefonico. Por otro lado, este componente permite realizar el registro de nuevos usuarios en el sistema.

Basicamente son dos clases las relacionadas con este componente, la clase Cliente que modela la informaci�n necesaria de cada cliente registrado, asi como tambi�n permite crear nuevos clientes. 

Luego tenemos la clase ControladorCliente que se encarga de interactuar con la gui, validando los clientes segun distintos atributos.

\textcolor{Red}{TODO: interacciones de estas clases con la GUI}

\textcolor{Red}{TODO: explicacion de metodos importantes}
\subsection{Modelado de escenarios}
\textcolor{Red}{TODO: escenarios que muestren el comportamiento de estas clases en los fenomenos pedidos en el enunciado}


%\section{Cocina}
La cocina es el componente de mayor complejidad del sistema. Al igual que en el manejo de pedidos fuera de la cocina, tenemos una clase que sirve de punto de entrada y de controlador de flujo dentro de la cocina, derivando a los pedidos al despachador o controlador correspondiente. Esta clase es la que tiene contacto con el controlador de ingresos, de modo que todo pedido que quiere entrar en la cocina pasa por este coordinador. Cuando recibe un pedido, o pide un pedido, esta clase es la que decide quien debe hacerse cargo de recibir al pedido. Si consideramos el funcionamiento actual de la pizzer�a, donde los pedidos que llegan a la cocina son preparables y cocinables, al ingresar un pedido a la cocina, el coordinador lo va a enviar al despachador de preparacion y luego cuando este lo termine se lo enviar� al despachador de cocci�n. 

El despachador de preparaci�n es una clase abstracta que tiene por responsabilidad mantener la cola de pedidos que deben ser preparados, conocer que subpedidos se prepararon y notifiar cuando el pedido ya fue preparado. Decidimos que sea abstracta porque es factible considerar que hay diferentes formas de manejar que pedido de los que estan esperando debe preparse a continuaci�n. Ademas, la implementaci�n de estas funcionalidades va a estar acoplada fuertemente con los tipos de productos existentes y el manejo que se le de a los mismos. Por ejemplo, es razonable que como la pizzer�a solo maneja pizzas y empanadas, las cuales son preparadas por un unico maestro, el despachador divida a un pedido en solo estas dos partes, sin embargo si en el futuro se agregan ensaladas, el pedido tendria que ser dividio de otra manera. Entonces a fin de dar mayor extensibilidad decidimos hacer que esta clase sea abstracta. En particular el despachador que se comporta como lo mostrado en la especificaci�n es implementado por DespachadorDePreparaci�nEstandard. Esta clase que hereda del despachador de preparaci�n, sabe distribuir pizzas y empanadas a sendos preparadores.

Preparador es una interfaz que tiene como metodo principal preparar. La idea es que este metodo sea el que hable con la gui para mostrar que se debe preparar. Decidimos hacer una interfaz para esto, porque si bien en este momento se muestra todo el contenido del pedido (o subpedido a preparar), esta estraetgia podr�a cambiar, si por ejemplo se desea tener un contro de cada producto del pedido. Entonces nuestro preparador especializado que implementa esta interfaz funciona como lo planteamos en la especificaci�n.

La clase despachadorDeHorno es la responsable del manejo de las colas de ingreso a los hornos, aplicando la politica correspondiente. En principio habiamos considerado que era conveniente separar la aplicaci�n de la politica del mantenimiento de las colas, sin embargo dado que la aplicaci�n de la politica requiere de un acceso completo a las colas, nos pareci� acertado acoplar ambas funcionalidades. La clase es abstracta, siguiendo el \textit{strategy pattern} a fin de permitir que se implementen diferentes politicas de manera flexible.

La clase ControladorHorno es una abstraccion de los modulos del horno, esta clase permite poner algo en un modulo, asi como tambi�n sacar algo de un modulo, o conocer que es lo que hay en cada modulo. Cada controladorHorno posee ademas un fraccionador que sabe fraccionar un pedido en partes que entran en un modulo, contando para eso con un diccionario que dado un tipo de pedido pueda decidir cuantos productos de ese tipo entran en cada modulo.

\textcolor{Red}{TODO: interacciones de estas clases con la GUI}

\textcolor{Red}{TODO: explicacion de metodos importantes}
\color{Blue}{
\subsection{Modelado de escenarios}
\subsubsection{Ingreso de pedidos a preparar}
En nuestra especificaci�n del sistema consideramos que el aviso de preparaci�n se produciria de manera automatica siempre que el maestro este ocioso y llegue un nuevo pedido. Tambi�n se produce de forma automatica cuando el maestro termina y hay algun pedido potencialmente preparable por el.

Es por esta raz�n que el ingreso de un nuevo pedido genera que se intente poner a preparar el pedido, ya que el controlador de ingreso no conoce el estado de los preparadores. El despachador de preparaci�n recibe estos pedidos y decide si puede derivarlos o no, en cuyo caso deben quedar encolados en el controlador de ingreso.

A continuaci�n modelaremos algunos escenarios que pueden producirse en estos casos. Para los escenarios utilizaremos los pedidos mixtos, ya que los pedidos simples son un caso particular de estos ultimos.

En el primer escenario, consideraremos que el pedido ingresa pero ambos maestros estan ocupados por lo cual el pedido debe encolarse en la cola de ingreso. 

El despachador estandar lo que hace es revisar si un pedido es asignable al maestro empanadero, para que esto ocurra el pedido tiene que contener alguna empanada y ademas el maestro empanadero debe estar libre. Si no, no se puede. En este escenario el maestro esta ocupado por lo que no se puede asignarle el pedido. La situaci�n del pizzero es similar, asi que tampoco se le puede asignar el pedido. Ergo, el mismo se rechaza.

\begin{figure}[H]
\centering
\includegraphics[height=6cm]{./figuras/mandanPrepararMixtoYEmpiezanNada.png}
\caption{Solicitud de preparacion de un pedido mixto cuando ambos maestros estan ocupados}
\end{figure}

Otro caso es aquel en el que uno de los maestros si esta dispuesto a comenzar la preparaci�n del pedido. En este caso, el pedido se encola para el empanadero (porque era mixto) pero comienza a preparse para el pizzero. Al hacerlo es necesario guardar alguna informaci�n sobre el pedido. Por ejemplo registrar que es mixto y que por lo tanto antes de estar listo se deben preparar los dos subpedidos (empanadas y pizzas). Y recordar que ese es el pedido que el pizzero esta preparando. Luego se entregan los subproductos a preparar al preparador y se notifica mediante el return que el pedido no debe quedar en la cola de ingreso, sino en preparaci�n.

\begin{figure}[H]
\centering
\includegraphics[height=10cm]{./figuras/mandanPrepararMixtoYEmpiezanPizzas.png}
\caption{Solicitud de preparacion de un pedido mixto cuando el maestro pizzero esta disponible}
\end{figure}

Finalmente podria ocurrir que ambos maestros esten ociosos, porque no habia ningun pedido en la pizzeria esperando por ser preparado. Y al llegar un nuevo pedido ambos comiencen a preparalo. La situaci�n es similar al caso anterior, pero en este caso no se encola el pedido, sino que se notifica a ambos preparadores para que estos luego realicen la notificaci�n.

\begin{figure}[H]
\centering
\includegraphics[height=11cm]{./figuras/mandanPrepararMixtoYEmpiezanAmbos.png}
\caption{Solicitud de preparacion de un pedido mixto cuando ambos maestros estan disponibles}
\end{figure}

\subsubsection{Terminancion de preparaci�n}


\subsubsection{Ingreso de pedidos al horno}
Luego de ser preparados los pedidos pasan a traves del coordinador de cocina hacia el despachador de coccion. Este despachador es quien se encarga de manejar las colas de los hornos segun alguna politica. En particular en este trabajo consideraremos la politica normal y la politica agil, las cuales fueron debidamente especificadas en el trabajo anterior.

Basicamente ambas politicas, o despachadores tienen una estructura interna similar. Se utiliza un diccionario de numero de modulo a pedido que lo ocupa, para cada horno, hay dos colas de pedidos, pueden ser listas, ya que se por ejemplo se busca dentro de ellas, un diccionario que permite saber que partes faltan cocinar de un pedido y finalmente una variable de estado (2 en el caso de la politica agil) que permiten conocer que pedido esta a mitad de coccion, con elementos sin cocinar, y elementos cocinados o dentro del horno (la politica agil necesita 2 de estas variables ya que puede haber un pedido grande y un pedido chico en esta condicion).


%TODO: tal vez este no es el lugar para tantos detalles
Luego de terminar la preparacion de un pedido, el despachador invoca al coordinador, el cual a su vez llama al despachador de cocci�n. En un primer escenario a considerar, el pedido llega al despachador y como no hay lugar para entrar a su horno, se lo encola. Antes de encolarlo, se invoca al fraccionador, correspondiente al controlador del horno del pedido, para que separe al pedido en grupos de productos que se pueden colocar en un modulo. Es decir si el pedido es 3 pizzas y 3 empanadas y entran 2 empanadas por modulo y una pizza por modulo, una posible separacion de los productos del pedido es \{ 1 pizza, 1 pizza, 1 pizza, 2 empanadas, 1 empanada \}

En este escenario se asume que el horno 1 es el asignado al pedido.

%TODO: colocar el pseudocodigo del despachador

El diagrama de secuencia resultante es el siguiente:

\begin{figure}[H]
\centering
\includegraphics[height=6cm]{./figuras/llegaCocinarYseEncola.png}
\end{figure}

Otro escenario a considerar, comienza de igual manera que el anterior, pero esta vez si hay lugar en el horno para que el pedido entre. Lo que se hace es meter una parte del pedido en algun modulo libre, esto se repite mientras queden modulos libres o se termine el pedido. Por simplicidad en este escenario consideramos el caso donde solo habia un modulo libre. Una vez hecho eso, lo que ocurre es que se actualiza el estado interno. Por eso en este diagrama, al igual que en el anterior hay muchos automensajes, los cuales se deben a la gran cantidad de informaci�n interna que se necesita para mantener las politicas de acceso a los hornos.

\begin{figure}[H]
\centering
\includegraphics[height=6cm]{./figuras/llegaCocinarYNoseEncola.png}
\end{figure}

Queremos notar que estos escenarios aplican tambi�n para la politica agil, ya que los pedidos que llegan pueden quedarse encolados si no hay lugar (independientemente de si son agiles o no) y por otro lado pueden entrar si hay lugar libre, pero si hay lugar libre entran porque la cola es en esencia \textit{FIFO} en estas circunstancias, es decir si hay lugar vacio y solo hay un pedido, como se busca maximizar el uso del horno, se ingresa al mismo mas alla de su condicion de chico o grande. Nada impide que mas adelante se implemente una nueva politica que trate de hacer por ejemplo mas corto primero con conocimiento futuro, de modo que, por ejemplo use una cierta probabilidad $p$ para decidir si pone al pedido recien llegado al horno o no.

\subsubsection{Fin de coccion de una parte}
%FIXME: diagramas obsoletos porq cambio el md
Hasta ahora solo consideramos cuando los pedidos entran al despachador provenientes de la etapa de prepraci�n. Consideremos entonces que ocurre cuando se notifica la terminaci�n de la cocci�n de una parte que estaba en el horno.

Supongamos una politica normal, y que la parte del pedido que sale no es lo ultimo que quedaba por cocinarse. Para determinar si un pedido se termino de cocinar, se pregunta si queda alguna parte por cocinar y si hay alguna en el horno, si ambas respuestas son negativas, se termino de cocinar. Notar que no se esta intentando saber que partes fueron cocinadas, sino cuantas faltan. Otra nueva politica podria interesarse en tener un control mas estricto, por ejemplo para informar de forma mas precisa el estado de un pedido que es al horno.

La notificaci�n de que una parte se termino de preparar la recibe el controlador de horno, el cual recibe que modulo se liber�. El controlador pasa el mensaje al despachador, el cual actualiza su estado en funci�n de este evento y busca si puede poner a cocinar algo.

Lo que hace el despachador normal es buscar si hay un pedido a medio cocinar, es decir con partes sin cocinar y fuera del horno. Si hay uno se toma una parte de ese. Podria ocurrir que al pedido a medio cocinar solo le quedaba esa parte por cocinar, en cuyo caso ahora no hay ningun pedido a medio cocinarse en ese horno.�

Si no habia ninguno a medio preparar toma el proximo pedido de la cola y busca alguna de sus partes para enviar al controlador de horno. Este nuevo pedido podr�a pasar a ser el nuevo pedido a medio cocinar.

Si no habia ningun pedido en la cola tampoco, no tiene nada que hacer luego de actualizar su estado.

\begin{figure}[H]
\centering
\includegraphics[height=18cm]{./figuras/saleUnPedidoNormalNoTermina.png}
\end{figure}

En el escenario donde la politica es agil, la situaci�n es similar, en particular en el caso de que el modulo que se vacia no es agil, es el mismo procedimiento. Si el modulo que se vacia si es agil, el funcionamiento se modifica. En vez de intentar meter el pedido a medio cocinar se busca el pedido chico a medio cocinar. Si no hay, lo que se hace es buscar si hay algun chico en la cola, cuando lo encuentra el proceso es similar al escenario anterior, solo que este pedido puede llegar a ser el chico a medio cocinar. Si tampoco lo encuentra, busca como si fuera una politica normal. Esto nos permite ver como la politica agil, frente a la ausencia de pedidos chicos, es exactamente igual a la politica normal.

\begin{figure}[H]
\centering
\includegraphics[height=18cm]{./figuras/saleUnPedidoNormalNoTermina.png}
\end{figure}

\subsubsection{Algo termina de cocionarse}

\textcolor{Red}{TODO: escenarios que muestren el comportamiento de estas clases en los fenomenos pedidos en el enunciado}

\textcolor{Red}{TODO: pseudocodigos que muestren algoritmos como por ej seleccion de proximo pedido a cocinar}


%\section{Gesti�n de stock y productos}

Este componente se encarga de la gesti�n de los modelos correspondientes a
productos, tipos de procuto, insumos y las operaciones de alta y baja de los
mismos, as� como de las actualizaciones de sus valores.

Mediante la funcionalidad de ABM provista como se describe en \ref{metodosEstaticos}
es posible realizar los cambios necesarios al stock, tales como modificaciones de
precios o el ingreso de nuevos productos.

El Controlador de Stock es una entidad abstracta que se encarga de las modificaciones
de los valores de stock frente al ingreso o cancelaci�n de pedidos. Esta abstracci�n
se introduce para respetar DIP, y puesto que podr�a ser interesante brindar funcionalidad
m�s sofisticada en esta entidad. Por ejemplo, un controlador de stock m�s inteligente podr�a
realizar un seguimiento individual de todas las transacciones de stock llevadas a cabo
que permitir�a \textit{trackear} cada insumo.

El Controlador de Stock es responsable de emitir el evento de notificaci�n de stock
cr�tico, al que la GUI se suscribe para poder indicar al usuario que el stock requiere
de su atenci�n.

Las clases principales en lo que a modelos de datos respecta dentro de este componente
son Insumo y Producto. La clase TipoProducto permite reconocer cuando varios productos
tienen el mismo tipo y por tanto pueden tratarse de forma an�loga en cuanto a su
preparaci�n y cocci�n. 

El tipo de producto permite especificar adem�s si el producto es
cocinable o preparable. Si bien en el sistema actual los productos son ya sea preparable
y cocinables o ninguno de los dos, en el futuro la pizzer�a podr�a desear, por ejemplo,
comercializar ensaladas que solo requieren de preparaci�n y no de cocci�n. La existencia
de estos atributos permite flexibilidad adicional para extender la operatoria del
restaurante. Si bien esta elecci�n de atributos puede parecer limitante o arbitraria,
evaluamos que es factible categorizar de esta manera a cualquier tipo de producto
que podr�a venderse en una pizzer�a. Por lo tanto, no nos pareci� razonable agregar
complejidad al modelo agregando ``propiedades'' gen�ricas a los tipos de producto
(propiedades de las que \textit{cocinable} y \textit{preparable} ser�an un caso particular).

% TODO: hablar del repositorStock


\subsection{Modelado de escenarios}
% TODO: traer los diagramas de la parte de ingreso de pedidos

\textcolor{Red}{TODO: explicacion de metodos importantes}
% FIXME: vale la pena? creo que co los DS que estan en la parte de ingreso de pedidos alcanza y sobra


%\section{Cancelaci�n}

La cancelaci�n de un pedido es un evento que puede producirse en cualquier momento de la
vida de un pedido hasta que pasa al estado finalizado. Como vimos en las secciones anteriores,
los pedidos pasan por los distintos componentes a lo largo de su estado de vida.

Para llevar a cabo una cancelaci�n, no es suficiente con cambiar el estado del pedido
a ``Cancelado'', sino que adem�s es necesario que el componente que est� manipulando
el pedido al momento de la cancelaci�n modifique su estado para reflejar este cambio.
Por ejemplo, un controlador de cola de horno debe retirar los elementos del pedido de las
colas correspondientes si se produce una cancelaci�n. Por esta raz�n, la cancelaci�n
afecta de alguna manera a todos los componentes del sistema, y es el �nico fen�meno
verdaderamente \textit{transcomponente} que debimos considerar.

Consideremos el esquema \ref{diag_componentes}, podemos ver en este diagrama como la cancelaci�n involucra a varios componentes. El siguiente esquema intenta mostrar esta idea:

\begin{figure}[H]
\centering
\includegraphics[height=10cm]{./figuras/conCancelacion.png}
\caption{Diagrama de componentes l�gicos con el evento de cancelaci�n}
\end{figure}

Es necesario presentar a la GUI una interfaz sencilla para permitir la cancelaci�n de un
pedido. En funci�n de esto, la GUI permite al usuario encontrar el pedido que desea cancelar
(utilizando alguno de los m�todos de b�squeda) y luego env�a un mensaje a una entidad id�nea
para realizar la tarea.
%FIXME: que onda con esto de los metodos de busqueda

En este punto nos encontramos con una dificultad particular: �Cu�l es esta entidad? Es claro
que no puede elegirse arbitrariamente alguno de los componentes del sistema puesto que no
es f�cil determinar cu�l de ellos est� ocup�ndose del pedido en un momento dado. Consideramos 
entonces varias opciones.

En primer lugar evaluamos agregar un m�todo \verb0cancelar(Pedido)0 en el CoordinadorDePedidos,
\textit{fa�ade} principal del sistema, que luego se encargue de determinar al componente
responsable de la cancelaci�n y le env�e la notificaci�n correspondiente. A su vez, para determinar
el responsable, consideramos dos estrategias:
\begin{itemize}
\item Propagar el mensaje desde el coordinador a todas las dem�s entidades, en una estrategia
      similar al \textit{flooding} en dispositivos de red. Esto tiene el problema de que no es
      limpio y no es f�cil determinar qu� fue lo que ocurri� con la cancelaci�n desde el
      coordinador.
\item Definir una entidad que sea capaz de, a partir de la informaci�n de un pedido, determinar
      qui�n es el responsable en ese momento y enviarle un mensaje. Esto tiene el problema de que
      est� fuertemente acoplado a la operatoria actual del sistema, y por lo tanto viola OCP
      ya que si se desean agregar nuevos tipos de pedido, ser� necesario modificar esta
      funcionalidad para que refleje las nuevas posibilidades. Para apegarse a SRP, esta funcionalidad
      ameritar�a un objeto nuevo, pero esto no evita el acoplamiento que se introduce y la
      disminuci�n de cohesi�n que se deriva.
\end{itemize}

Puesto que ninguna de estas opciones result� satisfactoria, investigamos un poco m�s
y finalmente elegimos una opci�n que se asemeja al patr�n \textit{Observer}. Apeg�ndonos
a ISP, definimos una interfaz que llamamos \textbf{Cancelador} y que indica que la 
entidad que la implementa es capaz de cancelar un pedido. Le asignamos a su vez a
cada pedido un \textbf{responsable}, que no es m�s que una clase que implementa
la interfaz Cancelador. Cuando un componente del sistema toma el control de un pedido,
se registra como responsable del mismo. Cuando el pedido reciba una cancelaci�n,
notificar� al responsable para que este lleve a cabo la cancelaci�n.

% TODO: si se van a pintar de algun color las clases que implementan Cancelador
% en el diagrama, aclararlo ac�.
Las clases que implementan la interfaz Cancelador (y por tanto pueden
hacerse responsables de un pedido) son:
\begin{itemize}
\item ControladorEntregas
\item ControladorIngreso
\item ControladorListos
\item DespachadorPreparacion
\item DespachadorDeCoccion
\item Preparador
\end{itemize}

Esta soluci�n es considerablemente limpia y permite f�cilmente extender el sistema
sin realizar modificaciones innecesarias. Veamos a continuaci�n algunos escenarios diferentes 
de cancelaci�n.

\subsection{Modelado de escenarios}
\subsubsection{Cancelaci�n de un pedido ingresado}

El primer escenario consiste en cancelar un pedido que estaba en la cola de 
ingreso. En este caso se debe sacar de dicha cola al pedido y se debe reestablecer 
el stock de los insumos que no se van a utilizar.

\begin{figure}[H]
\centering
\includegraphics[height=8cm]{./figuras/cancelacionIngreso.png}
\caption{Cancelaci�n de un pedido ingresado}
\end{figure}

\subsubsection{Cancelaci�n de un pedido en preparaci�n}

En segundo lugar podemos considerar que ocurre cuando se cancela un pedido que 
estaba en preparaci�n. En este caso puede ocurrir que el pedido se estuviera 
preparando en el momento de su cancelaci�n, por lo que debe indicarse que se 
detenga la preparaci�n y que se indiquen qu� insumos se pueden reponer al stock.

Si la cancelaci�n se hace para un pedido que estaba en la cola nada m�s, el 
proceso es muy simple, solo se desencola esta parte que estaba en espera y se corrige
el estado del Pedido. A continuaci�n modelaremos un escenario m�s interesante, que corresponde
a la cancelaci�n de un pedido que ten�a sus empanadas siendo preparadas, asumiendo 
que hay otro pedido con empanadas en la cola esperando para ser preparados.

\begin{figure}[H]
\centering
\includegraphics[height=11cm]{./figuras/cancelacionPreparacion.png}
\caption{Cancelaci�n de un pedido en preparaci�n}
\end{figure}

\subsubsection{Cancelaci�n de un pedido en cocci�n}

Otro lugar donde la cancelaci�n es conflictiva es durante la cocci�n de un pedido.
Si el mismo estaba en la cola del horno, el proceso es simple porque solo hay que 
sacarlo de la misma. Por otra parte, si el pedido estaba en el horno o estaba a medio cocinar, 
hay que avisar al maestro para que retire el pedido del horno y corregir el estado de los hornos.

Vamos a modelar el escenario en el que el pedido que se cancela era un pedido a medio 
cocinar del horno 1. Entonces, hay que avisar su cancelaci�n, y buscar un nuevo pedido 
para poner en su lugar. Vamos a suponer tambi�n que el pedido tenia dos partes en cocci�n, 
que no estaban en m�dulos �giles y que el pr�ximo pedido de la cola necesitaba de mas 
de 2 modulos para cocinarse.

\begin{figure}[H]
\centering
\includegraphics[height=19cm]{./figuras/cancelacionCoccion.png}
\caption{Cancelaci�n de un pedido en cocci�n}
\end{figure}

el m�todo getModulosPorPedido recorre el diccionario que guarda que pedido esta en cada m�dulo del horno del pedido para obtener en cuales se estaba cocinando el pedido, a fin de saber cuantos y cuales son los modulos que deben vaciarse y llenarse.

\begin{algorithm}[H]
\caption{Busca los modulos en los que se encuentra el pedido pe}
\begin{algorithmic}[1]
\PARAMS{p un pedido}
\STATE obtener la cantidad c de modulos del horno
\STATE modulo actual = primer modulo del horno de p
\WHILE{mientras hayan modulos por recorrer hacer}
\STATE obtener el pedido ped que se encuentra en el modulo i usando el diccionario
\IF{ped es p entonces}
\STATE agregar modulo actual en res
\ENDIF
\STATE modulo actual = siguiente modulo
\ENDWHILE
\RETURN res
\end{algorithmic}
\end{algorithm}

\subsubsection{Cancelaci�n de un pedido terminado}
En el caso de que se pretanda cancelar un pedido que esta en estado finalizado, se genera una excepcion. El escenario es simple, por lo que creemos que no es necesario un diagrama de secuencias.

\subsubsection{Otros escenarios}

La notificaci�n de cancelaci�n se propaga a la GUI que permite al maestro
reingresar insumos recuperables. Este evento es asincr�nico y la cancelaci�n
se concreta aunque el maestro no lleve a cabo la reposici�n, caso en el que
se asume que nada podr�a reponerse.

Se pueden seguir variando los par�metros pero la secuencia ser�a similar. 
Finalmente, la cancelaci�n tambi�n puede darse en el �mbito del controlador 
de entregas y del controlador de listos. En estos casos el manejo es simple, 
sin embargo, si el sistema dise�ado no correspondiente a la operaci�n de
contingencia podr�a existir una complejidad adicional producto de las notificaciones
al \textit{delivery}.

\subsubsection{Reponer el stock}
Cuando se produce una cancelaci�n, puede ser necesario reponer el stock, por ejemplo si el pedido estaba ingresado y no se utilizaron sus insumos. Como creiamos que hacer que el controlador maneje esta reposicion del stock no era adecuado (viola por ejemplo el principio de SRP, el controlador de ingreso solo se encarga de los pedidos ingresados, no tiene porque conocer los mecanismos de reposisici�n de stock). Por eso asociamos a la interfaz cancelador un repositor de stock.

El repositor de stock tiene la responsabilidad de devolver el stock cuando un pedido se cancela. Esto lo hace segun el estado del pedido, en particular si el pedido estaba preparandose, debe pedir que se ingresen los insumos salvables. 

Entonces consideremos el caso de un pedido ingresado. Al cancelarse todo su stock se puede reutilizar, de modo que la reposicion es completa. Veamos los siguientes diagramas de secuencias:

\begin{landscape}
\begin{figure}[H]
\centering
\includegraphics[height=16cm]{./figuras/repositorPedidoIngresado.png}
\caption{Reponer stock de pedido ingresado}
\end{figure}
\end{landscape}

\begin{landscape}
\begin{figure}[H]
\centering
\includegraphics[height=16cm]{./figuras/repositorPedidoEnPreparacion.png}
\caption{Reponer stock de pedido en preparacion}
\end{figure}
\end{landscape}

\begin{landscape}
\begin{figure}[H]
\centering
\includegraphics[height=15cm]{./figuras/repositorPedidoNoIngresadoNiPreparacion.png}
\caption{Reponer stock de pedidos en otro estado distinto}
\end{figure}
\end{landscape}

En el primer diagrama vemos que el estado sea ingresado y entonces procedemos a reponer todo el stock de sus productos.

Si el pedido se estaba preparando, hay que preguntar que hacer con los productos que son preparables, es decir pedir que insumos reponer y cuales ya fueron usados, esto se puede observar en el segundo diagrama.

Finalmente si el pedido ya esta totalmente preparado, la especificaci�n pide que solo se restablezca el stock de las bebidas. Eso es precisamente lo que se muestra en el �ltimo diagrama.

Notar que extendimos la especifiacion preguntando por los productos no preprables ni cocinables, de modo que por ejemplo si se venden por helados, al cancelar el pedido los helados se pueden reutilizar.


%
%\section{Explicación de las clases}
%\clase
%{ABMproductos}
%{Esta clase se encarga de realizar las altas, bajas y modificaciones de los productos}
%{}{}
%
%\clase
%{ABMstock}
%{Esta clase se encarga de realizar las altas, bajas y modificaciones de los insumos}
%{}{}
%
%\clase
%{Agil}
%{Especialización de gestor de horno, permite aplicar la politica agil}
%{}{}
%
%\clase
%{Aviso}
%{Clase utilizada para realizar el \textit{callback} desde el gestor de horno hacia el despachador de preparación, es la encargada de ejecutar el metodo del despachador que lo notifica de que se termino de preparar algo. Esta clase permite que el despachador no necesite saber quien le avisa, y por lo tanto permite que los preparadores no requieran de metodos diferenciados para avisar que se terminaron de preparar las pizzas, o las empanadas}
%{}{}
%
%\clase{Cliente}
%{Esta clase representa a un cliente, conteniendo todos los datos del mismo.}
%{}{}
%
%\clase{ColaListos} %FIXME: nombre poco feliz
%{Esta clase contiene a los pedidos que ya estan listos. Su responsabilidad es la de conocer a todos los que estan en este estado, a fin de que despachar un pedido se haga desde esta clase}
%{}{}
%
%\clase
%{ControladorDeIngreso}
%{Controla la cola de ingreso, la cual puede ser modificada por el encargado de pedidos. Cuando algun preparador queda libre, envia el proximo pedido a preparar}
%{}{}
%
%\clase
%{ControladorCliente}
%{El controlador de cliente tiene por responsabilidad encargarse de autentificar un usuario}
%{}{}
%
%\clase
%{ControladorStock}
%{El controlador de stock, tiene por responsabilidad chequear la disponibilidad de insumos al momento de un ingreso, asi como la de hacer el decremento del stock al ingresar un pedido, generando el aviso de stock critico en caso de ser necesario.}
%{}{}
%
%\clase
%{CoordinadorDePedidos}
%{El coordinador de pedidos se encarga de controlar el ingreso de pedidos, y su ciclo de vida fuera de la cocina}
%{}{}
%
%\clase
%{DespachadorDePreparación}
%{Esta clase tiene por responsabilidad manejar la cola de pedidos que se estan preparando, recordemos que puede existir una cola de preparaci'on si hay pedidos mixtos a la espera de uno de los maestros. El controlador de ingresos distribuye los pedidos a los distintos preparadores y despacha cada pedido a su gestor de horno correspondiente cuando ya esta preparado.}
%{}{}
%
%\clase
%{EstimadorDeTiempos}
%{El estimador de tiempos, como lo dice su nombre, se encarga de estimar el tiempo de preparacion y cocción de un pedido}
%{}{}
%
%\clase
%{GeneradorDePedidos}
%{Esta clase se encarga de crear pedidos, creando pedidos de solo bebidas o con comida segun los productos}%FIXME: justificacion
%{}{}
%
%\clase
%{GestorHorno}
%{Clase abstracta que permite implementar diferentes politicas para el manejo del horno}
%{}{}
%
%\clase
%{Insumo}
%{Contiene la información de los distintos insumos de la pizzería}
%{}{}
%
%\clase
%{Normal}
%{Permite implementar la politica normal de manejo del horno}
%{}{}
%
%\clase
%{Pedido}
%{Contiene la información de cada pedido}
%{}{}
%

\chapter{Conclusión}

El trabajo práctico nos permitió aplicar los principios y patrones de diseño
en un caso realista y de complejidad no trivial. Esto nos dio la oportunidad de
razonar sobre el diseño de los componentes de un sistema mediano, y sobre
las implicaciones que este diseño tiene sobre la flexibilidad, modificabilidad y
mantenibilidad de un sistema.

La cantidad sustancial de trabajo que requirió documentar, aún de forma 
incompleta, el diseño de un sistema, así como el razonamiento detrás del
mismo nos permite entrever las limitaciones propias del modelo en cascada.
Si bien parece factible realizar, con un esfuerzo importante, el diseño 
completo de un sistema utilizando las técnicas de UML vistas en clase,
simplemente no parece realista la modificación y actualización de un documento
de este tipo para reflejar los cambios que vayan sucediéndose en la medida
en que cambian los requerimientos o se agregan nuevos.

Por otra parte, la utilidad de una instancia de diseño en el momento
previo a la implementación de un sistema fue bastante aparente. Muchas
decisiones iniciales de diseño debieron ser revisadas para ajustarse a
nuevas necesidades en la medida que fuimos completando el trabajo.
Esta instancia además brinda una oportunidad para reflexionar sobre
los aspectos en los que es importante brindar flexibilidad, y aquellos
en los que resulta más conveniente optar por una implementación simple
y acotada.

\subsection{Dificultades en la realización del TP}

Al igual que con el trabajo práctico anterior enfrentamos varias
dificultades producto de la organización de los tiempos que sentimos
nos complicaron el desarrollo del TP.

Como en la entrega anterior, el plazo teórico de realización del TP se
vio fuertemente acotado por la escasa disponibilidad de clases de consulta.
En la práctica, el tiempo utilizable para llevar a cabo el trabajo fue
sustancialmente menor dado que pasamos gran parte del tiempo bloqueados
frente a dudas sobre criterios de corrección u otras decisiones de la
materia sobre lo que es aceptable y lo que no lo es.

Dada la naturaleza fuertemente dependiente de los diagramas de secuencia
respecto del diseño del sistema completo, nos fue imposible asignar de forma
uniforme el tiempo disponible. Debimos en cambio trabajar lentamente al
principio  hasta obtener un diagrama de clases aceptable y luego apurarnos
sobre el final para elaborar todos los diagramas de secuencia que se derivan
del mismo. Esto inevitablemente repercute en la calidad del trabajo final,
y prácticamente garantiza la existencia de inconsistencias de pequeña o 
mediana magnitud, ya que resulta imposible revisar todas las referencias
cuando se produce algún cambio.

\label{LastPage}
\end{document}
