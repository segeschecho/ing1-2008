\chapter{Conclusiones}

El trabajo de implementaci�n nos permiti� observar la utilidad de un dise�o
detallado preparado de antemano. Salvo peque�as (y en general muy sutiles) diferencias,
el pasaje a la implementaci�n fue bastante mec�nico a partir del dise�o que
hab�amos producido anteriormente. 

En cierta manera, la realizaci�n de un dise�o
extensivo y profundo de todo el sistema se asemeja a realizar un prototipo del
mismo. Sin embargo, las herramientas utilizadas (UML) tienen tanto ventajas como
desventajas respecto de la realizaci�n de un verdadero prototipo en un lenguaje de
muy alto nivel. Por un lado, la mayor facilidad de visualizaci�n que tienen los
diagramas sobre el c�digo fuente lo hacen m�s apto para un cierto tipo de 
comprensi�n. Por otra parte, la nula disponibilidad de herramientas de colaboraci�n
y versionado para trabajar sobre UML (en comparaci�n con un lenguaje de programaci�n
tradicional) hacen que el trabajo sea m�s complicado, lento e inamovible. Por �ltimo
si bien la utilizaci�n de diagramas de secuencia permite verificar que el dise�o
sea razonable, un prototipo funcional es una garant�a m�s fuerte. De cualquier modo
pareciera que hay lugar para ambas t�cnicas en el desarrollo de un software complejo.

Durante la fase de desarrollo pudimos comprobar la calidad de nuestro dise�o,
aunque tuvimos que lidiar con su complejidad lo que produjo una cantidad importante
de clases y m�dulos. Toda la operatoria dise�ada cerraba a medida que se fue
programando, y el bajo acoplamiento logrado entre los m�dulos permiti� excluir las
partes que no eran de inter�s para la implementaci�n, as� como reemplazar otras por
\textit{stubs} con el objeto de observar mejor el funcionamiento del sistema.

Utilizar Python fue sin duda una buena elecci�n que nos facilit� mucho el trabajo,
pudiendo ponernos a trabajar desde el primer momento en vez de invertir mucho
tiempo en familiarizarnos con un nuevo �mbito de trabajo con la complejidad que
caracteriza a una plataforma como Java. Sin duda este tipo de desarrollos, en
poco tiempo y cuya correctitud no es cr�tica, as� como el tama�o del equipo es
reducido, se benefician ampliamente del uso de este tipo de lenguajes. El c�digo
producido es m�s compacto y legible con los beneficios de mantenibilidad que eso
conlleva.

Respecto del testing, podemos decir que nos sirvi� para adquirir un grado de
confianza alto en la correctitud del sistema implementado. Si bien es posible que
para la vida real un testing de las caracter�sticas planteadas para una funcionalidad
tan trivial sea excesivo, lo cierto es que permite chequear en un buen grado que
el sistema funciona como es de esperarse. En la medida que los sistemas escalan, se
puede intentar mantener el nivel de exhaustividad del testing recurriendo a herramientas
de testing automatizado como *Unit o Selenium.
