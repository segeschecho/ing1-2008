\chapter{Testing}
En este apartado presentaremos el dise�o, asi como tambi�n las cuestiones propias a la implementaci�n y realizaci�n del testing funcional sobre la funcionalidad de registro de pedidos, para los casos donde hay a lo sumo un producto comestible y un producto del tipo bebida.

De acuerdo a nuestro modelo, esta funcionalidad esta encapsulada en una llamada al metodo ingresarPedido de la clase CoordinadorDePedidos.

\section{Analisis de factores, categorias y choices}
%basicamente poner las cosas de testing, hacer tablita, etc
Los inputs directos de la funcionalidad son:
\begin{itemize}
\item Cliente
\item Lista de productos
\item Forma de pago
\item Origen
\item N�mero de mesa
\end{itemize}

Hay algunas situaciones que por nuestra implementaci�n no se pueden dar y que por lo tanto no seran tenidas en cuenta en las tablas a fin de que sean mas claras, por ejemplo que el cliente no exista no es algo que se tenga en cuenta, ya que para ingresar un pedido se elige un cliente de entre todos los clientes registrados, si en cambio hay que tener en cuenta el caso en el que no se eligi� ningun cliente. Por ejemplo tampoco se testearan los casos donde el origen o la forma de pago sean invalidas, ya que la GUI no permite ingresar un origen distinto a los origenes validos.
 %chequear que sea asi
A su vez podemos considerar otros elementos del sistema que modifican el comportamiento de la funcionalidad y que deben ser tenidos en cuenta:
\begin{itemize}
\item Preparadores ocupados o no
\item �Hay pedidos en la cola de ingreso?
\item Stock de los insumos necesarios para el pedido
\item Insumos con cantidad critica
\end{itemize}

A continuaci�n daremos una revisi�n de los diferentes factores, categorias y choices que tendremos en cuenta para dise�ar el testing.

\subsection{Origen}
El origen es un factor a tener en cuenta, ya que seg�n si el pedido es de mostrador, mesa o telefono, hay ciertos invariantes que se debe preservar para poder ingresar un pedido de este tipo. Por ejemplo un pedido de mesa debe tener una mesa asociada, un pedido telefonico debe tener un cliente asociado, etc. Entonces vemos que la categoria que nos importa del origen es su valor. Las choices son los distintos valores de origen: Mostrador, Mesa y Tel�fono

\subsection{N�mero de mesa}
En el caso de que se pretenda ingresar un pedido de mesa, un factor a tener en cuenta es el n�mero de mesa, en particular nos importa el valor, que puede ser nulo o no. Si es nulo, se debe generar un mensaje de error ya que el n�mero es necesario. En el caso de que el pedido no sea de mesa, no lo consideraremos porque la GUI no permite ingresar un n�mero de mesa cuando se intenta ingresar otro pedido.

\subsection{Cliente}
El cliente es otro factor a tener en cuenta. Como dijimos anteriormente, no consideraremos el caso en el que el cliente sea por ejemplo un cliente no registrado, ya que la GUI solo permite elegir a los clientes de una lista, mostrando los clientes registrados. Si consideraremos como categoria el valor del cliente en el sentido de si es nulo o no. En el caso de que el cliente sea nulo, si el pedido era telefonico, debe dar error.

\subsection{Lista de productos}
Otro factor a tener en cuenta es la lista de productos que componen al pedido. En particular una posible categoria a tener en cuenta es como esta compuesta. Por enunciado, nos limitaremos a una comida y una bebida. Entonces dentro de esta categoria consideraremos el caso donde la lista esta compuesta por 0 productos, condicion de error, una bebida, una comida o una bebida y una comida. El caso de una comida lo consideraremos como unico porque creemos que no presenta algun interese particular. Por otro lado el caso de una bebida tambi�n lo consideraremos como unico porque es un caso simple. La justificaci�n es que creemos que el caso donde hay solo bebidas es simple de tratar, pero por una cuesti�n de completitud decidimos probarlo.
%capaz solo bebida no es unico
En los casos en los que hay una comida, nos va a interesar tambi�n que tipo de comida es, si pizza o empanada, ya que cada producto tiene por ejemplo su preparador asociado, etc.

\subsection{Forma de pago}
La forma de pago es un factor a tener en cuenta para los pedidos de mostrador que permiten escoger entre dos posibles: efectivo o tarjeta. Los demas origenes solo pemiten especificar una forma de pago: los de mesa, forma de pago nula (no se conoce hasta que se cierra la mesa) y los telefonicos mediante el efectivo.


\subsection{Relacion productos y stock}
En este factor tenemos varias cosas que considerar. En primer lugar la cantidad actual de los insumos en relaci�n con la cantidad necesaria para satisfacer al pedido. Nosotros consideraremos los casos donde todos los insumos tienen menos stock que el necesario (condici�n de error), algunos tienen menos stock que el necesario (condici�n de error), algunos tienen un valor igual al necesario (y el resto mayor o igual) como caso unico para testear el caso borde, y finalmente el caso donde todos los insumos tienen una cantidad mayor a la necesaria.

Otra categoria a tener en cuenta es la cantidad actual de los insumos necesarios en funci�n con su cantidad critica. Consideraremos el caso donde los todos los insumos estan en condici�n critica, algunos estan y otros no y finalmente el caso donde ninguno esta en estado critico. El caso todos y ninguno los trataremos como unicos por dos razones principales: mantener la cantidad de casos acotado, y porque el caso algunos parece mas general y abarcativo.

Una tercer categoria a tener en cuenta es si hay insumos que pasan de un estado normal a uno de stock critico, en cuyo caso de debera mostrar un mensaje informativo (notar que si ya estaban en estado critico, no se debe mostrar un mensaje).
Podemos considerar las choices: ninguno hace esta transici�n, alguno la hace y el resto queda en valores mas grandes que el critico, alguno la hace, algunos quedan en valores iguales al critico y el resto en valores mas grandes, y finalmente el caso donde todos realizan la transici�n. Estos dos ultimos casos los consideraremos como unicos.
%TODO:explicar un poco mas como se hace esto

\subsection{Cola de ingreso}
La cola de ingreso es tambi�n un factor a tener en cuenta. La categoria es su tama�o. Si bien mas que el tama�o, lo que importa para decidir si un pedido se encola o no es el estado del preparador, es por lo menos conveniente considerar con un caso unico cuando la cola esta vacia y separar el caso de cuando en la cola ya hay pedidos.

\subsection{Preparadores}
Los preparadores tienen importancia como factores, ya que son los que deciden para un pedido con comida si este se encola o se comienza a preparar. 

Entonces en los casos en los que el pedido sea de comida, utilizando el hecho de que solo se testean los pedidos con un unico item de comida, podemos considerar como categorias el estado del preparador asociado al tipo del producto y el del preparador del otro tipo de comida. Entonces podemos considerar los valores ocupado, y libre para cada uno de ellos. El preparador del otro tipo no deberia tener mucha importancia, pero podria ocurrir por ejemplo que si el preparador correcto esta ocupado y el otro libre se realice una asignacion erronea. Por eso lo consideramos, a fin de mantener la cantidad de casos en un n�mero manejable, lo que haremos sera tomar el caso ocupado como unico

La tabla se encuentra adjunta a este trabajo %TODO: decidir que hacer con esto, si se trata de meter aca o se deja asi
\section{Dise�o de los casos de test}
De acuerdo a los factores antes identificados armamos la tabla de casos de test. La misma se encuentra adjunta a este trabajo %TODO: decidir que se hace con esto

\section{Realizaci�n del testing}
%contar algo de como se hicieron los test, onda valores para las instancias, resultados, etc. se puede hacer una tabla
En este apartado daremos una descripci�n de los datos con los que probamos cada caso de test. Primero introduciremos la notaci�n que vamos a utilizar para describir a las instancias y luego desarrollaremos las distintas instancias con las que se probraron.

\subsection{Notaci�n}
Antes de mostrar las instancias que se utilizaron para armar las pruebas estableceremos una convencion de como indicaremos los distintos tipos de datos que vamos a usar.

Para declarar un insumo utilizamos:

\verb0Insumo(cantidad,cantidadCritica,nombre)0

Para declarar un tipo de producto:

\verb0TipoProducto(Nombre,cocinable,preparable)0

Si queremos indicar un producto hacemos:

\verb0Producto(nombre,precio,tiempoCoccion,tiempoPreparacion,TipoProducto,insumos)0

Para un cliente usamos:

\verb0Cliente(apellido,celular,id,nombre,passWeb,telefono,usrweb, direccion)0

Finalmente para los pedidos usamos:

\verb#PedidoMostrador(Cliente,productos,formaDePago)#\\
\verb#PedidoMesa(Cliente,productos,mesa)#\\
\verb#PedidoTelefono(Cliente,productos)#\\

segun el origen. 

Al utilizar la GUI de nuestra aplicaci�n, PedidoMostrador(Cliente,productos,formaDePago) debe interpretarse como ir a la pesata�a de cola de ingreso, click en ingresar nuevo pedido, elegir orgien mostrador, selecionar la forma de pago $formaDePago$, el cliente $Cliente$ de la lista (o no elegir a ninguno si se desea un cliente None) y elegir los productos de la lista de productos.

En general necesitamos para realizar los tests que hallan productos, insumos, clientes, etc ya cargados. Por eso, salvo que se indique lo contrario, consideraremos que la pizzeria tiene los siguientes datos:

\noindent
\ttfamily
\shorthandoff{"}
\hlstd{}\hlslc{\#Insumo(cantidad,cantidadCritica,nombre)}\hspace*{\fill}\\
\hlstd{bollo\ }\hlsym{=\ }\hlstd{}\hlkwd{Insumo}\hlstd{}\hlsym{(}\hlstd{}\hlnum{30}\hlstd{}\hlsym{,}\hlstd{}\hlnum{10}\hlstd{}\hlsym{,}\hlstd{}\hlstr{"Bollo\ de\ Pizza"}\hlstd{}\hlsym{)}\hspace*{\fill}\\
\hlstd{tapa\ }\hlsym{=\ }\hlstd{}\hlkwd{Insumo}\hlstd{}\hlsym{(}\hlstd{}\hlnum{50}\hlstd{}\hlsym{,}\hlstd{}\hlnum{20}\hlstd{}\hlsym{,}\hlstd{}\hlstr{"Tapa\ de\ Empanada"}\hlstd{}\hlsym{)}\hspace*{\fill}\\
\hlstd{insumo1\ }\hlsym{=\ }\hlstd{}\hlkwd{Insumo}\hlstd{}\hlsym{(}\hlstd{}\hlnum{10}\hlstd{}\hlsym{,}\hlstd{}\hlnum{3}\hlstd{}\hlsym{,}\hlstd{}\hlstr{"Kit\ Muzzarella"}\hlstd{}\hlsym{)}\hspace*{\fill}\\
\hlstd{insumo2\ }\hlsym{=\ }\hlstd{}\hlkwd{Insumo}\hlstd{}\hlsym{(}\hlstd{}\hlnum{25}\hlstd{}\hlsym{,}\hlstd{}\hlnum{25}\hlstd{}\hlsym{,}\hlstd{}\hlstr{"Kit\ Empanada\ de\ Carne"}\hlstd{}\hlsym{)}\hspace*{\fill}\\
\hlstd{insumo3\ }\hlsym{=\ }\hlstd{}\hlkwd{Insumo}\hlstd{}\hlsym{(}\hlstd{}\hlnum{13}\hlstd{}\hlsym{,}\hlstd{}\hlnum{13}\hlstd{}\hlsym{,}\hlstd{}\hlstr{"Kit\ Napolitana"}\hlstd{}\hlsym{)}\hspace*{\fill}\\
\hlstd{insumo4\ }\hlsym{=\ }\hlstd{}\hlkwd{Insumo}\hlstd{}\hlsym{(}\hlstd{}\hlnum{100}\hlstd{}\hlsym{,}\hlstd{}\hlnum{10}\hlstd{}\hlsym{,}\hlstd{}\hlstr{"Kit\ Empanada\ de\ Pollo"}\hlstd{}\hlsym{)}\hspace*{\fill}\\
\hlstd{insumo5\ }\hlsym{=\ }\hlstd{}\hlkwd{Insumo}\hlstd{}\hlsym{(}\hlstd{}\hlnum{10}\hlstd{}\hlsym{,}\hlstd{}\hlnum{11}\hlstd{}\hlsym{,}\hlstd{}\hlstr{"Kit\ Roquefort"}\hlstd{}\hlsym{)}\hspace*{\fill}\\
\hlstd{insumo6\ }\hlsym{=\ }\hlstd{}\hlkwd{Insumo}\hlstd{}\hlsym{(}\hlstd{}\hlnum{12}\hlstd{}\hlsym{,}\hlstd{}\hlnum{12}\hlstd{}\hlsym{,}\hlstd{}\hlstr{"Botella\ de\ Quilmes"}\hlstd{}\hlsym{)}\hspace*{\fill}\\
\hlstd{insumo7\ }\hlsym{=\ }\hlstd{}\hlkwd{Insumo}\hlstd{}\hlsym{(}\hlstd{}\hlnum{45}\hlstd{}\hlsym{,}\hlstd{}\hlnum{23}\hlstd{}\hlsym{,}\hlstd{}\hlstr{"Botella\ de\ Coca{-}Cola"}\hlstd{}\hlsym{)}\hspace*{\fill}\\
\hlstd{insumo8\ }\hlsym{=\ }\hlstd{}\hlkwd{Insumo}\hlstd{}\hlsym{(}\hlstd{}\hlnum{0}\hlstd{}\hlsym{,}\hlstd{}\hlnum{12}\hlstd{}\hlsym{,}\hlstd{}\hlstr{"Botella\ de\ Iguana"}\hlstd{}\hlsym{)}\hspace*{\fill}\\
\hlstd{insumo9\ }\hlsym{=\ }\hlstd{}\hlkwd{Insumo}\hlstd{}\hlsym{(}\hlstd{}\hlnum{22}\hlstd{}\hlsym{,}\hlstd{}\hlnum{23}\hlstd{}\hlsym{,}\hlstd{}\hlstr{"Botella\ de\ Pepsi"}\hlstd{}\hlsym{)}\hspace*{\fill}\\
\hlstd{insumo10\ }\hlsym{=\ }\hlstd{}\hlkwd{Insumo}\hlstd{}\hlsym{(}\hlstd{}\hlnum{0}\hlstd{}\hlsym{,}\hlstd{}\hlnum{1}\hlstd{}\hlsym{,}\hlstd{}\hlstr{"Kit\ Calabresa"}\hlstd{}\hlsym{)}\hspace*{\fill}\\
\hlstd{insumo11\ }\hlsym{=\ }\hlstd{}\hlkwd{Insumo}\hlstd{}\hlsym{(}\hlstd{}\hlnum{14}\hlstd{}\hlsym{,}\hlstd{}\hlnum{20}\hlstd{}\hlsym{,}\hlstd{}\hlstr{"Kit\ Empanada\ de\ humita"}\hlstd{}\hlsym{)}\hspace*{\fill}\\
\hlstd{insumo12\ }\hlsym{=\ }\hlstd{}\hlkwd{Insumo}\hlstd{}\hlsym{(}\hlstd{}\hlnum{0}\hlstd{}\hlsym{,}\hlstd{}\hlnum{12}\hlstd{}\hlsym{,}\hlstd{}\hlstr{"Kit\ Empanada\ de\ atun"}\hlstd{}\hlsym{)}\hspace*{\fill}\\
\hlstd{}\hspace*{\fill}\\
\hlslc{\#TipoProducto(Nombre,cocinable,preparabl)}\hspace*{\fill}\\
\hlstd{pizza}\hlsym{=\ }\hlstd{}\hlkwd{TipoProducto}\hlstd{}\hlsym{(}\hlstd{}\hlstr{"Pizza"}\hlstd{}\hlsym{,}\hlstd{}\hlkwa{True}\hlstd{}\hlsym{,}\hlstd{}\hlkwa{True}\hlstd{}\hlsym{)}\hspace*{\fill}\\
\hlstd{empanada}\hlsym{=\ }\hlstd{}\hlkwd{TipoProducto}\hlstd{}\hlsym{(}\hlstd{}\hlstr{"Empanada"}\hlstd{}\hlsym{,}\hlstd{}\hlkwa{True}\hlstd{}\hlsym{,}\hlstd{}\hlkwa{True}\hlstd{}\hlsym{)}\hspace*{\fill}\\
\hlstd{coca}\hlsym{=}\hlstd{\ \ }\hlsym{}\hlstd{}\hlkwd{TipoProducto}\hlstd{}\hlsym{(}\hlstd{}\hlstr{"Gaseosa"}\hlstd{}\hlsym{,}\hlstd{}\hlkwa{False}\hlstd{}\hlsym{,}\hlstd{}\hlkwa{False}\hlstd{}\hlsym{)}\hspace*{\fill}\\
\hlstd{cerveza}\hlsym{=\ }\hlstd{}\hlkwd{TipoProducto}\hlstd{}\hlsym{(}\hlstd{}\hlstr{"Cerveza"}\hlstd{}\hlsym{,}\hlstd{}\hlkwa{False}\hlstd{}\hlsym{,}\hlstd{}\hlkwa{False}\hlstd{}\hlsym{)}\hspace*{\fill}\\
\hlstd{}\hspace*{\fill}\\
\hlslc{\#Producto(nombre,precio,tiempoCoccion,tiempoPreparacion,TipoProducto,insumos)}\hspace*{\fill}\\
\hlstd{}\hspace*{\fill}\\
\hlslc{\#bebidas}\hspace*{\fill}\\
\hlstd{Bebida1\ }\hlsym{=\ }\hlstd{}\hlkwd{Producto}\hlstd{}\hlsym{(}\hlstd{}\hlstr{"Coca{-}Cola"}\hlstd{}\hlsym{,}\hlstd{}\hlnum{5}\hlstd{}\hlsym{,}\hlstd{}\hlnum{0.0}\hlstd{}\hlsym{,}\hlstd{}\hlnum{0.0}\hlstd{}\hlsym{,}\hlstd{coca}\hlsym{,{[}}\hlstd{insumo7}\hlsym{{]})\ }\hlstd{}\hlslc{\#no\ critico}\hspace*{\fill}\\
\hlstd{Bebida2\ }\hlsym{=\ }\hlstd{}\hlkwd{Producto}\hlstd{}\hlsym{(}\hlstd{}\hlstr{"Quilmes"}\hlstd{}\hlsym{,}\hlstd{}\hlnum{7}\hlstd{}\hlsym{,}\hlstd{}\hlnum{0.0}\hlstd{}\hlsym{,}\hlstd{}\hlnum{0.0}\hlstd{}\hlsym{,}\hlstd{cerveza}\hlsym{,{[}}\hlstd{insumo6}\hlsym{{]})\ }\hlstd{}\hlslc{\#al\ limite\ de\ critico}\hspace*{\fill}\\
\hlstd{Bebida3\ }\hlsym{=\ }\hlstd{}\hlkwd{Producto}\hlstd{}\hlsym{(}\hlstd{}\hlstr{"Pepsi\ Cola"}\hlstd{}\hlsym{,}\hlstd{}\hlnum{5}\hlstd{}\hlsym{,}\hlstd{}\hlnum{0.0}\hlstd{}\hlsym{,}\hlstd{}\hlnum{0.0}\hlstd{}\hlsym{,}\hlstd{coca}\hlsym{,{[}}\hlstd{insumo9}\hlsym{{]})\ }\hlstd{}\hlslc{\#\ critico}\hspace*{\fill}\\
\hlstd{Bebida2\ }\hlsym{=\ }\hlstd{}\hlkwd{Producto}\hlstd{}\hlsym{(}\hlstd{}\hlstr{"Iguana"}\hlstd{}\hlsym{,}\hlstd{}\hlnum{7}\hlstd{}\hlsym{,}\hlstd{}\hlnum{0.0}\hlstd{}\hlsym{,}\hlstd{}\hlnum{0.0}\hlstd{}\hlsym{,}\hlstd{cerveza}\hlsym{,{[}}\hlstd{insumo8}\hlsym{{]})\ }\hlstd{}\hlslc{\#\ agotada}\hspace*{\fill}\\
\hlstd{}\hspace*{\fill}\\
\hlslc{\#pizzas}\hspace*{\fill}\\
\hlstd{Pizza1\ }\hlsym{=\ }\hlstd{}\hlkwd{Producto}\hlstd{}\hlsym{(}\hlstd{}\hlstr{"Muzzarella"}\hlstd{}\hlsym{,}\hlstd{}\hlnum{25}\hlstd{}\hlsym{,}\hlstd{}\hlnum{5.0}\hlstd{}\hlsym{,}\hlstd{}\hlnum{10.2}\hlstd{}\hlsym{,}\hlstd{pizza}\hlsym{,{[}}\hlstd{bollo}\hlsym{,}\hlstd{insumo1}\hlsym{{]})\ }\hlstd{}\hlslc{\#no\ critico}\hspace*{\fill}\\
\hlstd{Pizza2\ }\hlsym{=\ }\hlstd{}\hlkwd{Producto}\hlstd{}\hlsym{(}\hlstd{}\hlstr{"Napolitana"}\hlstd{}\hlsym{,}\hlstd{}\hlnum{30}\hlstd{}\hlsym{,}\hlstd{}\hlnum{5.4}\hlstd{}\hlsym{,}\hlstd{}\hlnum{12.3}\hlstd{}\hlsym{,}\hlstd{pizza}\hlsym{,{[}}\hlstd{bollo}\hlsym{,}\hlstd{insumo3}\hlsym{{]})\ }\hlstd{}\hlslc{\#al\ limite\ de\ critico}\hspace*{\fill}\\
\hlstd{Pizza3\ }\hlsym{=\ }\hlstd{}\hlkwd{Producto}\hlstd{}\hlsym{(}\hlstd{}\hlstr{"Roquefort"}\hlstd{}\hlsym{,}\hlstd{}\hlnum{40}\hlstd{}\hlsym{,}\hlstd{}\hlnum{7}\hlstd{}\hlsym{,}\hlstd{}\hlnum{15.0}\hlstd{}\hlsym{,}\hlstd{pizza}\hlsym{,{[}}\hlstd{bollo}\hlsym{,}\hlstd{insumo5}\hlsym{{]})\ }\hlstd{}\hlslc{\#\ critico}\hspace*{\fill}\\
\hlstd{Pizza4\ }\hlsym{=\ }\hlstd{}\hlkwd{Producto}\hlstd{}\hlsym{(}\hlstd{}\hlstr{"Calabresa"}\hlstd{}\hlsym{,}\hlstd{}\hlnum{40}\hlstd{}\hlsym{,}\hlstd{}\hlnum{7.1}\hlstd{}\hlsym{,}\hlstd{}\hlnum{15.0}\hlstd{}\hlsym{,}\hlstd{pizza}\hlsym{,{[}}\hlstd{bollo}\hlsym{,}\hlstd{insumo10}\hlsym{{]})}\hlstd{}\hlslc{\#Agotada}\hspace*{\fill}\\
\hlstd{}\hspace*{\fill}\\
\hlslc{\#empanadas}\hspace*{\fill}\\
\hlstd{Empanada1\ }\hlsym{=\ }\hlstd{}\hlkwd{Producto}\hlstd{}\hlsym{(}\hlstd{}\hlstr{"Pollo"}\hlstd{}\hlsym{,}\hlstd{}\hlnum{3}\hlstd{}\hlsym{,}\hlstd{}\hlnum{10}\hlstd{}\hlsym{,}\hlstd{}\hlnum{11.0}\hlstd{}\hlsym{,}\hlstd{empanada}\hlsym{,{[}}\hlstd{tapa}\hlsym{,}\hlstd{insumo4}\hlsym{{]})\ }\hlstd{}\hlslc{\#No\ critico}\hspace*{\fill}\\
\hlstd{Empanada2\ }\hlsym{=\ }\hlstd{}\hlkwd{Producto}\hlstd{}\hlsym{(}\hlstd{}\hlstr{"Carne"}\hlstd{}\hlsym{,}\hlstd{}\hlnum{3}\hlstd{}\hlsym{,}\hlstd{}\hlnum{11}\hlstd{}\hlsym{,}\hlstd{}\hlnum{12.0}\hlstd{}\hlsym{,}\hlstd{empanada}\hlsym{,{[}}\hlstd{tapa}\hlsym{,}\hlstd{insumo2}\hlsym{{]})\ }\hlstd{}\hlslc{\#al\ limite\ de\ critico}\hspace*{\fill}\\
\hlstd{Empanada3\ }\hlsym{=\ }\hlstd{}\hlkwd{Producto}\hlstd{}\hlsym{(}\hlstd{}\hlstr{"Humita"}\hlstd{}\hlsym{,}\hlstd{}\hlnum{3}\hlstd{}\hlsym{,}\hlstd{}\hlnum{10}\hlstd{}\hlsym{,}\hlstd{}\hlnum{11.0}\hlstd{}\hlsym{,}\hlstd{empanada}\hlsym{,{[}}\hlstd{tapa}\hlsym{,}\hlstd{insumo11}\hlsym{{]})\ }\hlstd{}\hlslc{\#critico}\hspace*{\fill}\\
\hlstd{Empanada4\ }\hlsym{=\ }\hlstd{}\hlkwd{Producto}\hlstd{}\hlsym{(}\hlstd{}\hlstr{"Atun"}\hlstd{}\hlsym{,}\hlstd{}\hlnum{3}\hlstd{}\hlsym{,}\hlstd{}\hlnum{11}\hlstd{}\hlsym{,}\hlstd{}\hlnum{12.0}\hlstd{}\hlsym{,}\hlstd{empanada}\hlsym{,{[}}\hlstd{tapa}\hlsym{,}\hlstd{insumo12}\hlsym{{]})\ }\hlstd{}\hlslc{\#Agotada}\hspace*{\fill}\\
\hlstd{\hspace*{\fill}\\
\hspace*{\fill}\\
dir1\ }\hlsym{=\ }\hlstd{}\hlkwd{Direccion}\hlstd{}\hlsym{(}\hlstd{}\hlstr{"Trinidad"}\hlstd{}\hlsym{,\ }\hlstd{}\hlstr{"N/A"}\hlstd{}\hlsym{,\ }\hlstd{}\hlstr{"tortuguitas"}\hlstd{}\hlsym{,\ }\hlstd{}\hlnum{123}\hlstd{}\hlsym{)}\hspace*{\fill}\\
\hlstd{dir2\ }\hlsym{=\ }\hlstd{}\hlkwd{Direccion}\hlstd{}\hlsym{(}\hlstd{}\hlstr{"Los\ alamos"}\hlstd{}\hlsym{,\ }\hlstd{}\hlstr{"N/A"}\hlstd{}\hlsym{,\ }\hlstd{}\hlstr{"Wilde"}\hlstd{}\hlsym{,\ }\hlstd{}\hlnum{1465}\hlstd{}\hlsym{)}\hspace*{\fill}\\
\hlstd{dir3\ }\hlsym{=\ }\hlstd{}\hlkwd{Direccion}\hlstd{}\hlsym{(}\hlstd{}\hlstr{"San\ Martin"}\hlstd{}\hlsym{,\ }\hlstd{}\hlstr{"2do\ Piso"}\hlstd{}\hlsym{,\ }\hlstd{}\hlstr{"Capital\ Federal"}\hlstd{}\hlsym{,\ }\hlstd{}\hlnum{3988}\hlstd{}\hlsym{)}\hspace*{\fill}\\
\hlstd{dir4\ }\hlsym{=\ }\hlstd{}\hlkwd{Direccion}\hlstd{}\hlsym{(}\hlstd{}\hlstr{"Montaneses"}\hlstd{}\hlsym{,\ }\hlstd{}\hlstr{"1er\ Piso"}\hlstd{}\hlsym{,\ }\hlstd{}\hlstr{"Capital\ Federal"}\hlstd{}\hlsym{,\ }\hlstd{}\hlnum{345}\hlstd{}\hlsym{)}\hspace*{\fill}\\
\hlstd{dir5\ }\hlsym{=\ }\hlstd{}\hlkwd{Direccion}\hlstd{}\hlsym{(}\hlstd{}\hlstr{"Oliden"}\hlstd{}\hlsym{,\ }\hlstd{}\hlstr{"N/A"}\hlstd{}\hlsym{,\ }\hlstd{}\hlstr{"Vicente\ Lopez"}\hlstd{}\hlsym{,\ }\hlstd{}\hlnum{3433}\hlstd{}\hlsym{)}\hspace*{\fill}\\
\hlstd{dir6\ }\hlsym{=\ }\hlstd{}\hlkwd{Direccion}\hlstd{}\hlsym{(}\hlstd{}\hlstr{"Los\ Paraisos"}\hlstd{}\hlsym{,\ }\hlstd{}\hlstr{"N/A"}\hlstd{}\hlsym{,\ }\hlstd{}\hlstr{"Bursaco"}\hlstd{}\hlsym{,\ }\hlstd{}\hlnum{20}\hlstd{}\hlsym{)}\hspace*{\fill}\\
\hlstd{\hspace*{\fill}\\
cli1\ }\hlsym{=\ }\hlstd{}\hlkwd{Cliente}\hlstd{}\hlsym{(}\hlstd{}\hlstr{"Sainz\ Trapaga"}\hlstd{}\hlsym{,\ }\hlstd{}\hlnum{1555664488}\hlstd{}\hlsym{,\ }\hlstd{}\hlnum{1}\hlstd{}\hlsym{,\ }\hlstd{}\hlstr{"Gonzalo"}\hlstd{}\hlsym{,\ }\hlstd{}\hlstr{"hayQPaja"}\hlstd{}\hlsym{,\ }\hlstd{}\hlnum{48566633}\hlstd{}\hlsym{,\ }\hlstd{}\hlstr{"gomox"}\hlstd{}\hlsym{,\ }\hlstd{dir1}\hlsym{)}\hspace*{\fill}\\
\hlstd{cli2\ }\hlsym{=\ }\hlstd{}\hlkwd{Cliente}\hlstd{}\hlsym{(}\hlstd{}\hlstr{"Martinez"}\hlstd{}\hlsym{,\ }\hlstd{}\hlnum{1521356684}\hlstd{}\hlsym{,\ }\hlstd{}\hlnum{2}\hlstd{}\hlsym{,\ }\hlstd{}\hlstr{"Federico"}\hlstd{}\hlsym{,\ }\hlstd{}\hlstr{"estudiando"}\hlstd{}\hlsym{,\ }\hlstd{}\hlnum{46532233}\hlstd{}\hlsym{,\ }\hlstd{}\hlstr{"fedefly"}\hlstd{}\hlsym{,\ }\hlstd{dir2}\hlsym{)}\hspace*{\fill}\\
\hlstd{cli3\ }\hlsym{=\ }\hlstd{}\hlkwd{Cliente}\hlstd{}\hlsym{(}\hlstd{}\hlstr{"Gonzalez"}\hlstd{}\hlsym{,\ }\hlstd{}\hlnum{1523314655}\hlstd{}\hlsym{,\ }\hlstd{}\hlnum{3}\hlstd{}\hlsym{,\ }\hlstd{}\hlstr{"Emiliano"}\hlstd{}\hlsym{,\ }\hlstd{}\hlstr{"bartolo"}\hlstd{}\hlsym{,\ }\hlstd{}\hlnum{45632132}\hlstd{}\hlsym{,\ }\hlstd{}\hlstr{"emilio"}\hlstd{}\hlsym{,\ }\hlstd{dir3}\hlsym{)}\hspace*{\fill}\\
\hlstd{cli4\ }\hlsym{=\ }\hlstd{}\hlkwd{Cliente}\hlstd{}\hlsym{(}\hlstd{}\hlstr{"Gonzalez"}\hlstd{}\hlsym{,\ }\hlstd{}\hlnum{1532169788}\hlstd{}\hlsym{,\ }\hlstd{}\hlnum{4}\hlstd{}\hlsym{,\ }\hlstd{}\hlstr{"Sergio"}\hlstd{}\hlsym{,\ }\hlstd{}\hlstr{"lechuga"}\hlstd{}\hlsym{,\ }\hlstd{}\hlnum{45125398}\hlstd{}\hlsym{,\ }\hlstd{}\hlstr{"checho"}\hlstd{}\hlsym{,\ }\hlstd{dir4}\hlsym{)}\hspace*{\fill}\\
\hlstd{cli5\ }\hlsym{=\ }\hlstd{}\hlkwd{Cliente}\hlstd{}\hlsym{(}\hlstd{}\hlstr{"Rinaldi"}\hlstd{}\hlsym{,\ }\hlstd{}\hlnum{1564659777}\hlstd{}\hlsym{,\ }\hlstd{}\hlnum{5}\hlstd{}\hlsym{,\ }\hlstd{}\hlstr{"Nicolas"}\hlstd{}\hlsym{,\ }\hlstd{}\hlstr{"esteee"}\hlstd{}\hlsym{,\ }\hlstd{}\hlnum{46633221}\hlstd{}\hlsym{,\ }\hlstd{}\hlstr{"elcorrector"}\hlstd{}\hlsym{,\ }\hlstd{dir5}\hlsym{)}\hspace*{\fill}\\
\hlstd{cli6\ }\hlsym{=\ }\hlstd{}\hlkwd{Cliente}\hlstd{}\hlsym{(}\hlstd{}\hlstr{"D'arrigo"}\hlstd{}\hlsym{,\ }\hlstd{}\hlnum{1564521133}\hlstd{}\hlsym{,\ }\hlstd{}\hlnum{6}\hlstd{}\hlsym{,\ }\hlstd{}\hlstr{"Sergio"}\hlstd{}\hlsym{,\ }\hlstd{}\hlstr{"niato"}\hlstd{}\hlsym{,\ }\hlstd{}\hlnum{46632136}\hlstd{}\hlsym{,\ }\hlstd{}\hlstr{"jefetp"}\hlstd{}\hlsym{,\ }\hlstd{dir6}\hlsym{)}\hspace*{\fill}\\
\hlstd{}\hspace*{\fill}\\
\mbox{}
\normalfont
\shorthandon{"}

Tambi�n consideremos los siguientes pedidos simples:

\noindent
\ttfamily
\shorthandoff{"}
\hlstd{}\hlslc{\#Pedidos}\hspace*{\fill}\\
\hlstd{PedidoPizza\ }\hlsym{=\ }\hlstd{}\hlkwd{PedidoMostrador}\hlstd{}\hlsym{(}\hlstd{}\hlkwa{None}\hlstd{}\hlsym{,\ {[}}\hlstd{pizza1}\hlsym{{]},\ }\hlstd{efectivo}\hlsym{)}\hspace*{\fill}\\
\hlstd{PedidoEmpanada\ }\hlsym{=\ }\hlstd{}\hlkwd{PedidoMostrador}\hlstd{}\hlsym{(}\hlstd{}\hlkwa{None}\hlstd{}\hlsym{,{[}}\hlstd{Empanada1}\hlsym{{]},}\hlstd{efectivo}\hlsym{)}\hlstd{}\hspace*{\fill}\\
\mbox{}
\normalfont
\shorthandon{"}

De modo generico, cuando digamos un pedido pizzero o un pedido empanadero nos referimos a los pedidos arriba declarados.
\subsection{Datos de prueba}

\subsubsection{Caso de test 1}

Tomamos la pizzeria comun, y como pedido a agregar consideramos:

\verb0PedidoMesa(None,[pizza1],None)0

\subsubsection{Caso de test 2}

Tomamos la pizzeria comun, e ingresamos:

\verb0PedidoTelefono(None,[pizza1])0

\subsubsection{Caso de test 3}
\verb0PedidoMostrador(None,[],efectivo)0

\subsubsection{Caso de test 4}
Primero agregamos pedidos para que la cola de ingreso no este vacia, ingresamo entonces:
\verb0PedidoPizzas, PedidoPizzas0

Luego ingresamos:

\verb0PedidoMesa(None,[bebida1],4)0

\subsubsection{Caso de test 5}
Notemos que la pizza de calabresa de la pizzeria que definimos antes tiene su kit agotado, como tambi�n
queremos que el bollo este agotado,lo que hacemos primero es redefinir la cantidad de bollos, bajandola a 0.

Luego hacemos:

\verb0PedidoMostrador(None,[pizza4],efectivo)0

\subsubsection{Caso de test 6}
En este caso solo hacemos un pedido de calabresa

\verb0PedidoMostrador(None,[pizza4],efectivo)0

\subsubsection{Caso de test 7}
Tomamos a la pizzeria y redefinimos la cantidad de tapas para que esten en cantidad critica, y lo ponemos en 9. Asi al pedir 9 empanadas nos quedan 0 tapas.
Colocamos un pedido pizzero para que el preparador pizzero quede ocupado.

Luego ingresamos:

\verb0PedidoMostrador(cli1,[empanada3]*9+[bebida3],efectivo)0

\subsubsection{Caso de test 8}
Primero vamos a hacer lo siguiente: ponemos la cantidad de bollos en 17. Luego, ingresamos dos pedidos empanaderos para ocupar al preparador y dejar la cola no vacia.

Luego ingresamos el siguiente pedido:

\verb0PedodTelefono(cli2,[pizza1]*8+[bebida2])0

\subsubsection{Caso de test 9}
A la pizzeria normal, le cambiamos el stock de bollos a 15, y le ingresamos dos pedidosPizzeros, con lo cual dejamos la cola de ingreso con un pedido y el preparador pizzero ocupado.

Luego ingresamos:
\verb0PedidoMesa(cli2,[pizza3,pizza3,pizza3],2)0

\subsubsection{Caso de test 10}
Agregamos 2 pedidos pizzeros, luego hacemos:

\verb0PedidoMostrador(cli3,[pizza3,pizza3,bebida2,bebida2],efectivo)0

\subsubsection{Caso de test 11}
El caso no se puede generar ya que la cola de ingreso tiene pedidos encolados pero ambos preparadores estan ociosos, con lo cual
la pizzeria no cumple su invariante. Notar que generar esta instancia es responsabilidad de nuestra funcionalidad, ya que deben agregar pedidos.

\subsubsection{Caso de test 12}
\verb0PedidoMostrador(cli3,[pizza3,pizza3,bebida2,bebida2],tarjeta)0

\subsubsection{Caso de test 13}
Analogo al 11, no se puede representar.

\subsubsection{Caso de test 14}
Ingresamos 2 pedidos empanaderos a la pizzeria.

Luego para hacer el testing ingresamos:

\verb0PedidoMostrador(cli4,[empanada3,empanada3,bebida2,bebida2],efectivo)0
												
\subsubsection{Caso de test 15}
No se puede representar	

\subsubsection{Caso de test 16}
Igual al 14 pero ingresamos este pedido:

\verb0PedidoMostrador(cli4,[empanada3,empanada3,bebida2,bebida2],tarjeta)0

\subsubsection{Caso de test 17}
No se puede representar			

\subsubsection{Caso de test 18}
Agregamos 2 pedidos pizzeros, luego hacemos:

\verb0PedidoTelefono(cli3,[pizza3,pizza3,bebida2,bebida2])0							
											
\subsubsection{Caso de test 19}
No se puede generar
												
\subsubsection{Caso de test 20}
Ingresamos 2 pedidos empanaderos a la pizzeria.

Luego para hacer el testing ingresamos:

\verb0PedidoTelefono(cli4,[empanada3,empanada3,bebida2,bebida2])0
												
\subsubsection{Caso de test 21}
No se puede representar

\subsubsection{Caso de test 22}
Agregamos 2 pedidos pizzeros, luego hacemos:

\verb0PedidoMesa(cli2,[pizza3,pizza3,bebida2,bebida2],5)0												
											
\subsubsection{Caso de test 23}
No se puede generar

\subsubsection{Caso de test 24}
Ingresamos 2 pedidos empanaderos a la pizzeria.

Luego para hacer el testing ingresamos:

\verb0PedidoMostrador(cli4,[empanada3,empanada3,bebida2,bebida2],efectivo)0												

\subsubsection{Caso de test 25}
No se puede generar
		
\subsubsection{Caso de test 26}
Agregamos 2 pedidos pizzeros, un pedido Empanadero y luego hacemos:

\verb0PedidoMostrador(cli3,[pizza3,pizza3,bebida2,bebida2],efectivo)0

\subsubsection{Caso de test 27}
Agregamos 2 pedidos Empanaderos y luego hacemos:

\verb0PedidoMostrador(cli3,[pizza3,pizza3,bebida2,bebida2],efectivo)0

\subsubsection{Caso de test 28}
Agregamos 2 pedidos pizzeros y un empanadero, luego hacemos:

\verb0PedidoMostrador(cli3,[pizza3,pizza3,bebida2,bebida2],tarjeta)0

\subsubsection{Caso de test 29}
Simil al 28, pero agregamos 2 pedidos empanaderos, el pedido a ingresar es:

\verb0PedidoMostrador(cli3,[pizza3,pizza3,bebida2,bebida2],tarjeta)0

\subsubsection{Caso de test 30}
Ingresamos 2 pedidos empanaderos y un pizzero a la pizzeria.

Luego para hacer el testing ingresamos:

\verb0PedidoMostrador(cli4,[empanada3,empanada3,bebida2,bebida2],efectivo])0
												
\subsubsection{Caso de test 31}
idem 30, pero solo ingresamos 2 empanaderos y hacemos:

\verb0PedidoMostrador(cli4,[empanada3,empanada3,bebida2,bebida2],efectivo])0

\subsubsection{Caso de test 32}
Ingresamos 2 pedidos empanaderos y un pizzero a la pizzeria.

Luego para hacer el testing ingresamos:

\verb0PedidoMostrador(cli4,[empanada3,empanada3,bebida2,bebida2],tarjeta)0

\subsubsection{Caso de test 33}
Ingresamos 2 pedidos pizzeros.

Luego para hacer el testing ingresamos:

\verb0PedidoMostrador(cli4,[empanada3,empanada3,bebida2,bebida2],tarjeta)0

\subsubsection{Caso de test 34}
Agregamos 2 pedidos pizzeros y 1 empanadero, luego hacemos:

\verb0PedidoTelefono(cli3,[pizza3,pizza3,bebida2,bebida2])0							
											
\subsubsection{Caso de test 35}
Agregamos 2 pedidos empanaderos

\verb0PedidoTelefono(cli1,[pizza3,pizza3,bebida2,bebida2])0
												
\subsubsection{Caso de test 36}
Ingresamos 2 pedidos empanaderos y un pizzero a la pizzeria.

Luego para hacer el testing ingresamos:

\verb0PedidoTelefono(cli4,[empanada3,empanada3,bebida2,bebida2]]0
												
\subsubsection{Caso de test 37}
Ingresamos 2 pedidos pizzeros a la pizzeria.

Luego para hacer el testing ingresamos:

\verb0PedidoTelefono(cli4,[empanada3,empanada3,bebida2,bebida2]]0

\subsubsection{Caso de test 38}
Agregamos 2 pedidos pizzeros y uno empanadero, luego hacemos:

\verb0PedidoMesa(cli2,[pizza3,pizza3,bebida2,bebida2],5)0												
											
\subsubsection{Caso de test 39}
Agregamos 2 pedidos empanaderos, luego hacemos:

\verb0PedidoMesa(cli2,[pizza3,pizza3,bebida2,bebida2],14)0												

\subsubsection{Caso de test 40}
Ingresamos 2 pedidos empanaderos y uno pizzero a la pizzeria.

Luego para hacer el testing ingresamos:
\verb0PedidoMostrador(cli4,[empanada3,empanada3,bebida2,bebida2],efectivo]0												

\subsubsection{Caso de test 41}
Ingresamos 2 pedidos pizzeros a la pizzeria.

Luego para hacer el testing ingresamos:
\verb0PedidoMostrador(cli3,[empanada3,empanada3,bebida2,bebida2],efectivo]0																							


\subsection{Ejecucion de las pruebas}
Para ejecutar los testings se puede crear las pizzarias a mano y luego utilizar el modulo pickle de python para dumpearla, y utilizando la funcionlidad de abrir que brinda nuestra gui cargar dicha instancia y luego hacer uso de la funcionalidad de ingresar pedido para observar los resultados.

Por ejemplo el siguiente script permite crear una pizzeria como la descripta anteriormente, agregandole un pedido de una empanada y la deja lista para ser cargada por la aplicacion en el archivo pizzeriaTesting.pyp\\
\noindent
\ttfamily
\shorthandoff{"}
\hlstd{}\hlline{\ \ \ \ 1\ }\hlkwa{from\ }\hlstd{inicializador\ }\hlkwa{import\ }\hlstd{}\hlsym{{*}}\\
\hlline{\ \ \ \ 2\ }\hlstd{}\\
\hlline{\ \ \ \ 3\ }\hlkwa{import\ }\hlstd{pickle}\\
\hlline{\ \ \ \ 4\ }\\
\hlline{\ \ \ \ 5\ }\hlslc{\#Insumo(cantidad,cantidadCritica,nombre)}\\
\hlline{\ \ \ \ 6\ }\hlstd{\\
\hlline{\ \ \ \ 7\ }bollo\ }\hlsym{=\ }\hlstd{}\hlkwd{Insumo}\hlstd{}\hlsym{(}\hlstd{}\hlnum{30}\hlstd{}\hlsym{,}\hlstd{}\hlnum{10}\hlstd{}\hlsym{,}\hlstd{}\hlstr{"Bollo\ de\ Pizza"}\hlstd{}\hlsym{)}\\
\hlline{\ \ \ \ 8\ }\hlstd{tapa\ }\hlsym{=\ }\hlstd{}\hlkwd{Insumo}\hlstd{}\hlsym{(}\hlstd{}\hlnum{50}\hlstd{}\hlsym{,}\hlstd{}\hlnum{20}\hlstd{}\hlsym{,}\hlstd{}\hlstr{"Tapa\ de\ Empanada"}\hlstd{}\hlsym{)}\\
\hlline{\ \ \ \ 9\ }\hlstd{insumo1\ }\hlsym{=\ }\hlstd{}\hlkwd{Insumo}\hlstd{}\hlsym{(}\hlstd{}\hlnum{10}\hlstd{}\hlsym{,}\hlstd{}\hlnum{3}\hlstd{}\hlsym{,}\hlstd{}\hlstr{"Kit\ Muzzarella"}\hlstd{}\hlsym{)}\\
\hlline{\ \ \ 10\ }\hlstd{insumo2\ }\hlsym{=\ }\hlstd{}\hlkwd{Insumo}\hlstd{}\hlsym{(}\hlstd{}\hlnum{25}\hlstd{}\hlsym{,}\hlstd{}\hlnum{25}\hlstd{}\hlsym{,}\hlstd{}\hlstr{"Kit\ Empanada\ de\ Carne"}\hlstd{}\hlsym{)}\\
\hlline{\ \ \ 11\ }\hlstd{insumo3\ }\hlsym{=\ }\hlstd{}\hlkwd{Insumo}\hlstd{}\hlsym{(}\hlstd{}\hlnum{13}\hlstd{}\hlsym{,}\hlstd{}\hlnum{13}\hlstd{}\hlsym{,}\hlstd{}\hlstr{"Kit\ Napolitana"}\hlstd{}\hlsym{)}\\
\hlline{\ \ \ 12\ }\hlstd{insumo4\ }\hlsym{=\ }\hlstd{}\hlkwd{Insumo}\hlstd{}\hlsym{(}\hlstd{}\hlnum{100}\hlstd{}\hlsym{,}\hlstd{}\hlnum{10}\hlstd{}\hlsym{,}\hlstd{}\hlstr{"Kit\ Empanada\ de\ Pollo"}\hlstd{}\hlsym{)}\\
\hlline{\ \ \ 13\ }\hlstd{insumo5\ }\hlsym{=\ }\hlstd{}\hlkwd{Insumo}\hlstd{}\hlsym{(}\hlstd{}\hlnum{10}\hlstd{}\hlsym{,}\hlstd{}\hlnum{11}\hlstd{}\hlsym{,}\hlstd{}\hlstr{"Kit\ Roquefort"}\hlstd{}\hlsym{)}\\
\hlline{\ \ \ 14\ }\hlstd{insumo6\ }\hlsym{=\ }\hlstd{}\hlkwd{Insumo}\hlstd{}\hlsym{(}\hlstd{}\hlnum{12}\hlstd{}\hlsym{,}\hlstd{}\hlnum{12}\hlstd{}\hlsym{,}\hlstd{}\hlstr{"Botella\ de\ Quilmes"}\hlstd{}\hlsym{)}\\
\hlline{\ \ \ 15\ }\hlstd{insumo7\ }\hlsym{=\ }\hlstd{}\hlkwd{Insumo}\hlstd{}\hlsym{(}\hlstd{}\hlnum{45}\hlstd{}\hlsym{,}\hlstd{}\hlnum{23}\hlstd{}\hlsym{,}\hlstd{}\hlstr{"Botella\ de\ Coca{-}Cola"}\hlstd{}\hlsym{)}\\
\hlline{\ \ \ 16\ }\hlstd{insumo8\ }\hlsym{=\ }\hlstd{}\hlkwd{Insumo}\hlstd{}\hlsym{(}\hlstd{}\hlnum{0}\hlstd{}\hlsym{,}\hlstd{}\hlnum{12}\hlstd{}\hlsym{,}\hlstd{}\hlstr{"Botella\ de\ Iguana"}\hlstd{}\hlsym{)}\\
\hlline{\ \ \ 17\ }\hlstd{insumo9\ }\hlsym{=\ }\hlstd{}\hlkwd{Insumo}\hlstd{}\hlsym{(}\hlstd{}\hlnum{22}\hlstd{}\hlsym{,}\hlstd{}\hlnum{23}\hlstd{}\hlsym{,}\hlstd{}\hlstr{"Botella\ de\ Pepsi"}\hlstd{}\hlsym{)}\\
\hlline{\ \ \ 18\ }\hlstd{insumo10\ }\hlsym{=\ }\hlstd{}\hlkwd{Insumo}\hlstd{}\hlsym{(}\hlstd{}\hlnum{0}\hlstd{}\hlsym{,}\hlstd{}\hlnum{1}\hlstd{}\hlsym{,}\hlstd{}\hlstr{"Kit\ Calabresa"}\hlstd{}\hlsym{)}\\
\hlline{\ \ \ 19\ }\hlstd{insumo11\ }\hlsym{=\ }\hlstd{}\hlkwd{Insumo}\hlstd{}\hlsym{(}\hlstd{}\hlnum{14}\hlstd{}\hlsym{,}\hlstd{}\hlnum{20}\hlstd{}\hlsym{,}\hlstd{}\hlstr{"Kit\ Empanada\ de\ humita"}\hlstd{}\hlsym{)}\\
\hlline{\ \ \ 20\ }\hlstd{insumo12\ }\hlsym{=\ }\hlstd{}\hlkwd{Insumo}\hlstd{}\hlsym{(}\hlstd{}\hlnum{0}\hlstd{}\hlsym{,}\hlstd{}\hlnum{12}\hlstd{}\hlsym{,}\hlstd{}\hlstr{"Kit\ Empanada\ de\ atun"}\hlstd{}\hlsym{)}\\
\hlline{\ \ \ 21\ }\hlstd{}\\
\hlline{\ \ \ 22\ }\hlslc{\#TipoProducto(Nombre,cocinable,preparabl)}\\
\hlline{\ \ \ 23\ }\hlstd{pizza}\hlsym{=\ }\hlstd{}\hlkwd{TipoProducto}\hlstd{}\hlsym{(}\hlstd{}\hlstr{"Pizza"}\hlstd{}\hlsym{,}\hlstd{}\hlkwa{True}\hlstd{}\hlsym{,}\hlstd{}\hlkwa{True}\hlstd{}\hlsym{)}\\
\hlline{\ \ \ 24\ }\hlstd{empanada}\hlsym{=\ }\hlstd{}\hlkwd{TipoProducto}\hlstd{}\hlsym{(}\hlstd{}\hlstr{"Empanada"}\hlstd{}\hlsym{,}\hlstd{}\hlkwa{True}\hlstd{}\hlsym{,}\hlstd{}\hlkwa{True}\hlstd{}\hlsym{)}\\
\hlline{\ \ \ 25\ }\hlstd{coca}\hlsym{=}\hlstd{\ \ }\hlsym{}\hlstd{}\hlkwd{TipoProducto}\hlstd{}\hlsym{(}\hlstd{}\hlstr{"Gaseosa"}\hlstd{}\hlsym{,}\hlstd{}\hlkwa{False}\hlstd{}\hlsym{,}\hlstd{}\hlkwa{False}\hlstd{}\hlsym{)}\\
\hlline{\ \ \ 26\ }\hlstd{cerveza}\hlsym{=\ }\hlstd{}\hlkwd{TipoProducto}\hlstd{}\hlsym{(}\hlstd{}\hlstr{"Cerveza"}\hlstd{}\hlsym{,}\hlstd{}\hlkwa{False}\hlstd{}\hlsym{,}\hlstd{}\hlkwa{False}\hlstd{}\hlsym{)}\\
\hlline{\ \ \ 27\ }\hlstd{}\\
\hlline{\ \ \ 28\ }\hlslc{\#Producto(nombre,precio,tiempoCoccion,tiempoPreparacion,TipoProducto,}\\
\hlline{\ \ \ 29\ }\hlslc{}\hlstd{\ \ \ \ \ \ \ \ \ \ }\hlslc{insumos)}\\
\hlline{\ \ \ 30\ }\hlstd{}\\
\hlline{\ \ \ 31\ }\hlslc{\#bebidas}\\
\hlline{\ \ \ 32\ }\hlstd{Bebida1\ }\hlsym{=\ }\hlstd{}\hlkwd{Producto}\hlstd{}\hlsym{(}\hlstd{}\hlstr{"Coca{-}Cola"}\hlstd{}\hlsym{,}\hlstd{}\hlnum{5}\hlstd{}\hlsym{,}\hlstd{}\hlnum{0.0}\hlstd{}\hlsym{,}\hlstd{}\hlnum{0.0}\hlstd{}\hlsym{,}\hlstd{coca}\hlsym{,{[}}\hlstd{insumo7}\hlsym{{]})\ }\hlstd{}\hlslc{\#no\ critico}\\
\hlline{\ \ \ 33\ }\hlstd{Bebida2\ }\hlsym{=\ }\hlstd{}\hlkwd{Producto}\hlstd{}\hlsym{(}\hlstd{}\hlstr{"Quilmes"}\hlstd{}\hlsym{,}\hlstd{}\hlnum{7}\hlstd{}\hlsym{,}\hlstd{}\hlnum{0.0}\hlstd{}\hlsym{,}\hlstd{}\hlnum{0.0}\hlstd{}\hlsym{,}\hlstd{cerveza}\hlsym{,{[}}\hlstd{insumo6}\hlsym{{]})\ }\hlstd{}\hlslc{\#al\ limite\ de}\\
\hlline{\ \ \ 34\ }\hlslc{}\hlstd{\ \ \ \ \ \ \ \ \ \ }\hlslc{critico}\\
\hlline{\ \ \ 35\ }\hlstd{Bebida3\ }\hlsym{=\ }\hlstd{}\hlkwd{Producto}\hlstd{}\hlsym{(}\hlstd{}\hlstr{"Pepsi\ Cola"}\hlstd{}\hlsym{,}\hlstd{}\hlnum{5}\hlstd{}\hlsym{,}\hlstd{}\hlnum{0.0}\hlstd{}\hlsym{,}\hlstd{}\hlnum{0.0}\hlstd{}\hlsym{,}\hlstd{coca}\hlsym{,{[}}\hlstd{insumo9}\hlsym{{]})\ }\hlstd{}\hlslc{\#\ critico}\\
\hlline{\ \ \ 36\ }\hlstd{Bebida2\ }\hlsym{=\ }\hlstd{}\hlkwd{Producto}\hlstd{}\hlsym{(}\hlstd{}\hlstr{"Iguana"}\hlstd{}\hlsym{,}\hlstd{}\hlnum{7}\hlstd{}\hlsym{,}\hlstd{}\hlnum{0.0}\hlstd{}\hlsym{,}\hlstd{}\hlnum{0.0}\hlstd{}\hlsym{,}\hlstd{cerveza}\hlsym{,{[}}\hlstd{insumo8}\hlsym{{]})\ }\hlstd{}\hlslc{\#\ agotada}\\
\hlline{\ \ \ 37\ }\hlstd{}\\
\hlline{\ \ \ 38\ }\hlslc{\#pizzas}\\
\hlline{\ \ \ 39\ }\hlstd{Pizza1\ }\hlsym{=\ }\hlstd{}\hlkwd{Producto}\hlstd{}\hlsym{(}\hlstd{}\hlstr{"Muzzarella"}\hlstd{}\hlsym{,}\hlstd{}\hlnum{25}\hlstd{}\hlsym{,}\hlstd{}\hlnum{5.0}\hlstd{}\hlsym{,}\hlstd{}\hlnum{10.2}\hlstd{}\hlsym{,}\hlstd{pizza}\hlsym{,{[}}\hlstd{bollo}\hlsym{,}\hlstd{insumo1}\hlsym{{]})\ }\hlstd{}\hlslc{\#no}\\
\hlline{\ \ \ 40\ }\hlslc{}\hlstd{\ \ \ \ \ \ \ \ \ }\hlslc{critico}\\
\hlline{\ \ \ 41\ }\hlstd{Pizza2\ }\hlsym{=\ }\hlstd{}\hlkwd{Producto}\hlstd{}\hlsym{(}\hlstd{}\hlstr{"Napolitana"}\hlstd{}\hlsym{,}\hlstd{}\hlnum{30}\hlstd{}\hlsym{,}\hlstd{}\hlnum{5.4}\hlstd{}\hlsym{,}\hlstd{}\hlnum{12.3}\hlstd{}\hlsym{,}\hlstd{pizza}\hlsym{,{[}}\hlstd{bollo}\hlsym{,}\hlstd{insumo3}\hlsym{{]})\ }\hlstd{}\hlslc{\#al}\\
\hlline{\ \ \ 42\ }\hlslc{}\hlstd{\ \ \ \ \ \ \ \ \ }\hlslc{limite\ de\ critico}\\
\hlline{\ \ \ 43\ }\hlstd{Pizza3\ }\hlsym{=\ }\hlstd{}\hlkwd{Producto}\hlstd{}\hlsym{(}\hlstd{}\hlstr{"Roquefort"}\hlstd{}\hlsym{,}\hlstd{}\hlnum{40}\hlstd{}\hlsym{,}\hlstd{}\hlnum{7}\hlstd{}\hlsym{,}\hlstd{}\hlnum{15.0}\hlstd{}\hlsym{,}\hlstd{pizza}\hlsym{,{[}}\hlstd{bollo}\hlsym{,}\hlstd{insumo5}\hlsym{{]})\ }\hlstd{}\hlslc{\#\ critico}\\
\hlline{\ \ \ 44\ }\hlstd{Pizza4\ }\hlsym{=\ }\hlstd{}\hlkwd{Producto}\hlstd{}\hlsym{(}\hlstd{}\hlstr{"Calabresa"}\hlstd{}\hlsym{,}\hlstd{}\hlnum{40}\hlstd{}\hlsym{,}\hlstd{}\hlnum{7.1}\hlstd{}\hlsym{,}\hlstd{}\hlnum{15.0}\hlstd{}\hlsym{,}\hlstd{pizza}\hlsym{,{[}}\hlstd{bollo}\hlsym{,}\hlstd{insumo10}\hlsym{{]})}\hlstd{}\hlslc{\#Agotada}\\
\hlline{\ \ \ 45\ }\hlstd{}\\
\hlline{\ \ \ 46\ }\hlslc{\#empanadas}\\
\hlline{\ \ \ 47\ }\hlstd{Empanada1\ }\hlsym{=\ }\hlstd{}\hlkwd{Producto}\hlstd{}\hlsym{(}\hlstd{}\hlstr{"Pollo"}\hlstd{}\hlsym{,}\hlstd{}\hlnum{3}\hlstd{}\hlsym{,}\hlstd{}\hlnum{10}\hlstd{}\hlsym{,}\hlstd{}\hlnum{11.0}\hlstd{}\hlsym{,}\hlstd{empanada}\hlsym{,{[}}\hlstd{tapa}\hlsym{,}\hlstd{insumo4}\hlsym{{]})\ }\hlstd{}\hlslc{\#No\ critico}\\
\hlline{\ \ \ 48\ }\hlstd{Empanada2\ }\hlsym{=\ }\hlstd{}\hlkwd{Producto}\hlstd{}\hlsym{(}\hlstd{}\hlstr{"Carne"}\hlstd{}\hlsym{,}\hlstd{}\hlnum{3}\hlstd{}\hlsym{,}\hlstd{}\hlnum{11}\hlstd{}\hlsym{,}\hlstd{}\hlnum{12.0}\hlstd{}\hlsym{,}\hlstd{empanada}\hlsym{,{[}}\hlstd{tapa}\hlsym{,}\hlstd{insumo2}\hlsym{{]})\ }\hlstd{}\hlslc{\#al\ limite}\\
\hlline{\ \ \ 49\ }\hlslc{}\hlstd{\ \ \ \ \ \ \ \ \ \ \ \ }\hlslc{de\ critico}\\
\hlline{\ \ \ 50\ }\hlstd{Empanada3\ }\hlsym{=\ }\hlstd{}\hlkwd{Producto}\hlstd{}\hlsym{(}\hlstd{}\hlstr{"Humita"}\hlstd{}\hlsym{,}\hlstd{}\hlnum{3}\hlstd{}\hlsym{,}\hlstd{}\hlnum{10}\hlstd{}\hlsym{,}\hlstd{}\hlnum{11.0}\hlstd{}\hlsym{,}\hlstd{empanada}\hlsym{,{[}}\hlstd{tapa}\hlsym{,}\hlstd{insumo11}\hlsym{{]})\ }\hlstd{}\hlslc{\#critico}\\
\hlline{\ \ \ 51\ }\hlstd{Empanada4\ }\hlsym{=\ }\hlstd{}\hlkwd{Producto}\hlstd{}\hlsym{(}\hlstd{}\hlstr{"Atun"}\hlstd{}\hlsym{,}\hlstd{}\hlnum{3}\hlstd{}\hlsym{,}\hlstd{}\hlnum{11}\hlstd{}\hlsym{,}\hlstd{}\hlnum{12.0}\hlstd{}\hlsym{,}\hlstd{empanada}\hlsym{,{[}}\hlstd{tapa}\hlsym{,}\hlstd{insumo12}\hlsym{{]})\ }\hlstd{}\hlslc{\#Agotada}\\
\hlline{\ \ \ 52\ }\hlstd{\\
\hlline{\ \ \ 53\ }\\
\hlline{\ \ \ 54\ }dir1\ }\hlsym{=\ }\hlstd{}\hlkwd{Direccion}\hlstd{}\hlsym{(}\hlstd{}\hlstr{"Trinidad"}\hlstd{}\hlsym{,\ }\hlstd{}\hlstr{"N/A"}\hlstd{}\hlsym{,\ }\hlstd{}\hlstr{"tortuguitas"}\hlstd{}\hlsym{,\ }\hlstd{}\hlnum{123}\hlstd{}\hlsym{)}\\
\hlline{\ \ \ 55\ }\hlstd{dir2\ }\hlsym{=\ }\hlstd{}\hlkwd{Direccion}\hlstd{}\hlsym{(}\hlstd{}\hlstr{"Los\ alamos"}\hlstd{}\hlsym{,\ }\hlstd{}\hlstr{"N/A"}\hlstd{}\hlsym{,\ }\hlstd{}\hlstr{"Wilde"}\hlstd{}\hlsym{,\ }\hlstd{}\hlnum{1465}\hlstd{}\hlsym{)}\\
\hlline{\ \ \ 56\ }\hlstd{dir3\ }\hlsym{=\ }\hlstd{}\hlkwd{Direccion}\hlstd{}\hlsym{(}\hlstd{}\hlstr{"San\ Martin"}\hlstd{}\hlsym{,\ }\hlstd{}\hlstr{"2do\ Piso"}\hlstd{}\hlsym{,\ }\hlstd{}\hlstr{"Capital\ Federal"}\hlstd{}\hlsym{,\ }\hlstd{}\hlnum{3988}\hlstd{}\hlsym{)}\\
\hlline{\ \ \ 57\ }\hlstd{dir4\ }\hlsym{=\ }\hlstd{}\hlkwd{Direccion}\hlstd{}\hlsym{(}\hlstd{}\hlstr{"Montaneses"}\hlstd{}\hlsym{,\ }\hlstd{}\hlstr{"1er\ Piso"}\hlstd{}\hlsym{,\ }\hlstd{}\hlstr{"Capital\ Federal"}\hlstd{}\hlsym{,\ }\hlstd{}\hlnum{345}\hlstd{}\hlsym{)}\\
\hlline{\ \ \ 58\ }\hlstd{dir5\ }\hlsym{=\ }\hlstd{}\hlkwd{Direccion}\hlstd{}\hlsym{(}\hlstd{}\hlstr{"Oliden"}\hlstd{}\hlsym{,\ }\hlstd{}\hlstr{"N/A"}\hlstd{}\hlsym{,\ }\hlstd{}\hlstr{"Vicente\ Lopez"}\hlstd{}\hlsym{,\ }\hlstd{}\hlnum{3433}\hlstd{}\hlsym{)}\\
\hlline{\ \ \ 59\ }\hlstd{dir6\ }\hlsym{=\ }\hlstd{}\hlkwd{Direccion}\hlstd{}\hlsym{(}\hlstd{}\hlstr{"Los\ Paraisos"}\hlstd{}\hlsym{,\ }\hlstd{}\hlstr{"N/A"}\hlstd{}\hlsym{,\ }\hlstd{}\hlstr{"Bursaco"}\hlstd{}\hlsym{,\ }\hlstd{}\hlnum{20}\hlstd{}\hlsym{)}\\
\hlline{\ \ \ 60\ }\hlstd{\\
\hlline{\ \ \ 61\ }cli1\ }\hlsym{=\ }\hlstd{}\hlkwd{Cliente}\hlstd{}\hlsym{(}\hlstd{}\hlstr{"Sainz\ Trapaga"}\hlstd{}\hlsym{,\ }\hlstd{}\hlnum{1555664488}\hlstd{}\hlsym{,\ }\hlstd{}\hlnum{1}\hlstd{}\hlsym{,\ }\hlstd{}\hlstr{"Gonzalo"}\hlstd{}\hlsym{,\ }\hlstd{}\hlstr{"hayQPaja"}\hlstd{}\hlsym{,}\\
\hlline{\ \ \ 62\ }\hlstd{}\hlstd{\ \ \ \ \ \ \ }\hlstd{}\hlnum{48566633}\hlstd{}\hlsym{,\ }\hlstd{}\hlstr{"gomox"}\hlstd{}\hlsym{,\ }\hlstd{dir1}\hlsym{)}\\
\hlline{\ \ \ 63\ }\hlstd{cli2\ }\hlsym{=\ }\hlstd{}\hlkwd{Cliente}\hlstd{}\hlsym{(}\hlstd{}\hlstr{"Martinez"}\hlstd{}\hlsym{,\ }\hlstd{}\hlnum{1521356684}\hlstd{}\hlsym{,\ }\hlstd{}\hlnum{2}\hlstd{}\hlsym{,\ }\hlstd{}\hlstr{"Federico"}\hlstd{}\hlsym{,\ }\hlstd{}\hlstr{"estudiando"}\hlstd{}\hlsym{,}\\
\hlline{\ \ \ 64\ }\hlstd{}\hlstd{\ \ \ \ \ \ \ }\hlstd{}\hlnum{46532233}\hlstd{}\hlsym{,\ }\hlstd{}\hlstr{"fedefly"}\hlstd{}\hlsym{,\ }\hlstd{dir2}\hlsym{)}\\
\hlline{\ \ \ 65\ }\hlstd{cli3\ }\hlsym{=\ }\hlstd{}\hlkwd{Cliente}\hlstd{}\hlsym{(}\hlstd{}\hlstr{"Gonzalez"}\hlstd{}\hlsym{,\ }\hlstd{}\hlnum{1523314655}\hlstd{}\hlsym{,\ }\hlstd{}\hlnum{3}\hlstd{}\hlsym{,\ }\hlstd{}\hlstr{"Emiliano"}\hlstd{}\hlsym{,\ }\hlstd{}\hlstr{"bartolo"}\hlstd{}\hlsym{,\ }\hlstd{}\hlnum{45632132}\hlstd{}\hlsym{,}\\
\hlline{\ \ \ 66\ }\hlstd{}\hlstd{\ \ \ \ \ \ \ }\hlstd{}\hlstr{"emilio"}\hlstd{}\hlsym{,\ }\hlstd{dir3}\hlsym{)}\\
\hlline{\ \ \ 67\ }\hlstd{cli4\ }\hlsym{=\ }\hlstd{}\hlkwd{Cliente}\hlstd{}\hlsym{(}\hlstd{}\hlstr{"Gonzalez"}\hlstd{}\hlsym{,\ }\hlstd{}\hlnum{1532169788}\hlstd{}\hlsym{,\ }\hlstd{}\hlnum{4}\hlstd{}\hlsym{,\ }\hlstd{}\hlstr{"Sergio"}\hlstd{}\hlsym{,\ }\hlstd{}\hlstr{"lechuga"}\hlstd{}\hlsym{,\ }\hlstd{}\hlnum{45125398}\hlstd{}\hlsym{,}\\
\hlline{\ \ \ 68\ }\hlstd{}\hlstd{\ \ \ \ \ \ \ }\hlstd{}\hlstr{"checho"}\hlstd{}\hlsym{,\ }\hlstd{dir4}\hlsym{)}\\
\hlline{\ \ \ 69\ }\hlstd{cli5\ }\hlsym{=\ }\hlstd{}\hlkwd{Cliente}\hlstd{}\hlsym{(}\hlstd{}\hlstr{"Rinaldi"}\hlstd{}\hlsym{,\ }\hlstd{}\hlnum{1564659777}\hlstd{}\hlsym{,\ }\hlstd{}\hlnum{5}\hlstd{}\hlsym{,\ }\hlstd{}\hlstr{"Nicolas"}\hlstd{}\hlsym{,\ }\hlstd{}\hlstr{"esteee"}\hlstd{}\hlsym{,\ }\hlstd{}\hlnum{46633221}\hlstd{}\hlsym{,}\\
\hlline{\ \ \ 70\ }\hlstd{}\hlstd{\ \ \ \ \ \ \ }\hlstd{}\hlstr{"elcorrector"}\hlstd{}\hlsym{,\ }\hlstd{dir5}\hlsym{)}\\
\hlline{\ \ \ 71\ }\hlstd{cli6\ }\hlsym{=\ }\hlstd{}\hlkwd{Cliente}\hlstd{}\hlsym{(}\hlstd{}\hlstr{"Darrigo"}\hlstd{}\hlsym{,\ }\hlstd{}\hlnum{1564521133}\hlstd{}\hlsym{,\ }\hlstd{}\hlnum{6}\hlstd{}\hlsym{,\ }\hlstd{}\hlstr{"Sergio"}\hlstd{}\hlsym{,\ }\hlstd{}\hlstr{"niato"}\hlstd{}\hlsym{,\ }\hlstd{}\hlnum{46632136}\hlstd{}\hlsym{,}\\
\hlline{\ \ \ 72\ }\hlstd{}\hlstd{\ \ \ \ \ \ \ }\hlstd{}\hlstr{"jefetp"}\hlstd{}\hlsym{,\ }\hlstd{dir6}\hlsym{)}\\
\hlline{\ \ \ 73\ }\hlstd{\\
\hlline{\ \ \ 74\ }dicc\ }\hlsym{=\ \{}\hlstd{}\hlstr{``Insumos''}\hlstd{}\hlsym{:}\hlstd{Insumo}\hlsym{.}\hlstd{}\hlkwd{allInstances}\hlstd{}\hlsym{(),}\hlstd{}\hlstr{``Productos''}\hlstd{}\hlsym{:}\hlstd{Producto}\hlsym{.}\hlstd{}\hlkwd{allInstances}\hlstd{}\hlsym{(),}\\
\hlline{\ \ \ 75\ }\hlstd{}\hlstd{\ \ \ \ \ \ \ }\hlstd{}\hlstr{``TiposProducto''}\hlstd{}\hlsym{:\{}\hlstd{}\hlstr{``pizza''}\hlstd{}\hlsym{:}\hlstd{pizza}\hlsym{,}\hlstd{}\hlstr{``empanada''}\hlstd{}\hlsym{:}\hlstd{empanada}\hlsym{,}\hlstd{}\hlstr{``coca''}\hlstd{}\hlsym{:}\hlstd{coca}\hlsym{,}\\
\hlline{\ \ \ 76\ }\hlstd{}\hlstd{\ \ \ \ \ \ \ }\hlstd{}\hlstr{``birra''}\hlstd{}\hlsym{:}\hlstd{cerveza}\hlsym{\},\ }\hlstd{}\hlstr{``Clientes''}\hlstd{}\hlsym{:{[}}\hlstd{cli1}\hlsym{,}\hlstd{cli2}\hlsym{,}\hlstd{cli3}\hlsym{,}\hlstd{cli4}\hlsym{,}\hlstd{cli5}\hlsym{,}\hlstd{cli6}\hlsym{{]}\}}\\
\hlline{\ \ \ 77\ }\hlstd{}\hlslc{\#guradamos\ el\ diccionario\ con\ los\ datos\ necesarios\ para\ cargar\ a\ la}\\
\hlline{\ \ \ 78\ }\hlslc{pizzeria}\\
\hlline{\ \ \ 79\ }\hlstd{\\
\hlline{\ \ \ 80\ }pickle}\hlsym{.}\hlstd{}\hlkwd{dump}\hlstd{}\hlsym{(}\hlstd{dicc}\hlsym{,\ }\hlstd{}\hlkwb{open}\hlstd{}\hlsym{(}\hlstd{}\hlstr{``datos.pyp''}\hlstd{}\hlsym{,\ }\hlstd{}\hlstr{``wb''}\hlstd{}\hlsym{))}\\
\hlline{\ \ \ 81\ }\hlstd{\\
\hlline{\ \ \ 82\ }pizzeria\ }\hlsym{=\ }\hlstd{}\hlkwd{Pizzeria}\hlstd{}\hlsym{(}\hlstd{}\hlkwb{open}\hlstd{}\hlsym{(}\hlstd{}\hlstr{``datos.pyp''}\hlstd{}\hlsym{,}\hlstd{}\hlstr{``rb''}\hlstd{}\hlsym{))}\\
\hlline{\ \ \ 83\ }\hlstd{pizzeria}\hlsym{.}\hlstd{}\hlkwd{getCoordP}\hlstd{}\hlsym{().}\hlstd{}\hlkwd{ingresarPedido}\hlstd{}\hlsym{(}\hlstd{cli1}\hlsym{,{[}}\hlstd{Empanada1}\hlsym{{]},}\hlstd{}\hlstr{"efectivo"}\hlstd{}\hlsym{,\ }\hlstd{}\hlstr{"telefono"}\hlstd{}\hlsym{,}\\
\hlline{\ \ \ 84\ }\hlstd{}\hlstd{\ \ \ \ \ \ \ \ \ \ \ \ \ \ \ \ \ \ \ }\hlstd{}\hlkwa{None}\hlstd{}\hlsym{)}\\
\hlline{\ \ \ 85\ }\hlstd{\\
\hlline{\ \ \ 86\ }pickle}\hlsym{.}\hlstd{}\hlkwd{dump}\hlstd{}\hlsym{(}\hlstd{pizzeria}\hlsym{,}\hlstd{}\hlkwb{open}\hlstd{}\hlsym{(}\hlstd{}\hlstr{``pizzeriaTesting''}\hlstd{}\hlsym{,}\hlstd{}\hlstr{``wb''}\hlstd{}\hlsym{))}\hlstd{}\\
\mbox{}
\normalfont
\shorthandon{"}

El problema de este m�todo es que puede generar errores en este archivo que despues generen inconsistencias.

Otra forma de hacerlo es utilizar la propia interfaz grafica. La aplicacion toma los datos sobre insumos,clientes,etc de el archivo datos.pyp, si uno quiere cambiar los datos pisa ese archivo de manera similar a la primera parte del ejemplo anterior.
Luego inicia la aplicacion y mediante el boton de ingresar pedido va metiendo los pedidos necesarios para correr el caso de test con los datos dados.

