\chapter{Testing}
En este apartado presentaremos el dise�o, asi como tambi�n las cuestiones propias a la implementaci�n y realizaci�n del testing funcional sobre la funcionalidad de registro de pedidos, para los casos donde hay a lo sumo un producto comestible y un producto del tipo bebida.

De acuerdo a nuestro modelo, esta funcionalidad esta encapsulada en una llamada al metodo ingresarPedido de la clase CoordinadorDePedidos.

\section{Analisis de factores, categorias y choices}
%basicamente poner las cosas de testing, hacer tablita, etc
Los inputs directos de la funcionalidad son:
\begin{itemize}
\item Cliente
\item Lista de productos
\item Forma de pago
\item Origen
\item N�mero de mesa
\end{itemize}

Hay algunas situacione que por nuestro dise�o no se pueden dar y que por lo tanto no seran tenidas en cuenta, por ejemplo que el cliente no exista, ya que para ingresar un pedido se elige un cliente de entre todos los clientes registrados, si en cambio hay que tener en cuenta el caso en el que no se eligi� ningun cliente. Por otro lado la situacion con la mesa es similar, al igual que con el numero de mesa y el origen %chequear que sea asi
A su vez podemos considerar otros elementos del sistema que modifican el comportamiento de la funcionalidad y que deben ser tenidos en cuenta:
\begin{itemize}
\item Preparadores ocupados o no
\item �Alguna de las colas de los preparadores no esta vacia?
\item �Hay pedidos en la cola de ingreso?
\item Stock de los insumos necesarios para el pedido
\end{itemize}

A partir de esta exploraci�n del domino de la funcion consideramos la siguiente particion en factores,categorias y choices:
\newpage
\pagestyle{empty}
{
\fancypagestyle{plain}
    {
    \fancyhead{}
    \fancyfoot{}
    \renewcommand{\headrulewidth}{0.0pt}
    } % clear header and footer of plain page because of ToC
\begin{landscape}
\begin{tabular}{|c|c|c|c|}
\hline 
Factor & Categoria & Choice & Clasificacion\tabularnewline
\hline
\hline 
Origen & Valor & Mesa & \tabularnewline
 &  & Mostrador & \tabularnewline
 &  & Telefono & \tabularnewline
\hline 
Lista De Productos & Composici�n & 0 Productos & {[}Error]\tabularnewline
 &  & 1 Bebida & {[}�nico]{[}Property hayBebida]\tabularnewline
 &  & 1 Comida & {[}�nico]{[}Property hayComida]\tabularnewline
 &  & 1 Bebida y 1 Comida & {[}Property hayBebidaYComida]\tabularnewline
 & Cantidad de items de bebida & 1 & {[}�nico]{[}if hayBebida or hayBebidaYComida]\tabularnewline
 &  & > 1 & {[}if hayBebida or hayBebidaYComida]\tabularnewline
 & Cantidad de items de comida & 1 & {[}�nico] {[}if hayBebida or hayBebidaYComida]\tabularnewline
 &  & >1 & {[}if hayComida or hayBebidaYComida]\tabularnewline
 & Tipo de comida & Pizza & {[}if hayComida or hayBebidaYComida]\tabularnewline
 &  & Empanada & {[}if hayComida or hayBebidaYComida]\tabularnewline
\hline 
Mesa & N�mero de mesa & Si & {[}if Mesa]\tabularnewline
 &  & No & {[}if Mesa]{[}Error]\tabularnewline
 &  & Si & {[}if not Mesa]{[}Error]\tabularnewline
 &  & No & {[}if not Mesa]\tabularnewline
\hline 
Cliente & Valor & Nulo & {[}if not Telefono]{[}Error]\tabularnewline
 &  & No nulo & {[}if not Telefono]{[}�nico]\tabularnewline
 &  & No nulo & {[}if Telefono]\tabularnewline
 &  & Nulo & {[}if Telefono]{[}Error]\tabularnewline
\hline 
Forma De Pago & Tipo & Nula & {[}if mesa]\tabularnewline
 &  & No Nula & {[}if mesa]{[}Error]\tabularnewline
 &  & Nula & {[}if not mesa]{[}Error]\tabularnewline
 &  & pertenece a {[}Tarjeta,Efectivo] & {[}if Mostrador]\tabularnewline
 &  & Tarjeta & {[}if Telefono]{[}Error]\tabularnewline
 &  & Efectivo & {[}if Telefono]\tabularnewline
\hline 
Relacion entre stock de bebida y los productos & Stock necesario vs cantActual & > cantActual & {[}Error]{[}if hayBebida or hayBebidaYComida]\tabularnewline
 &  & = cantActual & {[}�nico]{[}if hayBebida or hayBebidaYComida]\tabularnewline
 &  & < cantActual & {[}if hayBebida or hayBebidaYComida]\tabularnewline
\hline 
Relacion entre stock de comida y los productos & Stock necesario vs cantActual & > cantACtual & {[}Error]{[}if hayComida or hayBebidaYComida]\tabularnewline
 &  & = cantActual & {[}if hayComida or hayBebidaYComida]\tabularnewline
 &  & < cantActual & {[}if hayComida or hayBebidaYComida]\tabularnewline
\hline 
Cola de ingreso & Cantidad de pedidos encolados & = 0 & {[}�nico]\tabularnewline
 &  & > 0 & \tabularnewline
\hline 
Preparador asociado al tipo de comida & Ocupado & Si & \tabularnewline
 &  & No & \tabularnewline
\hline 
Preparador asociado al otro tipo de comida & Ocupado & Si & {[}�nico]\tabularnewline
 &  & No & \tabularnewline
\hline 
Relacion insumos productos & Cantidad & Ninguno en cantidad critica & {[}�nico]{[}Property Ninguno]\tabularnewline
 &  & Algunos (pero no todos) en cantidad critica & \tabularnewline
 &  & Todos en cantidad critica & {[}�nico]\tabularnewline
\hline 
 & Cantidad vs cantidad necesaria & Ninguno queda en cantidad critica & {[}if Ninguno]\tabularnewline
 &  & Alguno queda en cantidad critica & \tabularnewline
 &  & Todos quedan en cantidad critica & {[}�nico]\tabularnewline
\hline
\end{tabular}
\end{landscape}
}
\pagenumbering{arabic}
\fancypagestyle{plain} {
    \fancyhead[LO]{Gonzalez, Gonzalez, Mart�nez, Sainz-Tr�paga}
    \fancyhead[C]{}
    \fancyhead[RO]{P\'agina \thepage\ de \pageref{LastPage}}
    \fancyfoot{}
    \renewcommand{\headrulewidth}{0.4pt}
}
\pagestyle{plain}

\subsection{Justificaci�n}
En primer lugar consideramos el origen, ya que seg�n si es de mesa, mostrador o telefonico, el comportamiento de la funcionaldiad var�a.

El segundo factor que consideramos es la lista de productos. De esta, son varios los aspectos que nos interesan



\section{Dise�o de los casos de test}
\begin{landscape}
\begin{tabular}{|c|c|c|c|c|c|c|c|c|c|c|c|c|c|c|c|}
\hline 
Origen & L de Prods & L de Prods & L de Prods & L de Prods & Mesa & Cliente & Pago & Rel stock B y los prods & Rel stock C y los prods & C Ing & Prep asoc & Prep & Rel Ins prods & Rel Ins prods & Resultado\tabularnewline
\hline
\hline 
Valor & Composici�n & Cant items B & Cant items C & Tipo de C & N� & Valor & Tipo & nec vs actual & nec vs actual & Cant & Ocupado & Ocupado & Cant & cant vs nec & Esperado\tabularnewline
\hline
\hline 
- & 0 prods & - & - & - & - & - & - & - & - & - & - & - & - & - & Excepci�n ProductosVacios. Se muestra mensaje informando del error.
El pedido no queda registrado\tabularnewline
\hline 
Mesa & 1 B y 1 C & > 1 & > 1  & Pizza & No & - & - & - & - & - & - & - & - & - & Excepcion PedidoDeMesaSinMesa. Se muestra mensaje informando del error.
El pedido no queda registrado \tabularnewline
\hline 
Mesa & 1 B y 1 C & > 1 & > 1  & Pizza & Si & Nulo & No Nulo & - & - & - & - & - & - & - & Excepci�n TipoDePagoInvalido. Se notifica del error. El pedido no
se registra\tabularnewline
\hline 
Mesa & 1 B y 1 C & > 1 & > 1  & Pizza & Si & Nulo & Nulo & > cantActual & - & - & - & - & - & - & Excepci�n ProductoInsatisfacible. Se notifica que un prod no se pudo
satisfacer. El pedido no se registra\tabularnewline
\hline 
Mesa & 1 B y 1 C & > 1 & > 1  & Pizza & Si & Nulo & Nulo & < cantActual & > cantActual & - & - & - & - & - & Excepci�n ProductoInsatisfacible. Se notifica que un prod no se pudo
satisfacer. El pedido no se registra\tabularnewline
\hline 
Mesa & 1 B y 1 C & > 1 & > 1  & Pizza & Si & Nulo & Nulo & < cantActual & < cantActual & >0 & Si & No & Algunos & Alguno  & Se registra un pedido de mesa, con la mesa del input, con los productos
ingresados. Se encola al final de la cola de ingreso, en el horno
pizzero. Se decrementan los stocks de los insumos involucrados y se
actualiza la lista de stock para mostrar que algunos quedaron en stock
critico. Se estiman los tiempos y se calcula el precio de acuerdo
a lo especificado\tabularnewline
\hline 
Mesa & 1 B y 1 C & > 1 & > 1  & Pizza & Si & Nulo & Nulo & < cantActual & < cantActual & >0 & No & No & Algunos & Alguno & Se registra un pedido de mesa, con la mesa del input, con los productos
ingresados. Se notifica que la pizza debe comenzar a prepararse. Se
decrementan los stocks de los insumos involucrados y se actualiza
la lista de stock para mostrar que algunos quedaron en stock critico.
Se estiman los tiempos y se calcula el precio de acuerdo a lo especificado\tabularnewline
\hline 
Mesa & 1 B y 1 C & > 1 & > 1  & Empanada & Si & Nulo & Nulo & < cantActual & < cantActual & >0 & Si & No & Algunos & Alguno & Se registra un pedido de mesa, con la mesa del input, con los productos
ingresados. Se encola al final de la cola de ingreso, en el horno
empanadero. Se decrementan los stocks de los insumos involucrados
y se actualiza la lista de stock para mostrar que algunos quedaron
en stock critico. Se estiman los tiempos y se calcula el precio de
acuerdo a lo especificado\tabularnewline
\hline 
Mesa & 1 B y 1 C & > 1 & > 1  & Empanada & Si & Nulo & Nulo & < cantActual & < cantActual & >0 & No & No & Algunos & Alguno & Se registra un pedido de mesa, con la mesa del input, con los productos
ingresados. Se notifica que las empanadas debe comenzar a prepararse.
Se decrementan los stocks de los insumos involucrados y se actualiza
la lista de stock para mostrar que algunos quedaron en stock critico.
Se estiman los tiempos y se calcula el precio de acuerdo a lo especificado\tabularnewline
\hline 
Mesa & 1 B y 1 C & > 1 & > 1  & Empanada & Si & No Nulo & Nulo & < cantActual & < cantActual & >0 & No & No & Algunos & Alguno & Se registra un pedido de mesa, con la mesa del input, con los productos
ingresados. Se encola al final de la cola de ingreso, en el horno
empanadero. Se decrementan los stocks de los insumos involucrados
y se actualiza la lista de stock para mostrar que algunos quedaron
en stock critico. El cliente asociado es el Cliente del input. Se
estiman los tiempos y se calcula el precio de acuerdo a lo especificado\tabularnewline
\hline 
Mostrador & 1 B y 1 C & >1 & >1 & Empanada & Si & - & - & - & - & - & - & - & - & - & Excepcion PedidoDeMostradorConMesa. Se muestra un mensaje de error.
No se registra el pedido\tabularnewline
\hline 
Mostrador & 1 B y 1 C & > 1 & > 1  & Empanada & No & Nulo & Nulo & - & - & - & - & - & - & - & Excepcion TipoDePagoInvalido. Se muestra un mensaje informando de
la situaci�n. El pedido no se registra\tabularnewline
\hline 
Mostrador & 1 B y 1 C & > 1 & > 1  & Empanada & No & Nulo & T o F & < cantActual & < cantActual & >0 & Si & No & Algunos & Alguno & Se registra un pedido de mostrador, con los productos ingresados.
Se encola en la cola de ingreso. Se decrementan los stocks de los
insumos involucrados y se actualiza la lista de stock para mostrar
que algunos quedaron en stock critico. Se estiman los tiempos y se
calcula el precio de acuerdo a lo especificado. Se asigna el horno
empanadero. La forma de pago es la del input\tabularnewline
\hline 
Mostrador & 1 B y 1 C & > 1 & > 1  & Empanada & No & Nulo & T o F & < cantActual & < cantActual & >0 & No & No & Algunos & Alguno & Se registra un pedido de mostrador, con los productos ingresados.
Se notifica que debe preparase. Se decrementan los stocks de los insumos
involucrados y se actualiza la lista de stock para mostrar que algunos
quedaron en stock critico. Se estiman los tiempos y se calcula el
precio de acuerdo a lo especificado. Se asigna el horno empanadero.
La forma de pago es la del input\tabularnewline
\hline 
Mostrador & 1 B y 1 C & > 1 & > 1  & Pizza & No & Nulo & T o F & < cantActual & < cantActual & >0 & Si & No & Algunos & Alguno & Se registra un pedido de mostrador, con los productos ingresados.
Se encola en la cola de ingreso. Se decrementan los stocks de los
insumos involucrados y se actualiza la lista de stock para mostrar
que algunos quedaron en stock critico. Se estiman los tiempos y se
calcula el precio de acuerdo a lo especificado. Se asigna el horno
pizzero. La forma de pago es la del input\tabularnewline
\hline 
Mostrador & 1 B y 1 C & > 1 & > 1  & Pizza & No & Nulo & T o F & < cantActual & < cantActual & >0 & No & No & Algunos & Alguno & Se registra un pedido de mostrador, con los productos ingresados.
Se notifica que debe preparase. Se decrementan los stocks de los insumos
involucrados y se actualiza la lista de stock para mostrar que algunos
quedaron en stock critico. Se estiman los tiempos y se calcula el
precio de acuerdo a lo especificado. Se asigna el horno pizzero. La
forma de pago es la del input\tabularnewline
\hline 
Telefono & 1 B y 1 C & > 1 & >1 & Empanada & No & Nulo & - & - & - & - & - & - & - & - & Excepcion Cliente Nulo. Se notifica mediante un mensaje. El pedido
no se registra\tabularnewline
\hline 
Telefono & 1 B y 1 C & > 1 & > 1  & Empanada & No & No nulo & Tarjeta & - & - & - & - & - & - & - & Excepcion TipoDePagoInvalido. Se muestra un mensaje informando la
situaci�n. El pedido no queda registrado\tabularnewline
\hline 
Telefono & 1 B y 1 C & > 1 & > 1  & Empanada & No & No nulo & Efectivo & < cantActual & < cantActual & >0 & Si & No & Algunos & Alguno & Se registra un pedido telefonico, con los productos ingresados. Se
encola en la cola de ingreso. Se decrementan los stocks de los insumos
involucrados y se actualiza la lista de stock para mostrar que algunos
quedaron en stock critico. Se estiman los tiempos y se calcula el
precio de acuerdo a lo especificado. Se asigna el horno empanadero.
La forma de pago es la del input\tabularnewline
\hline 
Telefono & 1 B y 1 C & > 1 & > 1  & Empanada & No & No nulo & Efectivo & < cantActual & < cantActual & >0 & No & No & Algunos & Alguno & Se registra un pedido de mostrador, con los productos ingresados.
Se notifica que debe preparase. Se decrementan los stocks de los insumos
involucrados y se actualiza la lista de stock para mostrar que algunos
quedaron en stock critico. Se estiman los tiempos y se calcula el
precio de acuerdo a lo especificado. Se asigna el horno empanadero.
La forma de pago es la del input\tabularnewline
\hline 
Telefono & 1 B y 1 C & > 1 & > 1  & Pizza & No & No nulo & Efectivo & < cantActual & < cantActual & >0 & Si & No & Algunos & Alguno & Se registra un pedido de mostrador, con los productos ingresados.
Se encola en la cola de ingreso. Se decrementan los stocks de los
insumos involucrados y se actualiza la lista de stock para mostrar
que algunos quedaron en stock critico. Se estiman los tiempos y se
calcula el precio de acuerdo a lo especificado. Se asigna el horno
pizzero. La forma de pago es la del input\tabularnewline
\hline 
Telefono & 1 B y 1 C & > 1 & > 1  & Pizza & No & No nulo & Efectivo & < cantActual & < cantActual & >0 & No & No & Algunos & Alguno & Se registra un pedido de mostrador, con los productos ingresados.
Se notifica que debe preparase. Se decrementan los stocks de los insumos
involucrados y se actualiza la lista de stock para mostrar que algunos
quedaron en stock critico. Se estiman los tiempos y se calcula el
precio de acuerdo a lo especificado. Se asigna el horno pizzero. La
forma de pago es la del input\tabularnewline
\hline 
Mostrador & 1 Bebida & =1 & No aplica & No aplica & No & Nulo & Tarjeta & = cantActual & No aplica & >0 & No aplica & No aplica & Algunos & Alguno & Se registra el pedido de mostrador de solo bebidas. El pedido queda
en la cola de listos. Se decrementa el stock de la bebida. Se notifica
que hay stocks que quedan en stock critico. No se asigna horno. La
forma de pago es la del input. El tiempo estimado es 0. Se calcula
el precio\tabularnewline
\hline 
Telefono & 1 Comida & No aplica & =1  & Pizza & No & No Nulo & Efectivo & No aplica & = cantActua & =0 & Si & Si & Algunos & Alguno & Se registra el pedido. El pedido queda en la cola de ingreso. Se decrementa
el stock de la comida. Se notifica que hay stocks que quedan en stock
critico. Se asigna el horno pizzero. La forma de pago es la del input.
Se estima el tiempo y se calcula el precio\tabularnewline
\hline
\hline 
Mostrador  & 1 B y 1 C & > 1  & > 1  & Empanada & No  & Nulo & Efectivo & < cantActual & < cantActual & >0 & No & Si & Ninguno & Ninguno & Se registra el pedido. Se notifica que hay que preparar las empanadas.
Se decrementa el stock de la comida. No se notifican nuevos productos
en stock critico. Se asigna el horno empanadero. La forma de pago
es la del input. Se estima el tiempo y se calcula el precio.\tabularnewline
\hline
\hline 
Mostrador  & 1 B y 1 C & > 1  & > 1  & Empanada & No  & Nulo & Efectivo & < cantActual & < cantActual & >0 & No & Si & Todos & Todos & Se registra el pedido. Se notifica que hay que preparar las empanadas.
Se decrementa el stock de la comida. Se notifican productos en stock
critico. Se asigna el horno empanadero. La forma de pago es la del
input. Se estima el tiempo y se calcula el precio.\tabularnewline
\hline
\end{tabular}
\end{landscape}

\section{Realizaci�n del testing}
%contar algo de como se hicieron los test, onda valores para las instancias, resultados, etc. se puede hacer una tabla
