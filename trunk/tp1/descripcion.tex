\chapter{Descripci�n General}
\textcolor{Blue}{Esta secci�n busca describir los factores generales que afectan al producto y sus requerimientos. No se detallan aqu� los requerimientos espec�ficos, pero provee un conocimiento b�sico general para dichos requerimientos.}

\section{Perspectiva del producto}
\textcolor{Blue}{Esta secci�n encuadra el producto en la perspectiva con otros productos relacionados. Si el producto es independiente y totalmente autocontenido, dicha situaci�n deber�a explicitarse aqu�.}

\section{Funciones principales del producto} 
\textcolor{Blue}{Esta secci�n deber�a proveer un resumen de las funciones principales que el producto debe realizar. }

\section{Caracter�sticas de los usuarios}
\textcolor{Blue}{Esta secci�n deber�a describir las caracter�sticas generales de los usuarios para los que est� pensado el producto.} 

\section{Restricciones}
\textcolor{Blue}{Esta secci�n deber�a proveer una descripci�n general de cualquier cuesti�n que limite las opciones del desarrollador (ej, regulaciones, limitaciones de hardware, requerimientos sobre lenguajes de alto nivel, etc.)}

\section{Supuestos y dependencias}
\textcolor{Blue}{Esta secci�n deber�a enumerar cada uno de los factores que afectan los requerimientos declarados en el documento. Estos factores no son restricciones de dise�o del software, sino m�s bien representan cuestiones cuyo cambio puede afectar los requerimientos.}
