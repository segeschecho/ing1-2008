\section{Gesti�n de clientes}
Este componente es el responsable de la gesti�n de los datos de clientes. Adem�s
de las funcionalidades b�sicas de ABM, este componente permite acciones tales
como validar clientes para realizar las operaciones que requieren de autenticaci�n.

Como se describiera en \ref{metodosEstaticos}, la clase Cliente tiene m�todos 
est�ticos as� como m�todos convencionales que permiten llevar a cabo las tareas
de ABM. Sin embargo, como tambi�n es necesaria la funcionalidad adicional de validaci�n
de usuarios, en este caso se implementa un objeto Controlador que es responsable de
llevar a cabo estas tareas. Esto se hace para respetar SRP.

Las implementaciones de los m�todos de estas clases son relativamente sencillas.
El Controlador realiza b�squedas en el conjunto de instancias para chequear los
datos requeridos por la interfaz, y responde en funci�n de los mismos. En principio
no parecen ser necesarios algoritmos sofisticados de b�squeda, pero eventualmente
se podr�a modificar la implementaci�n de dichos m�todos manteniendo su interfaz
para mejorar la eficiencia si se lo juzgara necesario. La encapsulaci�n que provee
el objeto hace que estos cambios no afecten al resto del sistema puesto que tanto
la signatura como la sem�ntica de los m�todos se mantendr� intacta.

\subsection{Modelado de escenarios}

% TODO: hacer algun DS, la verdad que no hay gran cosa para poner ac�
\textcolor{Red}{TODO: escenarios que muestren el comportamiento de estas clases en los fenomenos pedidos en el enunciado}
