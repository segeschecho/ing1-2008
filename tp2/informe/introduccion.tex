\chapter{Introducci�n}

\section{Consideraciones sobre el informe}

El siguiente informe encompasa el dise�o que desarrollamos para la implementaci�n del
software de pizzer�a que fuera especificado en el Trabajo Pr�ctico 1. Nuestros objetivos
a la hora de dise�ar se enfocaron esencialmente en la flexibilidad y mantenibilidad del
software, con el objeto de que introducir cambios y correcciones en el software en
el futuro sea tan sencillo como sea posible.

En esta primera secci�n se describe brevemente el siguiente informe, as� como
su organizaci�n y contenidos y los objetivos que se plantea. En particular, este
documento est� dirigido a los desarrolladores que deber�n implementar el sistema.
Si bien no es de car�cter definitivo, el esp�ritu del documento intenta no ser
ambiguo y tener en cuenta tantas decisiones de dise�o significativas como sea
posible.

Al realizar el dise�o fue necesario en algunos casos introducir modificaciones o mejoras
a lo anteriormente especificado. Dichos cambios se evaluaron y fueron juzgados necesarios
por las ventajas que representan en varios planos, pero particularmente en la extensibilidad
y flexibilidad del sistema para tolerar cambios futuros en su modo de operaci�n. La segunda
secci�n del informe describe los cambios realizados as� como la justificaci�n de los mismos
y el impacto que tienen en el dise�o.

En tercer lugar se presenta la descripci�n del dise�o propiamente dicha. Como se adopt�
un acercamiento por componentes, se describe inicialmente esta divisi�n en componentes,
sus ventajas y limitaciones as� como la raz�n de utilizar esta modalidad. 

Finalmente se presenta el dise�o de cada componente, as� como las relaciones que existen entre los
objetos de los mismos. Se utiliza la herramienta de diagramas de secuencia
para modelar las comunicaciones entre los objetos (dentro de un mismo componente, o
entre varios de ellos). En los casos en que fue considerado necesario, se adjunta adem�s
el pseudoc�digo de algunos algoritmos de inter�s.

\section{Convenciones de notaci�n}

A lo largo de este documento se utilizan las siguientes convenciones:
\begin{itemize}
\item \textbf{DIP}: \textit{Dependency Inversion Principle}
\item \textbf{OCP}: \textit{Open/Closed Principle}
\item \textbf{SRP}: \textit{Single Responsibility Principle}
\item \textbf{LSP}: \textit{Liskov Substitution Principle}
\item \textbf{ISP}: \textit{Interface Segregation Principle}
\end{itemize}

En los diagramas de secuencia muchas veces la sintaxis no se ajusta exactamente a la descripta
en clase (por ejemplo, en el caso de las variables locales a un \textit{loop}). Esto se debe en parte
a limitaciones de la herramienta utilizada para realizar los diagramas, y en parte a que la sintaxis
propuesta no es UML y por lo tanto no est� soportada.
