\section{Gesti�n de stock y productos}
Como dijimos anteriormente, este componente es el que brinda las funcionalidades de ABM de stock y productos, asi como tambi�n permite acceder a la informaci�n sobre los diversos insumos y productos.

Sus clases principales son basicamente la clase Insumo y la clase Producto. 

Para realizar los ABM se decidio que la GUI utilice directamente los metodos de las clases antes nombradas, de modo que para crear un Insumo, lo que hace es invocar el new de Insumo, consideramos que esto si bien acopla un poco la GUI al sistema, creemos que no era necesario hacer un \textit{proxy} entre la GUI y estas clases, ya que la interacci�n era simple.

Otra clase propia de este componente es el Tipo de producto, que permite identificar por ejemplo a las pizzas, a las empanadas, asi como tambi�n, decir si un tipo de producto es cocinable, y$/$o preparable.
\textcolor{Red}{TODO: interacciones de estas clases con la GUI}


\subsection{Modelado de escenarios}
%TODO: no me queda claro como va a ser la interaccion aca, explicar asi hago los diagramas
\textcolor{Red}{TODO: escenarios que muestren el comportamiento de estas clases en los fenomenos pedidos en el enunciado}

\textcolor{Red}{TODO: explicacion de metodos importantes}
