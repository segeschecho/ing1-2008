\section{Gesti�n de stock y productos}

Este componente se encarga de la gesti�n de los modelos correspondientes a
productos, tipos de procuto, insumos y las operaciones de alta y baja de los
mismos, as� como de las actualizaciones de sus valores.

Mediante la funcionalidad de ABM provista como se describe en \ref{metodosEstaticos}
es posible realizar los cambios necesarios al stock, tales como modificaciones de
precios o el ingreso de nuevos productos.

El Controlador de Stock es una entidad abstracta que se encarga de las modificaciones
de los valores de stock frente al ingreso o cancelaci�n de pedidos. Esta abstracci�n
se introduce para respetar DIP, y puesto que podr�a ser interesante brindar funcionalidad
m�s sofisticada en esta entidad. Por ejemplo, un controlador de stock m�s inteligente podr�a
realizar un seguimiento individual de todas las transacciones de stock llevadas a cabo
que permitir�a \textit{trackear} cada insumo.

El Controlador de Stock es responsable de emitir el evento de notificaci�n de stock
cr�tico, al que la GUI se suscribe para poder indicar al usuario que el stock requiere
de su atenci�n.

Las clases principales en lo que a modelos de datos respecta dentro de este componente
son Insumo y Producto. La clase TipoProducto permite reconocer cuando varios productos
tienen el mismo tipo y por tanto pueden tratarse de forma an�loga en cuanto a su
preparaci�n y cocci�n. 

El tipo de producto permite especificar adem�s si el producto es
cocinable o preparable. Si bien en el sistema actual los productos son ya sea preparable
y cocinables o ninguno de los dos, en el futuro la pizzer�a podr�a desear, por ejemplo,
comercializar ensaladas que solo requieren de preparaci�n y no de cocci�n. La existencia
de estos atributos permite flexibilidad adicional para extender la operatoria del
restaurante. Si bien esta elecci�n de atributos puede parecer limitante o arbitraria,
evaluamos que es factible categorizar de esta manera a cualquier tipo de producto
que podr�a venderse en una pizzer�a. Por lo tanto, no nos pareci� razonable agregar
complejidad al modelo agregando ``propiedades'' gen�ricas a los tipos de producto
(propiedades de las que \textit{cocinable} y \textit{preparable} ser�an un caso particular).

% TODO: hablar del repositorStock


\subsection{Modelado de escenarios}
% TODO: traer los diagramas de la parte de ingreso de pedidos

\textcolor{Red}{TODO: explicacion de metodos importantes}
% FIXME: vale la pena? creo que co los DS que estan en la parte de ingreso de pedidos alcanza y sobra
