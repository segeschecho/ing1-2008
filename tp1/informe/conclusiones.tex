\chapter{Anexo III: Conclusiones}

El trabajo nos permiti� comprender la dificultad y la enorme cantidad de trabajo que involucra la realizaci�n de una especificaci�n completa para un sistema inform�tico usando el modelo de \textit{Software Requirements Specification}. Tuvimos la oportunidad de ejercitar la gran mayor�a de las t�cnicas propuestas en clase que nos fueron de utilidad para ilustrar nuestra especificaci�n y facilitar su comprensi�n. Esto nos permiti� comprender mejor las aptitudes y limitaciones de cada t�cnica, as� como en qu� contextos es conveniente o no el uso de cada una.

Por el alto grado de acoplamiento que existe entre los elementos de un documento de este tipo, el trabajo represent� un desaf�o tambi�n desde el punto de vista de la coordinaci�n de los esfuerzos y de los miembros del equipo, para lograr ser productivos sin producir inconsistencias en el documento resultante.

\subsection{Dificultades en la realizaci�n del trabajo}

Durante la realizaci�n del trabajo enfrentamos varias dificultades extraordinarias que estaban fuera de nuestras manos. Quer�amos dejar constancia de las mismas en esta conclusi�n porque consideramos que afectaron seriamente la calidad del trabajo final. 

En primer lugar, es destacable el enorme salto de complejidad entre el trabajo pr�ctico y los ejercicios resueltos en las clases. Si bien entendemos que el trabajo debe ser de una dificultad mayor, hubo casos paradigm�ticos donde los contenidos fueron vistos de forma muy superficial o directamente no fueron presentados en clase, poni�ndonos en la necesidad de improvisar. En particular, el modelo de operaciones no fue visto en clase por fuera de su breve introducci�n en la te�rica, y los ejercicios presentados en la pr�ctica respecto de modelo conceptual eran de una dificultad significativamente menor a la del trabajo pr�ctico. No ayuda en esta situaci�n que la postura de los docentes sea que debemos desevolvernos con las herramientas dadas en clase, dado que en muchos casos el tratamiento de las mismas fue insuficiente.
% FIXME: si la ultima frase es muy bardera borrala

En segundo lugar, muchas veces nos encontramos bloqueados a la espera de que se resolvieran cuestiones sobre la operatoria que no fueron definidas correctamente. A dos semanas de comenzado el TP, todav�a se estaban resolviendo en clase. Fue el caso, por ejemplo, del funcionamiento de la cola de pedidos, que no estaba definido claramente en el documento de especificaci�n y fue definido bastante m�s adelante. Por esta raz�n pasamos muchos d�as, a veces incluso fines de semana enteros sin poder avanzar. En funci�n de esto, las primeras semanas del trabajo fueron muy poco productivas debiendo apurarnos sobre el final, una vez que ten�amos todas las herramientas, para lograr terminar en el plazo especificado.

La utilidad del TP0 se vio seriamente comprometida por problemas de aparente desinformaci�n entre los docentes. Cuando consultamos sobre el grado de refinamiento (o el criterio para hacerlo de forma selectiva) para el modelo de objetivos, obtuvimos respuestas diversas, predominantemente del tipo ``est� a su criterio, omitan lo que no importa''. Cuando nos corrigieron el trabajo, nos indicaron que hab�a refinamientos que no eran completos desde un punto de vista formal. Finalmente, cuando fue entregada la plantilla para el documento del TP1, se detallaba en ella un criterio preciso de qu� cosas se deben refinar y qu� cosas no, oblig�ndonos a rever lo producido y duplicar esfuerzos.

Por �ltimo, nunca fue definida en las clases pr�cticas la idea general del trabajo pr�ctico. Dicho entendimiento se dificult� a�n m�s en la medida de que las consultas a diversos ayudantes generaban respuestas diametralmente opuestas. No fue hasta que en la te�rica Sebasti�n Uchitel dijo que el documento que deb�amos preparar se llamaba SRS que pudimos investigar por nuestra cuenta cual era el objetivo de un documento de este tipo.

