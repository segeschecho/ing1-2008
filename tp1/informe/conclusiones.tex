\section{Anexo III: Conclusiones}
\subsection{Dificultades en la realizaci�n del trabajo}
Un aspecto que queriamos comentar sobre la realizaci�n de este trabajo es la cantidad de dificultades que se nos presentaron, las cuales no son, por lo menos en su totalidad, responsabilidad nuestra.

En primer lugar queremos hacer notar el salto entre el trabajo practico y los ejercicios resueltos en las clases. Si bien entendemos que el trabajo debe ser de una dificultada mayor, queremos remarcar que hubo casos, por ejemplo modelo de operaciones, donde los contenidos no se dieron, viendonos practicamente en la necesidad de improvisar, o por ejemplo en el caso del modelo conceptual, los ejercicios hechos en clase eran de una dificultad mucho mas baja.

En segundo lugar queremos comentar que muchas veces nos encontramos bloqueados a la espera de que se resolvieran cuestiones sobre la operatoria, como por ejemplo el funcionamiento de la cola de pedidos, lo cual hacia que pasaramos dias, en particular fines de semana enteros sin poder avanzar.

Por otro lado en general nos costo entender la idea a la que apuntaba el trabajo practico, entendimiento que se hacia mas dificil en la medida de que las consultas a diversos ayudantes generaban respuestas diametralmente diferentes, por ejemplo en lo que se debia modelar con el modelo conceptual. En este sentido queremos comentar que el trabajo practico 0 no consigui� orientarnos.

\subsection{Conclusiones generales}
Mas alla de las dificultades antes planteadas, queremos decir que el trabajo practico permiti� que comprendieramos la complejidad existente en la creacion de un SRS completo, que aplicaramos las diversas tecnicas vistas en clase observando sus capacidades asi como tambi�n sus limitaciones.