\documentclass[a4paper,10pt]{article}

%%%%%%%%%%%%%%%%%%%%%%%%%%%%
% Imports                  %
%%%%%%%%%%%%%%%%%%%%%%%%%%%%

\usepackage{amsmath, amsthm}
\usepackage[spanish,activeacute]{babel}
\usepackage[medium,compact]{titlesec}
\usepackage{enumitem}
\usepackage{a4wide}
\usepackage{hyperref}
\usepackage{fancyhdr}
\usepackage{graphicx} % Para el logo magico!
\usepackage{amssymb}
\usepackage{amsmath}
\usepackage[latin1]{inputenc}
\usepackage{subfigure}
\usepackage[dvipsnames,usenames]{color}
\usepackage{amsfonts}
\usepackage{pifont}

%%%%%%%%%%%%%%%%%%%%%%%%%%%%
% Opciones                 %
%%%%%%%%%%%%%%%%%%%%%%%%%%%%

\parskip    = 11 pt
\headheight	= 13.1pt

\addtolength{\headwidth}{1.0in}
\addtolength{\oddsidemargin}{-0.5in}
\addtolength{\textwidth}{1in}
\addtolength{\topmargin}{-0.7in}
\addtolength{\textheight}{1in}

\flushbottom

\newenvironment{packed_itemize}{
\begin{itemize}[topsep=0pt, partopsep=0pt]
\setlength{\itemsep}{1pt}
\setlength{\parskip}{0pt}
\setlength{\parsep}{0pt}
}{\end{itemize}}

%%%%%%%%%%%%%%%%%%%%%%%%%%%%
% Documento                %
%%%%%%%%%%%%%%%%%%%%%%%%%%%%


\begin{document}

\begin{center}
\textbf{\Large{TODO's}}\\[0.5cm]
\end{center}

\newcommand{\tick}[0]{{\textcolor{Green}{$\checkmark$}}}
\newcommand{\unresolved}[0]{{\textcolor{Goldenrod}{$\chi$}}}
\newcommand{\dudoso}[0]{{\textcolor{SkyBlue}{\textbf{$?$}}}}
Por favor no borrar los todos ya resueltos, colorearlos o marcarlos con \tick.

Si se revisa un todo y no se pudo resolver marcarlo con \unresolved

Si se revisa pero se tienen serias dudas sobre lo realizado marcar con \dudoso



\textcolor{Peach}{\huge{Diagrama de objetivos}}
\begin{itemize}
\item cambiar cuadradito de mozo con PDA por ``atender pedidos de mozos con PDA'' y encajarselo al mozo (requerimiento nuevo: poder recibir los pedidos de los mozos) \tick
\item cambiar cuadrito de due�o responsable de tercerizar como servicio de delivery responsable de atencion por sistema de delivery \tick
\item en lograr el cliente hace pedido por telefono agregar el encargado ingresa los pedidos telefonicos de clientes registrados \tick
\item en mantener base de datos de clientes hay q poner algo como el encargado de pedidos registra usuarios en la base de datos \tick
\item refinar la parte de pagar con tarjeta de debito, efectivo y tarjeta de credito. el sistema debe enterarse y ver si hay q agregar alguna expectativa \unresolved
\item poner q al cliente se le informa la demora aproximada, y q esto requiere cargar tiempos de delivery (probablemente CU) \unresolved
\item ver si no hay q unir mantener control de stock con lograr buen rendimiento de cocina \tick
\item desgranar el sistema puede funcionar en modo restringido \dudoso
\item refinar se puede modificar estado del pedido a lo largo de su ciclo de vida
\item se controlan cancelaciones: ojo q dice q se guarda el pedido y eso no lo queresmo, ademas tenemos el CU cancelar pedido, deberia haber un req llamado se pueden cancelar pedidos
\item fletar la parte de pedidos de antemano
\item no se q onda esta marcado el req de alertar al responsable de stock de stock critico
\item diferenciar abm de stock con abm de productos (ojo que esto puede dar algun caso de uso)
\end{itemize}

\textcolor{Blue}{\huge{Cosas varias}}
\begin{itemize}
\item Eliminar idea de en q sector esta cada cosa (del alcance)
\item el sistema no registra cobros (del alcance)
\item fragmentar el estado de al horno
\item explicar indicadores estadisticos
\item definir combo
\item definir que es perfil de usuario
\item agregar expectativa delivery entrega SMS
\item Eliminar preparacion anticipada de productos (las cancelaciones van a requerir tirar cosas)
\item al ingresar pedidos se informa la demanda aproximada
\end{itemize}

\textcolor{Red}{\huge{Cosas del diagrama de contexto}}
\begin{itemize}
\item buscar formas de partir diagrama de contexto en diagramas chiquitos (conservar el original)
\item relacionar sistema con banco
\item cajero pide al sistema que genere archivo de factura. no solicita factura, no registra cobro
\item granular lo que pide el maestro (proxima cosa a cocinar, no proximo pedido, etc)
\item granular los cambios de estado que hace el maestro (no solo de pedidos, sino de cachos)
\item agregar interacciones entre clietne y sistema (ej sistema manda codigo, informa demanda aproximada)
\item desarrollar la idea de contingencia, es decir texto no solo el diagrama
\item hacer el diagrama de contingencia sin referencias
\end{itemize}

\textcolor{ForestGreen}{\huge{Cosas del diagrama de CU}}
\begin{itemize}
\item agregar actores a pagando con tarjeta \unresolved
\item numerar casos de uso en el diagrama \tick
\item partir cu consultando estado de pedido \tick
\item partir cu informando proximo pedido \unresolved
\item partir modificando estado de pedido \tick
\item sacar estableciendo pedido de cola \textcolor{Red}{aca hay q ver q onda, sobre esto hay q profundizar la comprension de q se quiere} habria q poder modificar orden de pedidos a preparar y politica del horno \unresolved
\item cambiar lo de requerir facturar, el sistema solo genera un archivo cuando el cajero se lo pide \tick
\item eliminar sistema de facturaci�n \tick
\item cu cambiar orden de preparaci�n \tick
\item cu de contingencia (diagrama aparte)
\item maestro carga tiempos manualmente como CU \tick
\item si cajero queda automatico, sacarlo de caso de USO

\end{itemize}

\end{document}
