\chapter{Introducci�n}
\section{Objetivo del documento}
%\textcolor{Blue}{Aqu� se espera una breve introducci�n con respecto a este informe.}

\section{Convenciones de notaci�n}

%\textcolor{Blue}{Describir aqu� cualquier convenci�n de notaci�n que se utilice en el presente documento, de manera de facilitar la lectura y comprensi�n del mismo por parte de los destinatarios. Notar que esta secci�n no se refiere a la sintaxis de las t�cnicas a utilizar, sino de definir las condiciones, siglas, simbolog�a o abreviaturas utilizadas espec�ficamente por los autores en el documento. 
%Por ej.}

%\textcolor{ForestGreen}{ 	CdC: En el presente documento figurar�n con la presente notaci�n los comentarios especiales a la c�tedra respecto del presente informe.}

\section{Destinatarios del documento}
%\textcolor{Blue}{Aqu� se espera una enumeraci�n de los stakeholders del software.}
%
%\textcolor{ForestGreen}{ CdC: Considerar que los clientes pueden contar con un �rea de Calidad de Software, que tambi�n podr�a  recibir el presente documento. Cualquier comentario espec�fico para lectura de los docentes deber� presentarse en el informe como dirigido a esta �rea.}

\section{Descripci�n del problema }
%\textcolor{Blue}{Breve descripci�n del problema al que se refiere el proyecto.}

\section{Documentos relacionados}

%\textcolor{Blue}{Se deber�an enumerar aqu� los diferentes documentos relacionados con el presente y que pueden ser de inter�s para el lector de este informe.}
%
%\textcolor{ForestGreen}{CdC: No se espera aqu� que detallen bibliograf�a sino documentaci�n de inter�s directo con el proyecto. Por ej., aqu� podr�an estar mencionadas las minutas de reuniones de relevamiento.}

\section{Organizaci�n del informe}
%\textcolor{Blue}{Breve descripci�n del documento, aclaraciones sobre c�mo se debe leer, etc, etc.}
